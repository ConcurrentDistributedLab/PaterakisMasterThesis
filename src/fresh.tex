\chapter{FreSh and DFreSh}
\label{chapter:FreSh}

\section{FreSh}
\label{seq:static:FreSh}

This section we present FreSh the first lock-free locality-aware data series index. 
FreSh follows the data processing flow described in Chapter~\ref{chapter:traverse-object} 
thus employing $\mathit{BC}$, $\mathit{TP}$, $\mathit{PS}$, and $\mathit{RS}$. It also
repeatedly apply \textit{ReFreSh} (from Chapter~\ref{chapter:Locality-aware}) to implement 
each of these traverse objects. It also uses a counter object which supports the operation
\NextIndex\ that returns a positive integer. The implementation of \NextIndex\ is provided
in Section~\ref{sec:counter-object} Algorithm~\ref{alg:counter}


\subsection{Buffers Creation and Tree Population} 
\label{sec:buffers_and_tree}

\BC\ is implemented using a single buffer, called \RawData.
In \BC, \Put\ is never used, as we assume that the data are initially in \RawData.
% 
\Traverse\ is implemented by employing \Refresh. 
We split \RawData\ into $k$ equally-sized {\em chunks} of consecutive 
elements , $w_1, \ldots, w_k$. This way we have a number of 
$k$ workloads. 
Threads use a counter object to get assigned chunks to process.
To reduce the cost of helping, \Fresh\ calls \Refresh\ recursively.
Specifically, it splits each chunk into smaller parts,
called {\em groups}, and employ \Refresh\ a second time for processing the groups of a chunk.
% 
In more detail, \Fresh\ maintains an additional counter object 
for each chunk of $RawData$. 
Each thread $t$ that acquires or helps a chunk, uses the counter object of the chunk to acquire
{\em groups} in the chunk to process. \Fresh\ also applies a third level of \Refresh\ recursion,
where each workload is comprised of the processing of just a single element of a group. 

Pseudocode for \BC.\Traverse\ is provided in Algorithm~\ref{alg:bc}. 
\RawData\ is comprised of $k$ chunks, each containing $m$ groups. Moreover, each group
contains $r$ elements (line~\ref{alg:bc:r}). 
\Fresh\ uses three sets of done flags, $\mathit{DChunks}$, $\mathit{DGroups}$,
and $\mathit{DElements}$ (line~\ref{alg:bc:c}), storing one done flag for each chunk,
for each group, and for each element, respectively.
% 
Similarly, \Fresh\ employs three sets of counter objects, $\mathit{Chunks}$, $\mathit{Groups}$,
and $\mathit{Elements}$ (line~\ref{alg:bc:e}), to count the chunks, groups and elements, assigned
to threads for processing. 
%
\Fresh\ also employs two sets of {\em helping} flags (line~\ref{alg:bc:h}), 
$\mathit{HChunks}$  (for helping chunks) and $\mathit{HGroups}$ (for helping groups). 
For each $1 \leq i \leq k$, $\mathit{HChunks[i]}$ identifies whether there are helpers for chunk $i$.
Similarly, for each $1 \leq j \leq m$, $\mathit{HGroups[i][j]}$ identifies whether there are helpers
for group $j$ of chunk $i$. Note that pseudocode for \textsc{BufferCreation} function is provided in 
Algorithm~\ref{alg:iSAXTraverse}.

When \BC.\Traverse\ is invoked \Traverse(\&\BufferCreation, \RawData, $\mathit{Dchunks}$, $\mathit{DGroups}$, $\mathit{DElements}$,
$\mathit{HChunks}$, $\mathit{HGroups}$, \False, $\mathit{Chunks}$, $\mathit{Groups}$, $\mathit{Elements}$, $1$),
$h$ is equal to \False. By the way a counter object works, it follows that 
no expeditive mode is ever executed at the first level of the recursion.
Note that at this level, the roles of $D_1$ and $H_1$ are played by the one-dimensional arrays
$DChunks$ and $HChunks$, respectively. Moreover, $DGroups$ and $DElements$ play the role of $D_2$ and $D_3$,
respectively, and $HGroups$ plays the role of $H_2$. Each chunk is processed by recursively calling
\Traverse\ ({\em level-2 recursion}) on line~\ref{alg:bc:recur} (with $\mathit{rlevel}$ being equal to $2$). 
The goal of a level-2 invocation of \Traverse\
is to process an entire chunk by splitting it into groups and calling
\Traverse\ once more ({\em level-3 recursion}) to process the elements of each group 
(recursive call of line~\ref{alg:bc:recur} with $\mathit{rlevel}$ being equal to $3$). 
Note that in a level-2 invocation corresponding to some chunk $i$, 
\RawData\ is the two-dimensional array containing the elements of the groups of chunk $i$.
Moreover, the role of $D_1$ is now played by the one-dimensional array $DGroups[i]$, 
and the role of $D_2$ by the two-dimensional array $DElements[i]$,
whereas $D_3$ is no longer needed and is $\mathit{NULL}$.
The role of $H_1$ is now played by the one-dimensional array $HGroups[i]$. 
Helping (lines~\ref{alg:bc:scan:for}-\ref{alg:bc:help:c:true}) follows the general pattern
described in Algorithm~\ref{alg:refresh}. 

%%%%%%%%%%%%%%%%%%%%%%%%%%%%%%%%%%%%%% PSEYDOCODE%%%%%%%%%%%%%%%%%%%%%%%%%%%%%%%%%%%%%%%%%%%%%%%%%%%%%%%%%%%%%%%%%%%%%%


\begin{algorithm}[t]
    \footnotesize
    \vspace*{2mm}

    \begin{algorithmic}[1]
    
    \State \textbf{Shared variables:}
    \State Set $\mathit{RawData}[1..k][1..m][1..r]$, initially containing all data series \label{alg:bc:r}
    \State \textbf{Boolean} $\mathit{DChunks}[1..k]$, $\mathit{DGroups}[1..k][1..m]$, $\mathit{DElements}[1..k][1..m][1..r]$, initially all \textbf{False} \label{alg:bc:c}
    \State \textbf{Boolean} $\mathit{HChunks}[1..k]$, $\mathit{HGroups}[1..k][1..m]$, initially all \textbf{False} \label{alg:bc:h}
    \State CounterObject $\mathit{Chunks}$, $\mathit{Groups}[1..k]$, $\mathit{Elements}[1..k][1..m]$ \label{alg:bc:e}
    \State \textbf{int} $\mathit{Size}[1..3] = \{k,m,r\}$

    \vspace*{1mm}
    \State \textbf{Code for each thread:}

    \Procedure{Traverse}{Function *BufferCreation, DataSeries $\mathit{RawData}[]$, Boolean $\mathit{D_1}[]$, Boolean $\mathit{D_2}[]$, Boolean $\mathit{D_3}[]$, Boolean $\mathit{H_1}[]$, Boolean $\mathit{H_2}[]$, Boolean $h$, CounterObject $Cnt_1$, CounterObject $Cnt_2[]$, CounterObject $Cnt_3[]$, int $\mathit{rlevel}$}
        \State \textbf{int} $\mathit{i}$
        \While{\textbf{True}} \label{alg:bc:while:start}
            \State $\langle i, * \rangle \gets Cnt_1.\NextIndex(\&h)$
            \If{$i > \mathit{Size}[rlevel]$} \textbf{break} \EndIf
            \State Mark $\mathit{RawData[i]}$ as acquired
            \If{$rlevel < 3$}
                \State Traverse(\textsc{BufferCreation}$\mathit{, RawData[i], D_2[i], D_3[i], D_3[i], H_2[i],}$
                \Statex \quad $\mathit{NULL, H_1[i], Cnt_2[i], Cnt_3[i], Cnt_3[i], rlevel+1}$) \label{alg:bc:recur}
            \Else
                \State $*$\Call{BufferCreation}{$RawData[i]$}
            \EndIf
            \State $\mathit{D_1[i]} \gets \textbf{True}$ \label{alg:bc:c:true}
        \EndWhile

        \ForAll{$j$ such that $\mathit{D_1[j]}$ is \textbf{False}} \label{alg:bc:scan:for}
            \State Backoff() \Comment{Avoid helping if possible} \label{alg:bc:help:backoff}
            \If{$\mathit{D_1[j]}$ is \textbf{False}} \label{alg:bc:help:if}
                \State $\mathit{H_1[j]} \gets \textbf{True}$ \label{alg:bc:h:true}
                \If{$rlevel < 3$}
                    \State Traverse(\textsc{BufferCreation}$\mathit{, RawData[j], D_2[j], D_3[j], D_3[j], H_2[j],}$
                    \Statex \quad $\mathit{NULL, H_1[j], Cnt_2[j], Cnt_3[j], Cnt_3[j], rlevel+1}$) \label{alg:bc:help:process}
                \Else
                    \State $*$\Call{BufferCreation}{$RawData[j]$}
                \EndIf
                \State $\mathit{D_1[j]} \gets \textbf{True}$ \label{alg:bc:help:c:true}
            \EndIf
        \EndFor
    \EndProcedure

    \end{algorithmic}

    \caption{Pseudocode for BC \textsc{Traverse} in \textsc{FreSh}. Code for thread $t$.}
    \label{alg:bc}
\end{algorithm}

The backoff time in \Fresh\ (see line~\ref{alg:bc:help:backoff}) depends on the average
execution time required by each
thread to process a group. Each thread $\mathit{t}$ counts the average time 
$T_{avg}$ it has spent to process all the parts it acquired, and whenever
it encounters a group to help, it sets the backoff time to be proportional
to $T_{avg}$ and performs helping only after backoff, if it is still needed.
This allows an owner thread that has not crashed, to finish the processing of its last chunk,
while it still executes in expeditive mode, thus avoiding the cost of switching
to and executing a standard mode.
 
\Fresh\ implements \TP\ using a set of $2^w$ summarization buffers
($w$ is the number of segments of an iSAX summary), one for each bit sequence of $w$ bits.
To decide to which summarization buffer to store a pair,
\Fresh\ (as other iSAX-based indexes) 
examines the bit sequence consisting of the first bit of each 
of the $w$ segments of the pair's iSAX summary, 
and places the pair into the corresponding summarization buffer. 
Each of the summarization buffers is split into $N$ parts, one for each of the $N$ 
threads in the system. Each thread uses its own part in each buffer to store the
elements it inserts. 

Algorithm~\ref{alg:sb} presents pseudocode that implement the \TP.Put operation,
which insert elements into the summarization buffers. Each thread uses
its own part in each summarization buffer to store the elements it inserts.
Since threads work independently without any need for synchronization,  
the pseudocode for the standard mode is the same as that for the expeditive mode. 
Each thread $t$ maintains a private variable, called $\mathit{pos}$, which it
exclusively reads and updates.
\textit{SBInsert} procedure takes the buffer index $j$ as a parameter and
inserts the new element into $\mathit{B[j][t][pos]}$. 

\begin{algorithm}[t]
    \footnotesize
    \vspace*{2mm}
    
    \begin{algorithmic}[1]
    
    \Procedure{InitializeSharedVariables}{}
        \State Data $\mathit{SB}[0...2^w-1][0..N-1]$, initially $\langle <\bot,\ldots,\bot>, \ldots, <\bot,\ldots,\bot> \rangle$
    \EndProcedure

    \vspace*{1mm}
    \State{Code for each thread $t_i$, $i \in \{0, \ldots, N-1\}$:}    
    \vspace*{1mm}
    

    \State Integer $\mathit{pos}[0...2^w-1]$, initially  $\langle <0, \ldots, 0 >\rangle$
    
    \Procedure{SInsert}{Data $\mathit{data}$, BufferID $j$}
        \State $\mathit{B[j][i][pos]} \gets \mathit{data}$ \Comment{Store data in buffer} 
        \State $\mathit{pos[j]} \gets \mathit{pos[j]} + 1$ \Comment{Increment position} 
    \EndProcedure
    
    \end{algorithmic}
    
    \caption{TP.PUT: Insert in summarization  buffers.}
    \label{alg:sb}
\end{algorithm}


\noindent
{\bf Tree Population Stage.}
Similarly to the buffers creation stage, in tree population, the worker threads
have to traverse and process the elements of \TP, i.e. all pairs added in the summarization
buffers. Processing is now achieved by calling the \TreePopulation\ function 
(Algorithm~\ref{alg:iSAXTraverse}) for each pair. \TreePopulation\ finds the right subtree
of the index tree to place each pair, and then simply calls \PS.\Put\ to add the pair into
\PS, the next traverse object in the dataflow pipeline (see Algorithm~\ref{alg:iSAXTraverse}). 
To implement TP.\Traverse, we split the elements of \TP\ into $2^w$ workloads 
as the number of summarization buffers, and apply \Refresh.
Each thread $\mathit{t}$ repeatedly acquires summarization buffers using \FAI,
and process them to produce the corresponding trees. 
Each summarization buffer could be further split into chunks and groups, and
\Refresh\ could be called recursively.
Pseudocode for \Traverse\ of \TP\ closely follows that for \BC.
\BC\ and \TP\ are lock-free implementations of a traverse object. 

\subsection{Prunning}

In \Fresh, \PS\ is implemented as a forest of $2^w$ leaf-oriented trees,
one for each of the summarization buffers. 
The trees of the forest are the root subtrees of a standard iSAX-based tree.
\TreePopulation\ simply transfers the pairs from a summarization buffer
to the appropriate subtree of the index tree. 
To support the concurrent population of a subtree by multiple threads, 
\Fresh\ utilizes Algorithm~\ref{alg:tree} 
%
To implement \PS.\Traverse, \Fresh\ (Algorithm ~\ref{alg:ps}) uses \Refresh\ to process 
the different subtrees of the index tree. Specifically, 
each thread $t$ access a counter (using getIndex) to get assigned a subtree $T$ to process.
To process the nodes of $T$, \Refresh\ is applied recursively.
A thread $t$ that is assigned node $i$ of $T$,
first searches for the $i$-th unprocessed node, according to preorder,
and then processes it by invoking the \Prunning\ function 
of Algorithm~\ref{alg:iSAXTraverse}. 

%%%%%%%%%%%%%%%%%%%%%%%%%%%%%%%%%%%% Pseudocode for PRUNNING %%%%%%%%%%%%%%%%%%%%%%%%%%%%%%%%%%%%%%%
\begin{algorithm}[htbp]
    \footnotesize
    \vspace*{2mm}
    
    \begin{algorithmic}[1]
    
    \State \textbf{Shared variables:}
    \State TreeNode *$\mathit{IndexTree}[1..2^w]$ \label{alg:ps:r}
    \State \textbf{bool} $\mathit{DTree[1..2^w]}$, $\mathit{HTree[1..2^w]}$, initially all \textbf{false} \label{alg:ps:c}
    \State CounterObject $\mathit{TreeCnt[1..2^w]}$
    
    \vspace*{1mm}
    \Procedure{Traverse}{Function *\Prunning, TreeNode *$T$, CounterObject *$\mathit{Cnt}$, 
        int $x$, bool $h$, int $\mathit{rlevel}$}
        \State int $\mathit{i}$
    
        \While{\textbf{true}}
            \State $\mathit{\langle i, * \rangle = Cnt.\NextIndex(\&h)}$ \label{alg:ps:help:c:cnt}
            \If{$\mathit{i > x}$}
                \State \textbf{break}
            \EndIf
            \If{$\mathit{rlevel < 2}$}
                \State \textbf{mark} $\mathit{IndexTree[i]}$ as acquired
                % \State $\mathit{totalNds} \gets$ \Call{TotalNodes}{$\mathit{IndexTree[i]}$}
                \State Traverse($\mathit{Prunning, IndexTree[i], TreeCnt[i], x,}$
                \Statex \quad $\mathit{\textbf{false}, rlevel+1}$) \label{alg:ps:help:c:rec}                
                \State $\mathit{DTree[i] := \textbf{true}}$ \label{alg:ps:help:c:true}
            \Else
                % \State $\mathit{nd} \gets$ \Call{FindNode}{$\mathit{T,i}$} \label{alg:ps:findnode}
                % \State \textbf{mark} $\mathit{nd}$ as acquired
                \State \Call{\Prunning}{$\mathit{T}$} \label{alg:ps:prunning}
                \State Traverse($\mathit{Prunning, T \rightarrow left, TreeCnt[i], x,}$
                \Statex \quad $\mathit{\textbf{false}, rlevel}$) \label{alg:ps:rec:left-child}
                \State Traverse($\mathit{Prunning, T\rightarrow right, TreeCnt[i], x,}$
                \Statex \quad $\mathit{\textbf{false}, rlevel}$) \label{alg:ps:rec:right-child}
                \State \textbf{mark} $\mathit{T}$ as done
            \EndIf
        \EndWhile
    
        \ForAll{$j$ such that $\mathit{DTree[j]}$ is \textbf{false}} \label{alg:ps:scan:for}
            \State \Call{Backoff}{} \Comment{Avoid helping, if possible} \label{alg:ps:help:backoff}
            \If{$\mathit{DTree[j]} == \textbf{false}$} \label{alg:ps:help:if}
                \State $\mathit{HTree[j] := \textbf{true}}$ \label{alg:ps:h:true}
                \State \Call{HelpTree}{\Prunning, $\mathit{IndexTree[j]}$}
            \EndIf
            \State $\mathit{DTree[j] := \textbf{true}}$ \label{alg:ps:help:c:true:helping}
        \EndFor
    \EndProcedure
    
    \vspace*{1mm}
    % \Procedure{FindNode}{TreeNode *$T$, int $i$} \Comment{Returns TreeNode*} \label{alg:ps:help:c:treenode-start}
    %     \State TreeNode *$\mathit{p} \gets T$
    %     \State int $\mathit{nds} \gets 0$
    
    %     \While{$\mathit{p \neq NULL\ \And\ nds \neq i}$}
    %         \If{$\mathit{nds + p \rightarrow cnt + 1 < i}$}
    %             \State $\mathit{nds} \gets \mathit{nds + p \rightarrow cnt + 1}$
    %             \State $\mathit{p} \gets p \rightarrow rc$
    %         \Else
    %             \State $\mathit{p} \gets p \rightarrow lc$
    %         \EndIf
    %     \EndWhile
    %     \State \Return $\mathit{p}$ \label{alg:ps:help:c:treenode-end}
    % \EndProcedure
    
    \vspace*{1mm}
    \Procedure{HelpTree}{Function *$f$, TreeNode *$T$}
        \If{$T == NULL$}
            \State \Return
        \EndIf
        \State \Call{HelpTree}{$f$, $T \rightarrow lc$}
        \If{$*T$ is unprocessed}
            \State $*f(*T)$
        \EndIf
        \State \Call{HelpTree}{$f$, $T \rightarrow rc$}
    \EndProcedure
    
    \end{algorithmic}
    
    \caption{Pseudocode for \Put\ and \Traverse\ of \PS\ in \Fresh. Code for thread $t \in \{ 1, \ldots, N-1\}$.}
    \label{alg:ps}
    \end{algorithm}


    \subsubsection{Insert in a Leaf-Oriented Tree}
    \label{sec:leaf-oriented}
    Each node of the tree stores a key and the pointers
    to its left and right children.  (refer to Algorithm~\ref{alg:tree}).
    A leaf node stores additionally an array $D$, where the leaf's data are stored. 
    We assume that each data item is a pair containing a key and the associated information.
    A node may have its own key. For instance, in iSAX-based indexes, this key is the node's
    iSAX summary which summarizes all data series stored in it.
    The proposed implementation allows multiple insert operations to concurrently update array
    $D$ of a leaf. This results in enhanced parallelism and performance. 
    To achieve this, each leaf $\ell$ contains a counter, called $\mathit{Elements}$.
    % 
    Each thread $\mathit{t}$ that tries to insert data in $\ell$, uses
    $\mathit{Elements}$ to acquire a position $\mathit{pos}$ in the array $D$ of $\ell$.
    If $D$ is not full, (line~\ref{alg:tree:pos-in-D}) (i.e. $\mathit{pos} < M$), 
    $\mathit{t}$ stores the new element in $D[\mathit{pos}]$ (line~\ref{alg:tree:store-in-D}).
    Otherwise, in case the array is full, i.e. $\mathit{pos \geq M}$, then 
    $\mathit{t}$ attempts to split the leaf (line~\ref{alg:tree:pos-not-in-D}). 
    % 
    During spliting, $D$ may contain empty positions, since some
    threads may have acquired positions in $D$ but have not yet stored their
    elements there (line~\ref{alg:tree:store-in-D}).
    To avoid situations of missing elements, each leaf contains an $\mathit{Announce}$ array 
    with one position for each thread. A thread announces its operation
    in $\mathit{Announce}$ before it attempts to acquire a position in $D$.
    During spliting, a thread distributes to the new leaves it creates not only the
    elements found in $D$ but also those in $\mathit{Announce}$.

    %%%%%%%%%%%%%%%%%%%%%%%%%%%%%%Tree Insert Code%%%%%%%%%%%%%%%%%%%%%%%%%%%%%%%
    \begin{algorithm}[htbp]
        \footnotesize
        \caption{Type Definitions for the Lock-Free Tree}
        \label{alg:tree-types}
        \begin{algorithmic}[1]
            \Statex \textbf{Type Definitions:}
            \State \textbf{Type} Node:
            \State \quad int $\mathit{key}$
            \State \quad $\{Node,Leaf\}$ *$\mathit{left}$
            \State \quad $\{Node,Leaf\}$ *$\mathit{right}$
            \State \quad InsertRec $\mathit{Announce}[0..n-1]$
            \State \quad \textbf{Boolean} $\mathit{helpersExist}$
    
            \Statex
            \State \textbf{Type} InsertRec:
            \State \quad Data $\mathit{data}$
            \State \quad int $\mathit{position}$
    
            \Statex
            \State \textbf{Type} Leaf \textbf{extends} Node:
            \State \quad Data $D[0..m-1]$
            \State \quad CounterObject Elements
        \end{algorithmic}
    \end{algorithm}
    


    \begin{algorithm}[htbp]
        \footnotesize
        \caption{TraverseTree: a lock-free leaf-oriented tree with fat leaves, implementing a traverse object.
         Code for thread $t \in \{ 1, \ldots, N-1\}$.}
        \label{alg:tree}
        \begin{algorithmic}[1]
    
            \Statex \textbf{Shared Variables:}
            \State $\mathit{Tree} \gets \NULL$ \Comment{Initially points to a Leaf with initialized values}
    
            \Procedure{TreeInsert}{$\mathit{data}, \mathit{isHelper}$}
                \State $\mathit{leaf} \gets$ \NULL
                \State $\mathit{parent} \gets \mathit{Tree}$
                \State $\mathit{ptr} \gets$ \NULL
                \State $\mathit{pos}, \mathit{val} \gets 0$
                \State $\mathit{expeditive} \gets$ \False
    
                \While{\True} \label{alg:tree:while}
                    \State $\langle \mathit{leaf, parent} \rangle \gets$ \Call{Search}{$\mathit{data, parent}$} \label{alg:tree:search}
                    \If{$\mathit{parent} = \NULL$} \label{alg:tree:ptr-parent:null}
                        \State $\mathit{ptr} \gets \&\mathit{Tree}$
                    \ElsIf{$\mathit{parent} \rightarrow \mathit{left} = \mathit{leaf}$}
                        \State $\mathit{ptr} \gets \&\mathit{parent} \rightarrow \mathit{left}$ \label{alg:tree:ptr-left}
                    \Else
                        \State $\mathit{ptr} \gets \&\mathit{parent} \rightarrow \mathit{right}$ \label{alg:tree:ptr-right}
                    \EndIf
    
                    \If{$\mathit{isHelper} = \True$ \textbf{and} $\mathit{leaf} \rightarrow \mathit{helpersExist} = \False$}
                        \State $\mathit{leaf} \rightarrow \mathit{helpersExist} \gets \True$ \label{alg:tree:switch-mode:}
                    \EndIf
    
                    \State $\langle \mathit{pos, expeditive} \rangle \gets$ \Call{NextIndex}{$\& \mathit{leaf} \rightarrow \mathit{helpersExist}$} \label{alg:tree:get-pos}
    
                    \If{$\mathit{expeditive} = \False$}
                        \State $\mathit{leaf} \rightarrow \mathit{Announce}[t] \gets \langle \mathit{data, \bot} \rangle$ \label{alg:tree:announce:op}
                    \EndIf
    
                    \If{$\mathit{pos} < M$} \label{alg:tree:pos-in-D}
                        \If{$\mathit{expeditive} = \False$}
                            \State $\mathit{leaf} \rightarrow \mathit{Announce}[t].\mathit{position} \gets \mathit{pos}$ \label{alg:tree:announce:pos-in-D}
                        \EndIf
                        \State $\mathit{leaf} \rightarrow \mathit{D}[\mathit{pos}] \gets \mathit{data}$ \label{alg:tree:store-in-D}
                    \Else \label{alg:tree:pos-not-in-D}
                        \State \Call{SplitLeaf}{$\mathit{leaf}, \mathit{ptr}, \mathit{expeditive}$}
                    \EndIf
    
                    \If{$(*\mathit{ptr}) \rightarrow \mathit{helpersExist} = \True$} \label{alg:tree:st:finish}
                        \If{$\mathit{expeditive} = \False$ \textbf{and} $(*\mathit{ptr}) \rightarrow \mathit{Announce}[t].\mathit{position} \neq \bot$} \label{alg:tree:op-is-applied}
                            \State $(*\mathit{ptr}) \rightarrow \mathit{Announce}[t] \gets \langle \bot, \bot \rangle$ \label{alg:tree:clean}
                        \Else
                            \State \textbf{continue} \label{alg:tree:re-attempt}
                        \EndIf
                    \EndIf
                    \State \Return
                \EndWhile
            \EndProcedure
    
            \Procedure{SplitLeaf}{$\mathit{leaf}, \mathit{prt}, \mathit{expeditive}$}
                \State $\mathit{newNode} \gets$ \Call{NewNode}{} \label{alg:tree:split:new-internal-node}
                \State $\mathit{newNode} \rightarrow \mathit{left} \gets$ \Call{NewLeaf}{}
                \State $\mathit{newNode} \rightarrow \mathit{right} \gets$ \Call{NewLeaf}{} \label{alg:tree:split:new-right-child}
                \State $\mathit{splitBuffer} \gets \emptyset$
    
                \If{$\mathit{expeditive} = \False$} \label{alg:tree:em:Announce-scan}
                    \For{$i \in \{0, \dots, n-1\}$ \textbf{where} $\mathit{leaf} \rightarrow \mathit{Announce}[i].\mathit{data} \neq \bot$} \label{alg:tree:Announce-scan}
                        \State $\mathit{ldata} \gets \mathit{leaf} \rightarrow \mathit{Announce}[i].\mathit{data}$
                        \If{$\mathit{leaf} \rightarrow \mathit{Announce}[i].\mathit{position} \neq \bot$} \label{alg:tree:announce-scan:pos-not-bot}
                            \State $\mathit{leaf} \rightarrow \mathit{D}[\mathit{leaf} \rightarrow \mathit{Announce}[i].\mathit{position}] \gets \mathit{ldata}$ \label{alg:tree:announce-scan:store-in-D}
                        \Else
                            \State \Call{AddToBuffer}{$\mathit{ldata}, \mathit{splitBuffer}$} \label{alg:tree:announce-scan:data-copy}
                        \EndIf
                        \State $\mathit{newNode} \rightarrow \mathit{Announce}[i] \gets \langle \mathit{ldata}, -1 \rangle$  \label{alg:tree:announce-scan:mark-op-applied}
                    \EndFor
                \EndIf
    
                \State \Call{Distribute}{$\mathit{leaf} \rightarrow \mathit{D}, \mathit{splitBuffer}, \mathit{newNode} \rightarrow \mathit{left}, \mathit{newNode} \rightarrow \mathit{right}$} \label{alg:tree:split}
                \State \Call{CAS}{$*\mathit{prt}, \mathit{leaf}, \mathit{newNode}$} \label{alg:tree:CAS}
            \EndProcedure
    
        \end{algorithmic}
    \end{algorithm}
    
    More specifically, a thread $t$ executing \TreeInsert\ repeatedly executes the following actions. 
    It first calls a standard \textit{Search} routine to traverse a path of the tree and find an appropriate
    $\mathit{leaf}$ and its $\mathit{parent}$ (line~\ref{alg:tree:search}).
    Pointer $\mathit{ptr}$ is a reference to the appropriate child field of $\mathit{parent}$,
    which needs to be changed to perform \TreeInsert  (lines~\ref{alg:tree:ptr-parent:null}-\ref{alg:tree:ptr-right}).
    Next, $t$ accesses the counter object to acquire a position in $D$ (line~\ref{alg:tree:get-pos}).
    If the execution mode is expeditive and the leaf is not full, the thread directly inserts
    the element into the tree at the position assigned by the counter object. Otherwise, in standard mode, the
    thread first announces the data it intends to in the tree (line~\ref{alg:tree:announce:op}).
    After making this announcement, the thread records its assigned position in $\mathit{Announce}$ and
    stores the data in $D[\mathit{pos}] $(line~\ref{alg:tree:store-in-D}), provided that $D$ is not full 
    (line~\ref{alg:tree:pos-in-D}). If $D$ is full, it calls \SplitLeaf\ to split $\mathit{leaf}$.
    
    During a split operation, the thread responsible for this task locally creates three new
    nodes: an internal node and its left and right children  
    (lines~\ref{alg:tree:split:new-internal-node}-\ref{alg:tree:split:new-right-child}).  
    In standard execution mode, to ensure that no elements are lost during the split,  
    the newly created leaves must contain both the elements from the current leaf being split  
    and any elements recorded in the announce array.  
    To achieve this, the thread employs a temporary buffer, \textit{splitBuffer}, which is
    initially empty. For each announced element whose position remains unknown, the thread
    places it into \textit{splitBuffer}. Conversely, if the position has already been
    determined, the thread updates the corresponding location in the original leaf
    (lines~\ref{alg:tree:Announce-scan}-\ref{alg:tree:announce-scan:mark-op-applied}).  
    After processing all announced elements, the thread redistributes the elements from
    \textit{splitBuffer}, along with those from the original leaf, to the newly created left
    and right child nodes. Subsequently, the thread attempts to integrate the new internal
    node into the tree using a \CAS\ operation.  

    Assume that the owner thread $t$ calls \TreeInsert,
    reaches a leaf $\mathit{l}$, and acquires the last valid position in array $D$ of $\mathit{l}$. 
    Thread $t$ executes in expeditive mode (so it does not announce its data),
    and before it records its data in $D$, it becomes slow. Next, a helper thread $t'$ reaches $\mathit{l}$, switches 
    $\mathit{l}$'s execution mode to standard, and splits $\ell$ (executing on standard mode). 
    During this split, $t'$ will not take into 
    consideration the data of $t$, since $t$ neither has announced its operation
    (since $t$ was executing in expeditive mode), nor has yet written its data into $D$. 
    To disallow thread $t$ from finishing its operation without inserting its data, \TreeInsert\ provides the 
    following mechanism (lines~\ref{alg:tree:st:finish}-\ref{alg:tree:re-attempt}). Before it terminates, 
    thread $t$ re-reads the appropriate child field of the parent of $\ell$ (through $\mathit{ptr}$) 
    and checks the $\mathit{helpersExist}$ flag of the node $nd$ that $\mathit{ptr}$ points to,
    to figure out whether it can still operate on expeditive mode. In the scenario above, 
    $nd$ will be the node that $t'$ has allocated to replace $\ell$, and thus it has its 
    $\mathit{helpersExist}$ flag equal to \True\ (line~\ref{alg:tree:split:new-internal-node}). 
    This way, $t$ discovers that the execution mode for $\mathit{l}$ has changed
    (line~\ref{alg:tree:st:finish} and first condition of line~\ref{alg:tree:op-is-applied}),
    and re-attempts its \Insert\ (line~\ref{alg:tree:re-attempt}).
        
    \begin{lemma}
    \label{lem:tree}
    Algorithm~\ref{alg:tree} is a {\em linearizable, lock-free} implementation of a leaf-oriented
    tree with fat leaves, supporting only insert operations.
    \end{lemma}

    %%%%%%%%%%%%%%%%%%%%%%%%%%%%% Refinement %%%%%%%%%%%%%%%%%%%%%%%%%%%%%%%%%%%
    \subsection{Refinement}

    To implement \RS, \Fresh\ uses a set of priorities queues
    each implemented using an array ( Algorithm~\ref{alg:pq}). 
    A thread inserts elements in all arrays in a round-robin fashion. 
    This technique results in almost equally-sized arrays, which is crucial
    for achieving load-balancing. 
    
    \begin{algorithm}[htbp]
        \footnotesize
        \caption{Priority Queue of \Fresh. Code for thread $t$.}
        \label{alg:pq}
        \begin{algorithmic}[1]
    
            \State \textbf{Shared variables:}
            \State $\mathit{\langle int, Data \rangle}$ $A[0..k-1]$, initially all $\langle \bot, \bot \rangle$
            \State CounterObject $\mathit{Cnt}$, initially $0$
            \State Boolean $\mathit{helpersExist}$, initially $\False$
            \State int $\mathit{insPos}$, initially $0$
            \State $\mathit{\langle int, Data \rangle}$ *$\mathit{SA}$, initially $\NULL$
    
            \vspace{2mm}
            
            \Procedure{Insert}{int $\mathit{priority},Data \mathit{values}$}
                \State int $\mathit{pos} \gets \FAI(\mathit{insPos})$ \label{alg:pq:ins:FAI}
                \State $\mathit{A[pos]} \gets \langle \mathit{priority, value} \rangle$ \label{alg:pq:ins:store}
            \EndProcedure
    
            \vspace{2mm}
    
            \Procedure{InitDeletePhase}{}
                \State $\mathit{\langle int, Data \rangle}$ *$\mathit{sa}$
                \State $\mathit{sa} \gets$ allocate local array of $insPos$ elements
                \State Copy non-$\bot$ elements of $A$ into $\mathit{sa}$ and sort them \label{alg:sa:local-copy}
                \State $\mathit{\CAS(\&SA, \NULL, sa)}$ \label{alg:sa:CAS}
            \EndProcedure
    
            \vspace{2mm}
    
            \Function{DeleteMin}{boolean $ \mathit{isHelper}$} \textbf{returns} $\mathit{Data}$
                \If{$\mathit{isHelper = \True}$ \textbf{and} $\mathit{helpersExist = \False}$}
                    \State $\mathit{helpersExist} \gets \True$
                \EndIf
                \State int $\mathit{pos} \gets \mathit{Cnt.\NextIndex(\&helpersExist})$ \label{alg:pq:deleteMin:next}
                \If{$\mathit{pos} \geq insPos$}
                    \State \Return $\bot$
                \EndIf
                \State \Return $\mathit{SA[pos].data}$
            \EndFunction
    
        \end{algorithmic}
    \end{algorithm}
    
    
    During query answering, an application may use the same index tree
    to answer more than one query. In that case, the done flags of the nodes and other
    variables need to be reset each time a new query starts. This may require syncrhonization. 
    To avoid this case, \Fresh\ implements the {\em done} flag as a counter (rather than as a boolean). 
    This counter describes the number of queries for which the node has been processed.
    
    To implement \RS.\Traverse, \Fresh\ first comes up with sorted versions of the arrays,
    shared to all threads. Then, it uses \Refresh\ to assign sorted arrays to threads 
    for processing. To process the elements of a sorted array $SA$, \Refresh\ is
    applied recursively. 
    Processing of an array element is performed by invoking the \Refinement\ function
    (Algorithm~\ref{alg:iSAXTraverse}). Helping is done at the level of 
    a) each individual priority queue and 
    2) the set of priority queues, in a way similar to that in \PS.
    \RS\ is a linearizable lock-free implementation of a traverse object. 
    
    To update BSF, \Fresh\ repeatedly reads the current value $y$ of BSF, and attempts to
    atomically change it from $v$ to the new value $v'$, using \CAS, until it either succeeds or some value 
    smaller than or equal to $v'$ is written in BSF.
    
    \begin{lemma}
    \label{lem:rs}
    {\bf (1)} \RS\ is a linearizable lock-free implementation of a traverse object that supports \Put\ and
    \Traverse($\mathit{*, *, 0}$). 
    {\bf (2)} For every thread $t$, when an invocation of \Traverse(\&$\mathit{\Prunning, *, \True}$)\ by $t$ on \RS\ completes, 
    for every leaf $\ell$ in $\RS$, $\ell$ either has been processed or it has been pruned.
    \end{lemma}
    
    \begin{theorem}
    \label{thm:qa}
    \Fresh\ solves the 1-NN problem and provides a lock-free implementation of \QueryAnswering\
    (Algorithm~\ref{alg:iSAXTraverse}). 
    \end{theorem}