\chapter{FreSh}
\label{chapter:FreSh}

We follow the data processing flow, described in Chapter~\ref{chapter:traverse-object},
employ $\mathit{BC}$, $\mathit{TP}$, $\mathit{PS}$, and $\mathit{RS}$ (from Section~\ref{chapter:traverse-object}),
and repeatedly apply \textit{ReFreSh} (from Section~\ref{chapter:Locality-aware}) to come up with \textit{FreSh},
the first lock-free locality-aware data series index.

\section{Counter Object}

\rb{Not sure if we have to include this!!!! Found it on old versions}


Many of our implementations use a {\em counter object}, which supports the
operation \NextIndex\ (Algorithm~\ref{alg:counter}) which returns a positive integer. 
Let \NextIndex\ be invoked $K$ times.
If crashes do not occur, every index in the range $R = 
\{0, \ldots, K-1\}$ should be returned exactly once; 
(i.e. by a distinct of the $K$ invocations) each integer 
is returned exactly once otherwise, as many indexes from $R$,
as the number of crashed  threads, 
are not returned, but the rest are returned exactly once.

Algorithm~\ref{alg:counter} uses two counters. The first, $\mathit{cntEM}$,
is used in the expeditive mode and is updated by simple writes
(line~\ref{alg:co:em:inc-elements}),
whereas the second, $\mathit{cntSM}$, is used in the standard mode and
is updated using \FAI\ (line~\ref{alg:co:FAI-pos}).
As long as no helpers exist ($\mathit{h = \False}$), the owner thread $t$ executes 
a code similar to the sequential code implementing a counter (lines~\ref{alg:co:em:acquire-pos}-
\ref{alg:co:em:checkh1}). When a helper arrives, it sets $\mathit{cntSM}$ to the current
value of $cntEM$ (lines~\ref{alg:co:em:ownerFlag}, \ref{alg:co:em:cntSM-init}),
and uses \FAI\ on $\mathit{cntSM}$ (line~\ref{alg:co:FAI-pos}) to get assigned new indexes. 

Lines~\ref{alg:co:em:elements-bot}-\ref{alg:co:FAI-pos} provide a mechanism for appropriately
synchronizing $t$ with the helper threads when they arrive. Assume that $t$ has read the
value $x$ in $\mathit{cntEM}$, and it is ready to write $x+1$ in $\mathit{cntEM}$,
when two helper threads $t_h$ and $t'_h$ arrive.
Thread $t_h$ reads $x$ in $\mathit{cntEM}$, then $t$ writes $x+1$ 
in $\mathit{cntEM}$ and $t'_h$ reads  $x+1$ in $\mathit{cntEM}$.
Depending on which of the two values will be written in $\mathit{cntSM}$
and the order in which the \FAI\ instructions will be executed, 
one of $t_h$, $t'_h$ may read $x$ in $\mathit{cntSM}$ and return it.
% 
In this case, $t$ should avoid returning $x$ a second time (to ensure the semantics of
a counter object). To achieve this, helpers record in $\mathit{first}$ the value with
which they initialize $\mathit{cntSM}$. 
% 
If $t$ discovers that helpers exist (lines~\ref{alg:co:em:checkh1},~\ref{alg:co:em:ownerFlag}), 
%is the thread to successfully execute the \CAS\ of line~\ref{27}, it stores
%the value $x+1$ there, so helper threads will return this or higher values. 
%In this case, it is safe for $t$ to return $x$ (line~\ref{28}).
%Otherwise, $t$ 
it may have to read $\mathit{first}$ to figure out which is the first value recorded
in $\mathit{cntSM}$. If it is $x$, it  retries
(line~\ref{alg:co:em:continue2}). Note that helpers should first agree on the value 
to record in $\mathit{cntSM}$ with the \CAS\ of line~\ref{alg:co:em:casfirst} and then all 
use this value to initialize $\mathit{cntSM}$ (lines~\ref{alg:co:em:pos2}-~\ref{alg:co:em:cntSM-init}).


\begin{algorithm}[t]
    \footnotesize
    \caption{Pseudocode for \NextIndex.}
    \label{alg:counter}
    \begin{algorithmic}[1]

        \State \textbf{Shared variables:}
        \State int $\mathit{cntEM}$ (initially $0$), $\mathit{cntSM}$ (initially $\bot$)

        \vspace{2mm}

        \Function{NextIndex}{$\mathit{*h}$} \textbf{returns} $\langle \mathit{int, \Boolean} \rangle$
            \State int $\mathit{pos}$
            \State int $\mathit{ownerFlag} \gets \False$
            
            \While{\True}
                \If{$\mathit{*h = \False}$} \label{alg:co:em}
                    \State $\mathit{pos} \gets \mathit{cntEM}$ \label{alg:co:em:acquire-pos}
                    \If{$\mathit{*h == \True}$} 
                        \State \textbf{continue} \label{alg:co:em:continue1}
                    \EndIf
                    \State $\mathit{cntEM} \gets \mathit{cntEM}+1$ \label{alg:co:em:inc-elements}
                    \If{$\mathit{*h == \False}$} 
                        \State \Return $\langle pos, \True \rangle$ \label{alg:co:em:checkh1}
                    \EndIf
                    \State $\mathit{ownerFlag} \gets \True$ \label{alg:co:em:ownerFlag}
                \EndIf

                \If{$\mathit{cntSM = \bot}$} \label{alg:co:em:elements-bot}
                    \If{$\mathit{ownerFlag \neq \True}$} \label{alg:co:em:ifHelper}
                        \State $\mathit{pos} \gets \mathit{cntEM}$ \label{alg:co:em:ownerFlag1}
                        \State $\mathit{\CAS(first, \bot, pos)}$ \label{alg:co:em:casfirst}
                        \State $\mathit{pos} \gets first$ \label{alg:co:em:pos2}
                    \EndIf
                    \State $\mathit{v} \gets \CAS(cntSM, \bot, pos)$ \label{alg:co:em:cntSM-init}
                    \If{$\mathit{v == \bot}$ \textbf{and} $\mathit{ownerFlag = \True}$}
                        \State \Return $\langle pos, \False \rangle$ \label{alg:co:em:ifOwner}
                    \ElsIf{$\mathit{first < cntEM}$} 
                        \State \textbf{continue} \label{alg:co:em:continue2}
                    \EndIf
                \EndIf

                \State $\mathit{pos} \gets \FAI(\mathit{cntSM})$ \label{alg:co:FAI-pos}
                \State \Return $\langle pos, \False \rangle$ \label{alg:co:em:return}
            \EndWhile

        \EndFunction

    \end{algorithmic}
\end{algorithm}



\section{Buffers Creation and Tree Population} 
\label{sec:buffers_and_tree}

\BC\ is implemented using a single buffer, called \RawData.
In \BC, \Put\ is never used, as we assume that the data are initially in \RawData.
% 
To implement \\
\Traverse, we employ \Refresh.  
We split \RawData\ into $k$ equally-sized {\em chunks} of consecutive 
elements , $w_1, \ldots, w_k$. This way we have a number of 
$k$ workloads. 
Threads use a counter object to get assigned chunks to process.
To reduce the cost of helping, \Fresh\ calls \Refresh\ recursively.
Specifically, it splits each chunk into smaller parts,
called {\em groups}, and employ \Refresh\ a second time for processing the groups of a chunk.
% 
In more detail, \Fresh\ maintains an additional counter object 
for each chunk of $RawData$. 
Each thread $t$ that acquires or helps a chunk, uses the counter object of the chunk to acquire
{\em groups} in the chunk to process. \Fresh\ also applies a third level of \Refresh\ recursion,
where each workload is comprised of the processing of just a single element of a group. 

Pseudocode for \BC.\Put\ and \BC.\Traverse\ is provided 
in Algorithm~\ref{alg:bc}. \\ \RawData\ is comprised of $k$ chunks,
each containing $m$ groups. Moreover, each group contains $r$ elements (line~\ref{alg:bc:r}). 
\Fresh\ uses three sets of done flags, $\mathit{DChunks}$, $\mathit{DGroups}$,
and $\mathit{DElements}$ (line~\ref{alg:bc:c}), storing one done flag for each chunk,
for each group, and for each element, respectively.
% 
Similarly, \Fresh\ employs three sets of counter objects, $\mathit{Chunks}$, $\mathit{Groups}$,
and $\mathit{Elements}$ (line~\ref{alg:bc:e}), to count the chunks, groups and elements, assigned
to threads for processing. 
%
\Fresh\ also employs two sets of {\em helping} flags (line~\ref{alg:bc:h}), 
$\mathit{HChunks}$  (for helping chunks) and $\mathit{HGroups}$ (for helping groups). 
For each $1 \leq i \leq k$, $\mathit{HChunks[i]}$ identifies whether there are helpers for chunk $i$.
Similarly, for each $1 \leq j \leq m$, $\mathit{HGroups[i][j]}$ identifies whether there are helpers
for group $j$ of chunk $i$. 

% 
In an invocation of
\Traverse(\&\BufferCreation, \RawData, $\mathit{Dchunks}$, \\ $\mathit{DGroups}$, $\mathit{DElements}$,
$\mathit{HChunks}$, $\mathit{HGroups}$, \False, $\mathit{Chunks}$, $\mathit{Groups}$, $\mathit{Elements}$, $1$),
$h$ is equal to \False. By the way a counter object works, it follows that 
no expeditive mode is ever executed at the first level of the recursion.
Note that at this level, the roles of $D_1$ and $H_1$ are played by the one-dimensional arrays
$DChunks$ and $HChunks$, respectively. Moreover, $DGroups$ and $DElements$ play the role of $D_2$ and $D_3$,
respectively, and $HGroups$ plays the role of $H_2$. Each chunk is processed by recursively calling
\Traverse\ ({\em level-2 recursion}) on line~\ref{alg:bc:recur} (with $\mathit{rlevel}$ being equal to $2$). 
The goal of a level-2 invocation of \Traverse\
is to process an entire chunk by splitting it into groups and calling
\Traverse\ once more ({\em level-3 recursion}) to process the elements of each group 
(recursive call of line~\ref{alg:bc:recur} with $\mathit{rlevel}$ being equal to $3$). 
Note that in a level-2 invocation corresponding to some chunk $i$, 
\RawData\ is the two-dimensional array containing the elements of the groups of chunk $i$.
Moreover, the role of $D_1$ is now played by the one-dimensional array $DGroups[i]$, 
and the role of $D_2$ by the two-dimensional array $DElements[i]$,
whereas $D_3$ is no longer needed and is $\mathit{NULL}$.
The role of $H_1$ is now played by the one-dimensional array $HGroups[i]$. 
Helping (lines~\ref{alg:bc:scan:for}-\ref{alg:bc:help:c:true}) follows the general pattern
described in Algorithm~\ref{alg:refresh}. 

%%%%%%%%%%%%%%%%%%%%%%%%%%%%%%%%%%%%%% PSEYDOCODE%%%%%%%%%%%%%%%%%%%%%%%%%%%%%%%%%%%%%%%%%%%%%%%%%%%%%%%%%%%%%%%%%%%%%%


\begin{algorithm}[t]
    \footnotesize
    \vspace*{2mm}

    \begin{algorithmic}[1]
    
    \State \textbf{Shared variables:}
    \State Set $\mathit{RawData}[1..k][1..m][1..r]$, initially containing all data series \label{alg:bc:r}
    \State \textbf{Boolean} $\mathit{DChunks}[1..k]$, $\mathit{DGroups}[1..k][1..m]$, $\mathit{DElements}[1..k][1..m][1..r]$, initially all \textbf{False} \label{alg:bc:c}
    \State \textbf{Boolean} $\mathit{HChunks}[1..k]$, $\mathit{HGroups}[1..k][1..m]$, initially all \textbf{False} \label{alg:bc:h}
    \State CounterObject $\mathit{Chunks}$, $\mathit{Groups}[1..k]$, $\mathit{Elements}[1..k][1..m]$ \label{alg:bc:e}
    \State \textbf{int} $\mathit{Size}[1..3] = \{k,m,r\}$

    \vspace*{1mm}
    \State \textbf{Code for each thread:}

    \Procedure{Traverse}{Function *BufferCreation, DataSeries $\mathit{RawData}[]$, Boolean $\mathit{D_1}[]$, Boolean $\mathit{D_2}[]$, Boolean $\mathit{D_3}[]$, Boolean $\mathit{H_1}[]$, Boolean $\mathit{H_2}[]$, Boolean $h$, CounterObject $Cnt_1$, CounterObject $Cnt_2[]$, CounterObject $Cnt_3[]$, int $\mathit{rlevel}$}
        \State \textbf{int} $\mathit{i}$
        \While{\textbf{True}} \label{alg:bc:while:start}
            \State $\langle i, * \rangle \gets Cnt_1.\NextIndex(\&h)$
            \If{$i > \mathit{Size}[rlevel]$} \textbf{break} \EndIf
            \State Mark $\mathit{RawData[i]}$ as acquired
            \If{$rlevel < 3$}
                \State Traverse($\mathit{BufferCreation, RawData[i], D_2[i], D_3[i], D_3[i], H_2[i],}$
                \Statex \quad $\mathit{NULL, H_1[i], Cnt_2[i], Cnt_3[i], Cnt_3[i], rlevel+1}$) \label{alg:bc:recur}
            \Else
                \State $*$\Call{BufferCreation}{$RawData[i]$}
            \EndIf
            \State $\mathit{D_1[i]} \gets \textbf{True}$ \label{alg:bc:c:true}
        \EndWhile

        \ForAll{$j$ such that $\mathit{D_1[j]}$ is \textbf{False}} \label{alg:bc:scan:for}
            \State Backoff() \Comment{Avoid helping if possible} \label{alg:bc:help:backoff}
            \If{$\mathit{D_1[j]}$ is \textbf{False}} \label{alg:bc:help:if}
                \State $\mathit{H_1[j]} \gets \textbf{True}$ \label{alg:bc:h:true}
                \If{$rlevel < 3$}
                    \State Traverse($\mathit{BufferCreation, RawData[j], D_2[j], D_3[j], D_3[j], H_2[j],}$
                    \Statex \quad $\mathit{NULL, H_1[j], Cnt_2[j], Cnt_3[j], Cnt_3[j], rlevel+1}$) \label{alg:bc:help:process}
                \Else
                    \State $*$\Call{BufferCreation}{$RawData[j]$}
                \EndIf
                \State $\mathit{D_1[j]} \gets \textbf{True}$ \label{alg:bc:help:c:true}
            \EndIf
        \EndFor
    \EndProcedure

    \end{algorithmic}

    \caption{Pseudocode for \textsc{Traverse} in \textsc{FreSh}. Code for thread $t$.}
    \label{alg:bc}
\end{algorithm}

The backoff time in \Fresh\ depends on the average execution time required by each
thread to process a group. Each thread $\mathit{t}$ counts the average time 
$T_{avg}$ it has spent to process all the parts it acquired, and whenever
it encounters a group to help, it sets the backoff time to be proportional
to $T_{avg}$ and performs helping only after backoff, if it is still needed.
This allows an owner thread that has not crashed, to finish the processing of its last chunk,
while it still executes in expeditive mode, thus avoiding the cost of switching
to and executing a standard mode.
% 
\Fresh\ implements \TP\ using a set of $2^w$ summarization buffers
($w$ is the number of segments of an iSAX summary),one for each bit sequence of $w$ bits.
To decide to which summarization buffer to store a pair,
\Fresh\ (as other iSAX-based indexes) 
examines the bit sequence consisting of the first bit of each 
of the $w$ segments of the pair's iSAX summary, 
and places the pair into the corresponding summarization buffer. 
Each of the summarization buffers is split into $N$ parts, one for each of the $N$ threads in the system.
Each thread uses its own part in each buffer to store the elements it inserts.


\noindent
{\bf Tree Population Stage.}
Similarly to the buffers creation stage, in tree population, the worker threads
have to traverse and process the elements of \TP, i.e. all pairs added in the summarization buffers.
Processing is now achieved by calling the \TreePopulation\ function (Algorithm~\ref{alg:iSAXTraverse})
for each pair. \TreePopulation\ finds the right subtree of the index tree to place each pair, and then 
simply calls \PS.\Put\ to add the pair into \PS, the next traverse object in the dataflow pipeline. 
% Recall that \TP.\Put\ is implemented by calling \MBInsert\ (Algorithm~\ref{alg:mb}).
To implement \TP.\Traverse, we split the elements of \TP\ into $2^w$ workloads 
as the number of summarization buffers, and apply \Refresh.
%,
Each thread $\mathit{t}$ repeatedly acquires summarization buffers using \FAI,
and process them to produce the corresponding trees. 
% 
Each summarization buffer could be further split into chunks and groups, and \Refresh\ could be called recursively.
Pseudocode for \Traverse\ of \TP\ closely follows that for \BC.
\BC\ and \TP\ are lock-free implementations of a traverse object. 

\section{Prunning and Refinement}

In \Fresh, \PS\ is implemented as a forest of $2^w$ leaf-oriented trees,
one for each of the summarization buffers. 
The trees of the forest are the root subtrees of a standard iSAX-based tree.
\TreePopulation\ simply transfers the pairs from a summarization buffer
to the appropriate subtree of the index tree. 
To support the concurrent population of a subtree by multiple threads, 
\Fresh\ utilizes Algorithm~\ref{alg:tree} 
%presented in Section~\ref{sec:trees}. % (and discussed later). 
%Specifically, the \Put\ operation of \PS\ (see Algorithm~\ref{alg:ps}) is implemented
%by simply calling \TreeInsert\ (Algorithm~\ref{alg:tree}). Whenever this routine
%is called by a thread that process an originally assigned summarization buffer, 
%it should have its $\mathit{isHelper}$ argument equal to $\False$,
%whereas when it is called by a thread that helps, it should have this argument equal to $\True$.
%
To implement \PS.\Traverse, \Fresh\ (Algorithm ~\ref{alg:ps}) uses \Refresh\ to process 
the different subtrees of the index tree. Specifically, 
each thread $t$ access a counter to get assigned a subtree $T$ to process.
To process the nodes of $T$, \Refresh\ is applied recursively. % (line~\ref{alg:ps:help:c:rec}). 
%Specifically, the threads working on $T$ %(owner and helpers) 
%use a counter to get assigned nodes in $T$.
A thread $t$ that is assigned node $i$ of $T$,
first searches for the $i$-th node, according to inorder,
and then processes it by invoking the \Prunning\ function %(line~\ref{alg:ps:prunning}) 
of Algorithm~\ref{alg:iSAXTraverse}. 
%As soon as $t$ figures out that all nodes of $T$ have been assigned for processing, 
%it re-traverses the nodes of $T$,
%checking their done flags to figure out whether there are nodes 
%whose processing has not been completed yet, and helps
%whenever it is needed. Similarly, when $t$ figures out that all subtrees have been
%assigned for processing, it examines the done flags of the subtrees to figure out
%whether there are subtrees whose processing has not yet been completed, and it helps if needed. 
%
%Pseudocode for implementing \Traverse\ of \PS\ is provided in .
To find the $i$-th node of $T$ in an efficient way, 
for each node $nd$ of $T$, \Fresh\ maintains a counter
$\mathit{cnt_{nd}}$ that counts the number of nodes in the left subtree
of $\mathit{nd}$. \Fresh\ %(lines~\ref{alg:ps:help:c:treenode-start}-\ref{alg:ps:help:c:treenode-end})
uses these counters to find the $i$-th node of $T$ by simply traversing a path in $T$.
The total number of nodes in T is calcualted by simply traversing the righmost path of
$T$ and summing up the counters stored in the traversed nodes.
%Pseudocode for \FindNode\ and \TotalNodes\ is provided in Algorithm~\ref{alg:ps}.
%\here{It is unclear to which algorithm the line numbers above refer.}

%%%%%%%%%%%%%%%%%%%%%%%%%%%%%%%%%%%% Pseudocode for PRUNNING %%%%%%%%%%%%%%%%%%%%%%%%%%%%%%%%%%%%%%%

\begin{algorithm}[htbp]
    \footnotesize
    \vspace*{2mm}
    
    \begin{algorithmic}[1]
    
    \State \textbf{Shared variables:}
    \State TreeNode *$\mathit{IndexTree}[1..2^w]$ \label{alg:ps:r}
    \State \textbf{bool} $\mathit{DTree[1..2^w]}$, $\mathit{HTree[1..2^w]}$, initially all \textbf{false} \label{alg:ps:c}
    \State CounterObject $\mathit{TreeCnt[1..2^w]}$
    
    \vspace*{1mm}
    \Procedure{Traverse}{Function *\Prunning, TreeNode *$T$, CounterObject *$\mathit{Cnt}$, 
        int $x$, bool $h$, int $\mathit{rlevel}$}
        \State int $\mathit{i}$
    
        \While{\textbf{true}}
            \State $\mathit{\langle i, * \rangle = Cnt.\NextIndex(\&h)}$ \label{alg:ps:help:c:cnt}
            \If{$\mathit{i > x}$}
                \State \textbf{break}
            \EndIf
            \If{$\mathit{rlevel < 2}$}
                \State \textbf{mark} $\mathit{IndexTree[i]}$ as acquired
                \State $\mathit{totalNds} \gets$ \Call{TotalNodes}{$\mathit{IndexTree[i]}$}
                \State Traverse($\mathit{Prunning, IndexTree[i], TreeCnt[i], totalNds,}$
                \Statex \quad $\mathit{\textbf{false}, rlevel+1}$) \label{alg:ps:help:c:rec}                
                \State $\mathit{DTree[i] := \textbf{true}}$ \label{alg:ps:help:c:true}
            \Else
                \State $\mathit{nd} \gets$ \Call{FindNode}{$\mathit{T,i}$} \label{alg:ps:findnode}
                \State \textbf{mark} $\mathit{nd}$ as acquired
                \State \Call{\Prunning}{$\mathit{nd}$} \label{alg:ps:prunning}
                \State \textbf{mark} $\mathit{nd}$ as done
            \EndIf
        \EndWhile
    
        \ForAll{$j$ such that $\mathit{DTree[j]}$ is \textbf{false}} \label{alg:ps:scan:for}
            \State \Call{Backoff}{} \Comment{Avoid helping, if possible} \label{alg:ps:help:backoff}
            \If{$\mathit{DTree[j]} == \textbf{false}$} \label{alg:ps:help:if}
                \State $\mathit{HTree[j] := \textbf{true}}$ \label{alg:ps:h:true}
                \State \Call{HelpTree}{\Prunning, $\mathit{IndexTree[j]}$}
            \EndIf
            \State $\mathit{DTree[j] := \textbf{true}}$ \label{alg:ps:help:c:true:helping}
        \EndFor
    \EndProcedure
    
    \vspace*{1mm}
    \Procedure{FindNode}{TreeNode *$T$, int $i$} \Comment{Returns TreeNode*} \label{alg:ps:help:c:treenode-start}
        \State TreeNode *$\mathit{p} \gets T$
        \State int $\mathit{nds} \gets 0$
    
        \While{$\mathit{p \neq NULL\ \And\ nds \neq i}$}
            \If{$\mathit{nds + p \rightarrow cnt + 1 < i}$}
                \State $\mathit{nds} \gets \mathit{nds + p \rightarrow cnt + 1}$
                \State $\mathit{p} \gets p \rightarrow rc$
            \Else
                \State $\mathit{p} \gets p \rightarrow lc$
            \EndIf
        \EndWhile
        \State \Return $\mathit{p}$ \label{alg:ps:help:c:treenode-end}
    \EndProcedure
    
    \vspace*{1mm}
    \Procedure{HelpTree}{Function *$f$, TreeNode *$T$}
        \If{$T == NULL$}
            \State \Return
        \EndIf
        \State \Call{HelpTree}{$f$, $T \rightarrow lc$}
        \If{$*T$ is unprocessed}
            \State $*f(*T)$
        \EndIf
        \State \Call{HelpTree}{$f$, $T \rightarrow rc$}
    \EndProcedure
    
    \end{algorithmic}
    
    \caption{Pseudocode for \Put\ and \Traverse\ of \PS\ in \Fresh. Code for thread $t \in \{ 1, \ldots, N-1\}$.}
    \label{alg:ps}
    \end{algorithm}


    \subsection{Insert in Leaf-Oriented Tree}
    \label{sec:leaf-oriented}

    Each node of the tree stores a key and the pointers
    to its left and right children.  (refer to Algorithm~\ref{alg:tree}).
    A leaf node stores additionally an array $D$, where the leaf's data are stored. 
    We assume that each data item is a pair containing a key and the associated information.
    A node may have its own key. For instance, in iSAX-based indexes, this key is the node's
    iSAX summary which summarizes all data series stored in it.
    The proposed implementation allows multiple insert operations to concurrently update array
    $D$ of a leaf. This results in enhanced parallelism and performance. 
    To achieve this, each leaf $\ell$ contains a counter, called $\mathit{Elements}$.
    % 
    Each thread $\mathit{t}$ that tries to insert data in $\ell$, uses
    $\mathit{Elements}$ to acquire a position $\mathit{pos}$ in the array $D$ of $\ell$.
    If $D$ is not full, (line~\ref{alg:tree:pos-in-D}) (i.e. $\mathit{pos} < M$), 
    $\mathit{t}$ stores the new element in $D[\mathit{pos}]$ (line~\ref{alg:tree:store-in-D}).
    Otherwise, in case the array is full, i.e. $\mathit{pos \geq M}$, then 
    $\mathit{t}$ attempts to split the leaf (line~\ref{alg:tree:pos-not-in-D}). 
    % 
    During spliting, $D$ may contain empty positions, since some
    threads may have acquired positions in $D$ but have not yet stored their
    elements there (line~\ref{alg:tree:store-in-D}).
    To avoid situations of missing elements, each leaf contains an $\mathit{Announce}$ array 
    with one position for each thread. A thread announces its operation
    in $\mathit{Announce}$ before it attempts to acquire a position in $D$.
    During spliting, a thread distributes to the new leaves it creates not only the
    elements found in $D$ but also those in $\mathit{Announce}$.

    %%%%%%%%%%%%%%%%%%%%%%%%%%%%%%Tree Insert Code%%%%%%%%%%%%%%%%%%%%%%%%%%%%%%%
    \begin{algorithm}[t]
        \footnotesize
        \caption{Type Definitions for the Lock-Free Tree}
        \label{alg:tree-types}
        \begin{algorithmic}[1]
            \Statex \textbf{Type Definitions:}
            \State \textbf{Type} Node:
            \State \quad int $\mathit{key}$
            \State \quad $\{Node,Leaf\}$ *$\mathit{left}$
            \State \quad $\{Node,Leaf\}$ *$\mathit{right}$
            \State \quad InsertRec $\mathit{Announce}[0..n-1]$
            \State \quad \textbf{Boolean} $\mathit{helpersExist}$
    
            \Statex
            \State \textbf{Type} InsertRec:
            \State \quad Data $\mathit{data}$
            \State \quad int $\mathit{position}$
    
            \Statex
            \State \textbf{Type} Leaf \textbf{extends} Node:
            \State \quad Data $D[0..m-1]$
            \State \quad CounterObject Elements
        \end{algorithmic}
    \end{algorithm}
    


    \begin{algorithm}[t]
        \footnotesize
        \caption{TraverseTree: a lock-free leaf-oriented tree with fat leaves, implementing a traverse object.
         Code for thread $t \in \{ 1, \ldots, N-1\}$.}
        \label{alg:tree}
        \begin{algorithmic}[1]
    
            \Statex \textbf{Shared Variables:}
            \State $\mathit{Tree} \gets \NULL$ \Comment{Initially points to a Leaf with initialized values}
    
            \Procedure{TreeInsert}{$\mathit{data}, \mathit{isHelper}$}
                \State $\mathit{leaf} \gets$ \NULL
                \State $\mathit{parent} \gets \mathit{Tree}$
                \State $\mathit{ptr} \gets$ \NULL
                \State $\mathit{pos}, \mathit{val} \gets 0$
                \State $\mathit{expeditive} \gets$ \False
    
                \While{\True} \label{alg:tree:while}
                    \State $\langle \mathit{leaf, parent} \rangle \gets$ \Call{Search}{$\mathit{data, parent}$} \label{alg:tree:search}
                    \If{$\mathit{parent} = \NULL$} \label{alg:tree:ptr-parent:null}
                        \State $\mathit{ptr} \gets \&\mathit{Tree}$
                    \ElsIf{$\mathit{parent} \rightarrow \mathit{left} = \mathit{leaf}$}
                        \State $\mathit{ptr} \gets \&\mathit{parent} \rightarrow \mathit{left}$ \label{alg:tree:ptr-left}
                    \Else
                        \State $\mathit{ptr} \gets \&\mathit{parent} \rightarrow \mathit{right}$ \label{alg:tree:ptr-right}
                    \EndIf
    
                    \If{$\mathit{isHelper} = \True$ \textbf{and} $\mathit{leaf} \rightarrow \mathit{helpersExist} = \False$}
                        \State $\mathit{leaf} \rightarrow \mathit{helpersExist} \gets \True$ \label{alg:tree:switch-mode:}
                    \EndIf
    
                    \State $\langle \mathit{pos, expeditive} \rangle \gets$ \Call{NextIndex}{$\& \mathit{leaf} \rightarrow \mathit{helpersExist}$} \label{alg:tree:get-pos}
    
                    \If{$\mathit{expeditive} = \False$}
                        \State $\mathit{leaf} \rightarrow \mathit{Announce}[t] \gets \langle \mathit{data, \bot} \rangle$ \label{alg:tree:announce:op}
                    \EndIf
    
                    \If{$\mathit{pos} < M$} \label{alg:tree:pos-in-D}
                        \If{$\mathit{expeditive} = \False$}
                            \State $\mathit{leaf} \rightarrow \mathit{Announce}[t].\mathit{position} \gets \mathit{pos}$ \label{alg:tree:announce:pos-in-D}
                        \EndIf
                        \State $\mathit{leaf} \rightarrow \mathit{D}[\mathit{pos}] \gets \mathit{data}$ \label{alg:tree:store-in-D}
                    \Else \label{alg:tree:pos-not-in-D}
                        \State \Call{SplitLeaf}{$\mathit{leaf}, \mathit{ptr}, \mathit{expeditive}$}
                    \EndIf
    
                    \If{$(*\mathit{ptr}) \rightarrow \mathit{helpersExist} = \True$} \label{alg:tree:st:finish}
                        \If{$\mathit{expeditive} = \False$ \textbf{and} $(*\mathit{ptr}) \rightarrow \mathit{Announce}[t].\mathit{position} \neq \bot$} \label{alg:tree:op-is-applied}
                            \State $(*\mathit{ptr}) \rightarrow \mathit{Announce}[t] \gets \langle \bot, \bot \rangle$ \label{alg:tree:clean}
                        \Else
                            \State \textbf{continue} \label{alg:tree:re-attempt}
                        \EndIf
                    \EndIf
                    \State \Return
                \EndWhile
            \EndProcedure
    
            \Procedure{SplitLeaf}{$\mathit{leaf}, \mathit{prt}, \mathit{expeditive}$}
                \State $\mathit{newNode} \gets$ \Call{NewNode}{} \label{alg:tree:split:new-internal-node}
                \State $\mathit{newNode} \rightarrow \mathit{left} \gets$ \Call{NewLeaf}{}
                \State $\mathit{newNode} \rightarrow \mathit{right} \gets$ \Call{NewLeaf}{}
                \State $\mathit{splitBuffer} \gets \emptyset$
    
                \If{$\mathit{expeditive} = \False$} \label{alg:tree:em:Announce-scan}
                    \For{$i \in \{0, \dots, n-1\}$ \textbf{where} $\mathit{leaf} \rightarrow \mathit{Announce}[i].\mathit{data} \neq \bot$} \label{alg:tree:Announce-scan}
                        \State $\mathit{ldata} \gets \mathit{leaf} \rightarrow \mathit{Announce}[i].\mathit{data}$
                        \If{$\mathit{leaf} \rightarrow \mathit{Announce}[i].\mathit{position} \neq \bot$} \label{alg:tree:announce-scan:pos-not-bot}
                            \State $\mathit{leaf} \rightarrow \mathit{D}[\mathit{leaf} \rightarrow \mathit{Announce}[i].\mathit{position}] \gets \mathit{ldata}$ \label{alg:tree:announce-scan:store-in-D}
                        \Else
                            \State \Call{AddToBuffer}{$\mathit{ldata}, \mathit{splitBuffer}$} \label{alg:tree:announce-scan:data-copy}
                        \EndIf
                        \State $\mathit{newNode} \rightarrow \mathit{Announce}[i] \gets \langle \mathit{ldata}, -1 \rangle$  \label{alg:tree:announce-scan:mark-op-applied}
                    \EndFor
                \EndIf
    
                \State \Call{Distribute}{$\mathit{leaf} \rightarrow \mathit{D}, \mathit{splitBuffer}, \mathit{newNode} \rightarrow \mathit{left}, \mathit{newNode} \rightarrow \mathit{right}$} \label{alg:tree:split}
                \State \Call{CAS}{$*\mathit{prt}, \mathit{leaf}, \mathit{newNode}$} \label{alg:tree:CAS}
            \EndProcedure
    
        \end{algorithmic}
    \end{algorithm}
    
    More specifically, a thread $t$ executing \TreeInsert\ repeatedly executes the following actions. 
    It first calls a standard \textit{Search} routine to traverse a path of the tree and find an appropriate
    $\mathit{leaf}$ and its $\mathit{parent}$ (line~\ref{alg:tree:search}).
    Pointer $\mathit{ptr}$ is a reference to the appropriate child field of $\mathit{parent}$,
    which needs to be changed to perform \TreeInsert.  (lines~\ref{alg:tree:ptr-parent:null}-\ref{alg:tree:ptr-right}).
    Next, $t$ accesses the counter object to acquire a position in $D$ (line~\ref{alg:tree:get-pos})
    and proceeds to announce the data that it wants to insert in the tree (line~\ref{alg:tree:announce:op}).
    Afterwards, it announces this position in $\mathit{Announce}$ and stores the data in $D[\mathit{pos}]$
    (line~\ref{alg:tree:store-in-D}), if $D$ is not full (line~\ref{alg:tree:pos-in-D}). 
    If $D$ is full, it calls \SplitLeaf\ to split $\mathit{leaf}$.
    If this \CAS\ succeeds, then the data have been added and \TreeInsert\ completes. 
    Otherwise, some other thread has successfully split the node.
    
    We finally discuss the following subtle scenario. Assume that the owner thread $t$ calls \TreeInsert,
    reaches a leaf $\mathit{l}$, and acquires the last valid position in array $D$ of $\mathit{l}$. 
    Thread $t$ executes in expeditive mode (so it does not announce its data),
    and before it records its data in $D$, it becomes slow. Next, a helper thread $t'$ reaches $\mathit{l}$, switches 
    $\mathit{l}$'s execution mode to standard, and splits $\ell$ (executing on standard mode). 
    Unfortunately, during this split, $t'$ will not take into 
    consideration the data of $t$, since $t$ neither has announced its operation
    (since $t$ was executing in expeditive mode), nor has yet written its data into $D$. 
    To disallow thread $t$ from finishing its operation without inserting its data, \TreeInsert\ provides the 
    following mechanism (lines~\ref{alg:tree:st:finish}-\ref{alg:tree:re-attempt}). Before it terminates, 
    thread $t$ re-reads the appropriate child field of the parent of $\ell$ (through $\mathit{ptr}$) 
    and checks the $\mathit{helpersExist}$ flag of the node $nd$ that $\mathit{ptr}$ points to,
    to figure out whether it can still operate on expeditive mode. In the scenario above, 
    $nd$ will be the node that $t'$ has allocated to replace $\ell$, and thus it has its 
    $\mathit{helpersExist}$ flag equal to \True\ (line~\ref{alg:tree:split:new-internal-node}). 
    This way, $t$ discovers that the execution mode for $\mathit{l}$ has changed
    (line~\ref{alg:tree:st:finish} and first condition of line~\ref{alg:tree:op-is-applied}),
    and re-attempts its \Insert\ (line~\ref{alg:tree:re-attempt}).
        
    \begin{lemma}
    \label{lem:tree}
    Algor.~\ref{alg:tree} is a {\em linearizable, lock-free} implementation of a leaf-oriented
    tree with fat leaves, supporting only insert operations.
    \end{lemma}

    %%%%%%%%%%%%%%%%%%%%%%%%%%%%% Refinement %%%%%%%%%%%%%%%%%%%%%%%%%%%%%%%%%%%
    \subsection{Refinement}

    To implement \RS, \Fresh\ uses a set of priorities queues
    each implemented using an array ( Algorithm~\ref{alg:pq}). 
    A thread inserts elements in all arrays in a round-robin fashion. 
    This technique results in almost equally-sized arrays, which is crucial
    for achieving load-balancing. 

    %%%%%%%%%%%%%%%%%%%%%%%%%%% Sorted Arrays Code %%%%%%%%%%%%%%%%%%%%%%%%%%%%%%%%%%%%%%%%%%%%%%%%%%%%%
    
    \begin{algorithm}[t]
        \footnotesize
        \caption{Priority Queue of \Fresh. Code for thread $t$.}
        \label{alg:pq}
        \begin{algorithmic}[1]
    
            \State \textbf{Shared variables:}
            \State $\mathit{\langle int, Data \rangle}$ $A[0..k-1]$, initially all $\langle \bot, \bot \rangle$
            \State CounterObject $\mathit{Cnt}$, initially $0$
            \State Boolean $\mathit{helpersExist}$, initially $\False$
            \State int $\mathit{insPos}$, initially $0$
            \State $\mathit{\langle int, Data \rangle}$ *$\mathit{SA}$, initially $\NULL$
    
            \vspace{2mm}
            
            \Procedure{Insert}{int $\mathit{priority},Data \mathit{values}$}
                \State int $\mathit{pos} \gets \FAI(\mathit{insPos})$ \label{alg:pq:ins:FAI}
                \State $\mathit{A[pos]} \gets \langle \mathit{priority, value} \rangle$ \label{alg:pq:ins:store}
            \EndProcedure
    
            \vspace{2mm}
    
            \Procedure{InitDeletePhase}{}
                \State $\mathit{\langle int, Data \rangle}$ *$\mathit{sa}$
                \State $\mathit{sa} \gets$ allocate local array of $insPos$ elements
                \State Copy non-$\bot$ elements of $A$ into $\mathit{sa}$ and sort them \label{alg:sa:local-copy}
                \State $\mathit{\CAS(\&SA, \NULL, sa)}$ \label{alg:sa:CAS}
            \EndProcedure
    
            \vspace{2mm}
    
            \Function{DeleteMin}{boolean $ \mathit{isHelper}$} \textbf{returns} $\mathit{Data}$
                \If{$\mathit{isHelper = \True}$ \textbf{and} $\mathit{helpersExist = \False}$}
                    \State $\mathit{helpersExist} \gets \True$
                \EndIf
                \State int $\mathit{pos} \gets \mathit{Cnt.\NextIndex(\&helpersExist})$ \label{alg:pq:deleteMin:next}
                \If{$\mathit{pos} \geq insPos$}
                    \State \Return $\bot$
                \EndIf
                \State \Return $\mathit{SA[pos].data}$
            \EndFunction
    
        \end{algorithmic}
    \end{algorithm}
    
    
    
    During query answering, an application may use the same index tree
    to answer more than one query. In that case, the done flags of the nodes and other
    variables need to be reset each time a new query starts. This may require syncrhonization. 
    To avoid this case, \Fresh\ implements the {\em done} flag as a counter (rather than as a boolean). 
    This counter describes the number of queries for which the node has been processed.
    
    To implement \RS.\Traverse, \Fresh\ first comes up with sorted versions of the arrays,
    shared to all threads. Then, it uses \Refresh\ to assign sorted arrays to threads 
    for processing. To process the elements of a sorted array $SA$, \Refresh\ is
    applied recursively. 
    %Specifically, the threads working on $SA$ 
    %use a counter object to get assigned elements of $SA$.
    Processing of an array element is performed by invoking the \Refinement\ function
    (Algorithm~\ref{alg:iSAXTraverse}). Helping is done at the level of 
    1. each individual priority queue and 
    2. the set of priority queues, in a way similar to that in \PS.
    \RS\ is a linearizable lock-free implementation of a traverse object. 
    %As soon as $t$ discovers that all elements of $PQ$ have been assigned for processing, 
    %it re-traverses the nodes of $PQ$,
    %checking their done flags to figure out whether there are elements
    %whose processing has not been completed yet, and helps
    %whenever it is needed. Similarly, when $t$ figures out that all subtrees have been
    %assigned for processing, it examines the done flags of the subtrees to figure out
    %whether there are subtrees whose processing has not yet been completed, and it helps if needed. 
    %Pseudocode for implementing \Put\ and \Traverse\ of \RS\ resemles that of \PS\ and 
    %is omitted.
    
    To update BSF, \Fresh\ repeatedly reads the current value $y$ of BSF, and attempts to
    atomically change it from $v$ to the new value $v'$, using \CAS, until it either succeeds or some value 
    smaller than or equal to $v'$ is written in BSF.
    
    %When an invocation of \Traverse(\&$\mathit{\Prunning, *, \True}$)\ by a thread $t$ on \RS\ completes, 
    %for every leaf $\ell$ in $\RS$, $\ell$ either has been processed or it has been pruned.
    %By all the claims we present above:
    
    \begin{lemma}
    \label{lem:rs}
    {\bf (1)} \RS\ is a linearizable lock-free implementation of a traverse object that supports \Put\ and
    \Traverse($\mathit{*, *, 0}$). 
    {\bf (2)} For every thread $t$, when an invocation of \Traverse(\&$\mathit{\Prunning, *, \True}$)\ by $t$ on \RS\ completes, 
    for every leaf $\ell$ in $\RS$, $\ell$ either has been processed or it has been pruned.
    \end{lemma}
    
    \begin{theorem}
    \label{thm:qa}
    \Fresh\ solves the 1-NN problem and provides a lock-free implementation of \QueryAnswering\
    (Alg.~\ref{alg:iSAXTraverse}). 
    \end{theorem}