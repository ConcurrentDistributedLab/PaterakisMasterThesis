\chapter{Traverse Object}
\label{chapter:traverse-object}

\section{Traverse Object}

In this section, we introduce the \textbf{traverse object}, an abstract data type
that enables a modular design for iSAX-based indexes. The processing pipeline of
an iSAX-based index consists of multiple stages, where each stage processes data
produced by the previous one. The first stage handles the original dataset, while
subsequent stages refine and structure the data progressively. This structured
processing inspired the definition of the \textbf{traverse object}, whose
sequential specification is given below.

\subsection{Definition of the Traverse Object}

\begin{definition}
\label{def:traverse}
Let $U$ be a universe of elements. A \textbf{traverse object} $S$ stores elements
of $U$ (which may not be distinct) and supports the following operations:

\begin{itemize}
    \item \textit{Put}($S,e,\mathit{param}$): Adds an element $e \in U$ to $S$.
    The optional parameter $\mathit{param}$ allows implementations to pass
    additional arguments.
    
    \item \textit{Traverse}($S,f,\mathit{fparam},del$): Traverses $S$, applying
    the function $f$ with parameters $\mathit{fparam}$ to each element. If the
    $del$ flag is set, the elements are deleted from $S$ after being processed.
    The parameter $\mathit{fparam}$ serves the same purpose as in \textit{Put}.
\end{itemize}

A traverse object $S$ satisfies the \textbf{traversing property}:  
Each instance of \textit{Traverse} in a sequential execution applies $f$ at
least once to all distinct elements added to $S$ that have not been deleted
in a previous invocation of \textit{Traverse}.
\end{definition}

\subsection{Traverse Objects in iSAX-Based Indexing}

\begin{algorithm}[htbp]
    \footnotesize
    \vspace*{2mm}
    
    \begin{algorithmic}[1]
    
    \Procedure{InitializeSharedObjects}{}
        \State \BC $\gets$ initially contains all raw data series
        \State \TP, \PS, \RS $\gets$ initially empty
        \State $\mathit{BSF} \gets$ some initial value
    \EndProcedure
    
    \vspace*{1mm}
    \State{Code for thread $t_i$, $i \in \{0, \ldots, n-1\}$:}		
    \vspace*{1mm}

    \Procedure{\QueryAnswering(QuerySeriesSet $\mathit{SQ}$) \textbf{returns} int}{}
        \State \BC.\Traverse(\&BufferCreation(), $\mathit{BCParam}$, \False)
        \State \TP.\Traverse(\&TreePopulation(), $\mathit{TPParam}$, \False)
        \State \PS.\Traverse(\&Prunning(), $\mathit{PSParam}$, \False)
        \State \RS.\Traverse(\&Refinement(), $\mathit{RSParam}$, \True)
        \State \Return $\mathit{BSF}$
    \EndProcedure
    
    \vspace*{1mm}
    \Procedure{\BufferCreation(DataSeries $\mathit{ds}$)}{}
        \State iSAXSummary $\mathit{iSAX} \gets$ Calculate the iSAX summary for $\mathit{ds}$
        \State Index $\mathit{bind} \gets$ index to appropriate buffer based on $\mathit{iSAX}$
        \State \TP.\Put($\langle \mathit{iSAX}$, index of $\mathit{ds}$ $\rangle$, $\mathit{bind}$)
    \EndProcedure
    
    \vspace*{1mm}
    \Procedure{\TreePopulation(Summary $\mathit{iSAX}$, Index $\mathit{ind}$,
         Index $\mathit{bind}$, Boolean $\mathit{flag}$)}{}
        \State \PS.\Put($\langle \mathit{iSAX}$, $\mathit{ind} \rangle$, $\mathit{bind}$, $\mathit{flag}$)
    \EndProcedure
    
    \vspace*{1mm}
    \Procedure{\Prunning(DataSeries $Q$, DataSeriesSet $\mathit{E}$, Boolean
        $\mathit{flag}$) \textbf{returns} boolean}{}
            \State iSAXSummary $\mathit{iSAX} \gets$ Calculate the iSAX summary for $\mathit{E}$
            \State int $\mathit{lbDist} \gets$ lower bound distance between $\mathit{iSAX}$ and $Q$
            \If{$\mathit{lbDist} < \mathit{BSF}$}
                \State \RS.\Put($\langle \mathit{E, iSAX} \rangle$, $\mathit{flag}$)
                \Return \True
            \EndIf
            \Return \False
    \EndProcedure
    
    \vspace*{1mm}
    \Procedure{\Refinement(DataSeries $Q$, DataSeriesSet $E$, Summary $\mathit{iSAX}$,
     Function *\UpdateBSF) \textbf{returns} Boolean}{}
        \State int $\mathit{lbDist}$, $\mathit{rDist}$
        \State $\mathit{lbDist} \gets$ lower bound distance between $\mathit{iSAX}$ and $Q$
        \If{$\mathit{lbDist} < \mathit{BSF}$}
            \ForAll{pair $\langle \mathit{iSAX_{ds}, ind_{ds}} \rangle$ in $E$}
                \State $\mathit{lbDist} \gets$ lower bound distance between $\mathit{iSAX_{ds}}$ and $Q$
                \If{$\mathit{lbDist} < \mathit{BSF}$}
                    \State $\mathit{rDist} \gets$ real distance between $\mathit{ds}$ and $Q$
                    \If{$\mathit{rDist} < \mathit{BSF}$}
                        \State *\UpdateBSF($\mathit{BSF},\mathit{rDist}$) \Comment{user-provided routine}
                    \EndIf
                \EndIf
            \EndFor
            \Return \True
        \Else
            \Return \False
        \EndIf
    \EndProcedure
    
    \end{algorithmic}
    
    \caption{Implementation of an iSAX-based index using the traverse objects \BC, \TP, \PS, \RS.}
    \label{alg:iSAXTraverse}
    \end{algorithm}
    
    


To implement the different stages of an iSAX-based index, we use four instances of a traverse object 
to store and manipulate different types of data:

\begin{itemize}
    \item \textbf{Buffer Creation ($BC$)}: Stores the raw data series.
    \item \textbf{Tree Population ($TP$)}: Stores pairs of iSAX summaries and
     pointers to data series.
    \item \textbf{Pruning ($PS$)}: Organizes these pairs into sets corresponding
     to the leaf nodes of the tree.
    \item \textbf{Refinement ($RS$)}: Maintains priority queues of candidate
     series for query answering.
\end{itemize}

Each of these objects plays a crucial role in the indexing and query answering
process. The \textbf{buffer creation} phase utilizes an array, \RawData, to 
store the raw data series. Thus, elements of $BC$ are stored in \RawData. 
The \textbf{tree population} phase employs a set of \textbf{summarization buffers},
where iSAX summaries and pointers are initially stored. The traverse object $TP$
holds these pairs. The \textbf{pruning} stage organizes these pairs within a
leaf-oriented tree, where $PS$ manages sets of pairs corresponding to each tree leaf. 
Finally, the \textbf{refinement} phase employs priority queues to manage tree
leaves containing candidate series for further evaluation.

\subsection{Query Answering as a Sequence of Traversals}

Query answering in an iSAX-based index follows a structured sequence of four
\textit{Traverse} operations, applied in order to the different traverse objects.  
Algorithm~\ref{alg:iSAXTraverse} presents the pseudocode for implementing an iSAX-based index
using traverse objects.

Since the four stages do not overlap, their execution is typically synchronized
using barriers. In Algorithm~\ref{alg:iSAXTraverse}, these barriers (as well as any 
multithreading aspects) are embedded within the implementations of \textit{Put}
and \textit{Traverse}. This ensures that an iSAX-based index satisfies the
following key property:

\begin{definition}[\textbf{Non-Overlapping Property}]
\label{def:non-overlapping}
In every concurrent execution of the index, for every traverse object $S$, an
instance of \textit{Traverse} on $S$ is performed only after all \textit{Put}
operations that add distinct elements to $S$ have completed.
\end{definition}

\subsection{Ensuring Correctness Through Traversing and Non-Overlapping Properties}

Assuming that the \textbf{non-overlapping property} holds for $BC$, $TP$, $PS$,
and $RS$, and that \RawData initially contains all data series in the collection,
the \textbf{traversing property} guarantees that:

\begin{enumerate}
    \item The \BufferCreation function is invoked at least once for each data
     series in \RawData.
    \item The corresponding iSAX summary is computed, ensuring that at least one
     appropriate pair is inserted into $TP$.
    \item By the \textbf{non-overlapping property}, the \TreePopulation function
     processes all these pairs, inserting them into $PS$ (the index tree).
    \item The \Prunning function is applied to all elements in $PS$, and leaves
     that cannot be pruned are inserted into $RS$.
    \item The \Refinement function is applied to every traversed element of $RS$.
     Since \textit{Traverse} on $RS$ is invoked with $del = \mathit{True}$,
    priority queues can be used to implement $RS$, and \DeleteMin can be employed
    to remove elements as they are processed.
\end{enumerate}

Thus, the system ensures that only relevant candidates are further processed,
refining the query results by computing exact distances and updating the
\textbf{best-so-far} ($BSF$) value when necessary.

Implementations of \textit{Put} and \textit{Traverse} for $BC$, $TP$, $PS$, and
$RS$ in FreSh are provided in chapter~\ref{chapter:FreSh}.

\bl{
\section{Adapting the Traverse Object for Dynamic Data Processing}  

To support dynamic data processing in an iSAX-based index, the
\textit{traverse object} requires adaptation. In a dynamic setting,
threads execute concurrently and independently across the two distinct phases
of the index: \textit{index creation} and \textit{query answering}.
Consequently, the \BC\ and \TP\ traverse objects are assigned to the
\textit{index workers} (threads responsible for constructing the index), while
the \PS\ and \RS\ traverse objects are assigned to the \textit{query workers}
(threads responsible for answering queries).  

\subsubsection{Impact of Concurrency on Traverse Objects}  

The concurrent execution of these phases alters the procedure for answering
queries using traverse objects (see Algorithm~\ref{alg:DynamiciTraverseObject}).
Unlike the static setting, where the procedure involves four sequential
invocations of \Traverse\ (one for each traverse object), in the dynamic
setting, the sequence is reduced to only the last two traverse objects,
\PS\ and \RS.  

This change directly affects the \textit{non-overlapping property}
(Definition~\ref{def:non-overlapping}), which now applies only to traverse
objects within the same phase:  
\begin{itemize}  
    \item \textbf{Index workers:} \BC\ and \TP  
    \item \textbf{Query workers:} \PS\ and \RS  
\end{itemize}  

Since \Traverse\ and \Put\ operations can occur concurrently across phases,
situations arise where an index worker inserts new elements into the iSAX tree
while query workers are simultaneously pruning or refining their results.
This requires additional mechanisms to maintain correctness.  

\subsubsection{Handling Multiple Batches}  

In a dynamic environment where data arrives in batches, certain data structures
implementing traverse objects—specifically, \BC\ and \PS—must be
\textit{re-initialized} before processing a new batch. This re-initialization
is necessary because:  
\begin{itemize}  
    \item Each batch reuses the same buffers, meaning helpers must distinguish
    between \textbf{current} and \textbf{obsolete} data.  
    \item Helpers need to \textbf{skip outdated data} that no longer belongs to
    the active batch.  
    \item This differentiation is particularly important during the traversal of
    \TP\ and \PS\ objects.  
\end{itemize}  

\subsubsection{Ensuring Correctness with Sequence Numbers}  

To address these challenges, our approach introduces \textit{sequence numbers}
as a mechanism to track the state of the algorithm.  

\begin{definition}  
A \textbf{sequence number} is a monotonically increasing integer used to track  
the order of operations and ensure consistency in multi-threaded or
distributed environments.  
\end{definition}  

Each \textit{summarization buffer} is associated with a sequence number.
During \textbf{re-initialization}, an index worker performs the following steps:  
\begin{enumerate}  
    \item Resets the buffer size to zero.  
    \item Increments the buffer's sequence number by one.  
\end{enumerate}  

This mechanism prevents a thread from processing outdated data. If a worker reads
a nonzero buffer size, it indicates the presence of data. However, before proceeding,
the worker also checks the \textbf{sequence number}:  
\begin{itemize}  
    \item If the sequence number \textbf{matches} the current batch ID,
     the data is valid and can be processed.  
    \item If the sequence number is \textbf{smaller} than the current batch ID,
     the data is obsolete and must be ignored.  
\end{itemize}  

This ensures that no worker mistakenly processes data from a previous batch
(see Algorithm~\ref{alg:BufferReuse}).  


\begin{algorithm}[htbp]
    \footnotesize
    \vspace*{2mm}
    
    \begin{algorithmic}[1]
    
    \Procedure{InitializeSharedObjects}{}
        \State \BC $\gets$ initially contains all raw data series
        \State \TP, \PS, \RS $\gets$ initially empty
        \State $\mathit{BSF} \gets$ some initial value
        \State $\mathit{NextUpdateAvailable} \gets 0$
        \State $\mathit{GlobalSeq} \gets 0$
    \EndProcedure
    
    \vspace*{1mm}
    \State{Code for thread $t_i$, $i \in \{0, \ldots, k-1\}$:}    
    \vspace*{1mm}
    
    \Procedure{IndexCreation(int $\mathit{TotalUpdates}$) \textbf{returns} int}{}
        \State int $\mathit{currentBatch} \gets 1$
        \While{$\mathit{currentBatch} < \mathit{TotalBatches}$}
            \If{$\mathit{model} == \mathit{SystemTS}$ }
                \State batchTS $\gets$ getTS()
            \EndIf
            \State \BC.\Traverse(\&BufferCreation(), $\mathit{BCParam}$, \False)
            \State \TP.\Traverse(\&TreePopulation(), $\mathit{TPParam}$, \False)
            \ForAll{summarization buffers that have data}
                \State ReuseBuffers($\mathit{summBuffer}, \mathit{currentBatch}$)
            \EndFor
            \If{$\mathit{DELAY} >= \mathit{BatchProcessingTime}$}
                \State sleep($\mathit{DELAY}-\mathit{BatchProcessingTime}$)
            \EndIf
            \State $\mathit{currentBatch} \gets \mathit{currentBatch} + 1$
        \EndWhile
    \EndProcedure
    
    \vspace*{1mm}
    \State{Code for thread $t_i$, $i \in \{k, \ldots, n-1\}$:}    
    \vspace*{1mm}
    
    \Procedure{QueryAnswering(QuerySeriesSet $\mathit{SQ}$) \textbf{returns} int}{}
        \State int $\mathit{localTS} \gets 0$
        \While{$\mathit{!SQ.Empty}$}
            \If{$\mathit{model} == \mathit{LogicalTS}$}
                \If{$\mathit{localSeq} == \mathit{GlobalSeq}$}
                    \State \CAS(\&GlobalSeq, localSeq, localSeq + 1)
                \EndIf
            \ElsIf{$\mathit{model} == \mathit{SystemTS}$}
                \State queryTS $\gets$ getTS()
                \While {$\mathit{queryTS} >= \mathit{batchTS}$}
                    \State backoff()
                \EndWhile
            \EndIf
            \State \PS.\Traverse(\&Pruning(), $\mathit{PSParam}$, \False)
            \State \TP.\Traverse(\&Refinement(), $\mathit{RSParam}$, \False)
            \ForAll{summarization buffers that have data}
                \State ReuseBuffers($\mathit{summBuffer}, \mathit{currentBatch}$)
            \EndFor
            
            \If{$\mathit{model} == \mathit{LogicalTS}$}
                \State $\mathit{localSeq} = \mathit{localSeq} + 1$            
            \EndIf
        \EndWhile
        \State \Return $\mathit{BSF}$
    \EndProcedure
    
    \vspace*{1mm}
    \Procedure{ReuseSumBuffers(SumBuff * $\mathit{currBuff}$, int $\mathit{currentBatch}$)}{} \label{alg:BufferReuse}

        \State $\mathit{currBuff->size[workerID]} \gets 0$
        \If{$\mathit{currBuff->seq[workerID]} < \mathit{currentBatch}$}
            \State $\mathit{currBuff->seq[workerID]} \gets \mathit{currentBatch}$
        \EndIf
        \State $\mathit{currBuff->seq[workerID]++}$
    \EndProcedure
    
    \end{algorithmic}
    
    \caption{Implementation of a dynamic iSAX-based index using the traverse objects \BC, \TP.}
    \label{alg:DynamiciTraverseObject}
    \end{algorithm}
}