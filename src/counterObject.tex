\section{Counter Object}
\label{sec:counter-object}

Let \NextIndex\ be invoked $K$ times.
If crashes do not occur, every index in the range $R = 
\{0, \ldots, K-1\}$ should be returned exactly once 
be each of the $K$ invocations of \NextIndex\.
Otherwise, as many indexes from $R$, as the number of crashed threads, 
are not returned, whereas the rest are returned exactly once.

Algorithm~\ref{alg:counter} uses two counters. The first, $\mathit{cntEM}$,
is used in the expeditive mode and is updated by simple writes
(line~\ref{alg:co:em:inc-elements}),
whereas the second, $\mathit{cntSM}$, is used in the standard mode and
is updated using \FAI\ (line~\ref{alg:co:FAI-pos}).
As long as no helpers exist ($\mathit{h = \False}$), the owner thread $t$ executes 
a code similar to the sequential code implementing a counter (lines~\ref{alg:co:em:acquire-pos}-
\ref{alg:co:em:checkh1}). When a helper arrives, it sets $\mathit{cntSM}$ to the current
value of $cntEM$ (lines~\ref{alg:co:em:ownerFlag}, \ref{alg:co:em:cntSM-init}),
and uses \FAI\ on $\mathit{cntSM}$ (line~\ref{alg:co:FAI-pos}) to get assigned new indexes. 

Lines~\ref{alg:co:em:elements-bot}-\ref{alg:co:FAI-pos} provide a mechanism for appropriately
synchronizing $t$ with the helper threads when they arrive. Assume that $t$ has read the
value $x$ in $\mathit{cntEM}$, and it is ready to write $x+1$ in $\mathit{cntEM}$,
when two helper threads $t_h$ and $t'_h$ arrive.
Thread $t_h$ reads $x$ in $\mathit{cntEM}$, then $t$ writes $x+1$ 
in $\mathit{cntEM}$ and $t'_h$ reads  $x+1$ in $\mathit{cntEM}$.
Depending on which of the two values will be written in $\mathit{cntSM}$
and the order in which the \FAI\ instructions will be executed, 
one of $t_h$, $t'_h$ may read $x$ in $\mathit{cntSM}$ and return it.
% 
In this case, $t$ should avoid returning $x$ a second time (to ensure the semantics of
a counter object). To achieve this, helpers record in $\mathit{first}$ the value with
which they initialize $\mathit{cntSM}$. 
% 
If $t$ discovers that helpers exist (lines~\ref{alg:co:em:checkh1},~\ref{alg:co:em:ownerFlag}), 
%is the thread to successfully execute the \CAS\ of line~\ref{27}, it stores
%the value $x+1$ there, so helper threads will return this or higher values. 
%In this case, it is safe for $t$ to return $x$ (line~\ref{28}).
%Otherwise, $t$ 
it may have to read $\mathit{first}$ to figure out which is the first value recorded
in $\mathit{cntSM}$. If it is $x$, it  retries
(line~\ref{alg:co:em:continue2}). Note that helpers should first agree on the value 
to record in $\mathit{cntSM}$ with the \CAS\ of line~\ref{alg:co:em:casfirst} and then all 
use this value to initialize $\mathit{cntSM}$ (lines~\ref{alg:co:em:pos2}-~\ref{alg:co:em:cntSM-init}).


\begin{algorithm}[t]
    \footnotesize
    \caption{Pseudocode for \NextIndex.}
    \label{alg:counter}
    \begin{algorithmic}[1]

        \State \textbf{Shared variables:}
        \State int $\mathit{cntEM}$ (initially $0$), $\mathit{cntSM}$ (initially $\bot$)

        \vspace{2mm}

        \Function{NextIndex}{$\mathit{boolean *h}$} \textbf{returns} $\langle \mathit{int, \Boolean} \rangle$
            \State int $\mathit{pos}$
            \State int $\mathit{ownerFlag} \gets \False$
            
            \While{\True}
                \If{$\mathit{*h = \False}$} \label{alg:co:em}
                    \State $\mathit{pos} \gets \mathit{cntEM}$ \label{alg:co:em:acquire-pos}
                    \If{$\mathit{*h == \True}$} 
                        \State \textbf{continue} \label{alg:co:em:continue1}
                    \EndIf
                    \State $\mathit{cntEM} \gets \mathit{cntEM}+1$ \label{alg:co:em:inc-elements}
                    \If{$\mathit{*h == \False}$} 
                        \State \Return $\langle pos, \True \rangle$ \label{alg:co:em:checkh1}
                    \EndIf
                    \State $\mathit{ownerFlag} \gets \True$ \label{alg:co:em:ownerFlag}
                \EndIf

                \If{$\mathit{cntSM = \bot}$} \label{alg:co:em:elements-bot}
                    \If{$\mathit{ownerFlag \neq \True}$} \label{alg:co:em:ifHelper}
                        \State $\mathit{pos} \gets \mathit{cntEM}$ \label{alg:co:em:ownerFlag1}
                        \State $\mathit{\CAS(first, \bot, pos)}$ \label{alg:co:em:casfirst}
                        \State $\mathit{pos} \gets first$ \label{alg:co:em:pos2}
                    \EndIf
                    \State $\mathit{v} \gets \CAS(cntSM, \bot, pos)$ \label{alg:co:em:cntSM-init}
                    \If{$\mathit{v == \bot}$ \textbf{and} $\mathit{ownerFlag = \True}$}
                        \State \Return $\langle pos, \False \rangle$ \label{alg:co:em:ifOwner}
                    \ElsIf{$\mathit{first < cntEM}$} 
                        \State \textbf{continue} \label{alg:co:em:continue2}
                    \EndIf
                \EndIf

                \State $\mathit{pos} \gets \FAI(\mathit{cntSM})$ \label{alg:co:FAI-pos}
                \State \Return $\langle pos, \False \rangle$ \label{alg:co:em:return}
            \EndWhile

        \EndFunction

    \end{algorithmic}
\end{algorithm}