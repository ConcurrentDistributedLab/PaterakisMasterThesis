\documentclass[a4paper,11pt,twoside,openany]{book}

\usepackage{tipx}
\usepackage[named]{libtex/algo}
\usepackage{algorithm}
\usepackage[noend]{algorithmic}
\usepackage{amssymb}
\setcounter{tocdepth}{3}
\usepackage{graphicx}
\usepackage{amsfonts}
\usepackage{epsfig}
\usepackage{multirow}
\usepackage{amsfonts,amsmath}
\usepackage{srcltx}
\usepackage{multirow}
\usepackage{dsfont}
\usepackage{booktabs}
\usepackage{subfloat}
\usepackage{color}
\usepackage[table]{xcolor}
\usepackage{tabularx}
\usepackage{subfig}
\usepackage{url}
\usepackage{makeidx}
\usepackage{libtex/thesisStyle}


%%%%%%%%%%%%%
\long\def\comment#1{}


\newlength{\figtblfootnotemargin}
\newlength{\figtblfootnotewidth}

\setlength{\figtblfootnotemargin}{15mm}
\setlength{\figtblfootnotewidth}{\textwidth}
\addtolength{\figtblfootnotewidth}{-\figtblfootnotemargin}


%---------------------------------

\newcommand{\db}[1]{
    {\tt #1}
}

%---------------------------------
\newcommand{\outcomment}[1]{
    %\comment{
    \begin{quote}
       {\scriptsize
       {\em OC:} #1
%        #1
       }
    \end{quote}
    %}
}

%---------------------------------

\newcommand{\myheader}[1]{
    \godown
    $\bullet$ {\bf #1} \nn
}

%---------------------------------

\newcommand{\SquareOfSize}[2]{
 \fbox{\hsize #1cm \hbox to #1cm{\vbox{#2}}}
}

\newcommand{\figSquare}[1]{
   \hfill
        \SquareOfSize{13}
            { \vparag \vparag #1 \vparag \vparag }
    \hfill
}

\newcommand{\figSquareLarge}[1]{
   \hfill
        \SquareOfSize{16}
            { \vparag \vparag #1 \parag \vparag }
    \hfill
}

\newcommand{\SQUARE}[1]{
        \SquareOfSize{15}{#1}
}

\newcommand{\SquareSlite}[1]{
        \SquareOfSize{16}{#1}
}


\newcommand{\Syntax}[2]{ ``#1'', where #2 }

\newcommand{\SYNTAX}[2]{
    \SquareOfSize{14}{ \small
        #1 \hfill #2
    }
}

\newcommand{\UpdateProcess}[1]{
    \SquareOfSize{14}{  \bf
        #1
    }
}


\newcommand{\mathdef}[1]
{
    {\small
    \bq {\em Definition.}
        #1
    \eq
    }
}

\newcommand{\pqiCom}[2]{
    \parbox[t]{2cm}{#1} \ \parbox[t]{13.5cm}{ #2}
}

\newcommand{\KEYPOINT}[2]{
\SQUARE{\parbox[t]{3cm}{\small \em  #1} \ \parbox[t]{10cm}{{\bf #2}}}}

%\newcommand{\KEYPOINT}[2]{
%\SQUARE{\small \em  #1: \ \ \parbox[t]{10cm}{{\bf #2}}}}

%___________________________________________________________
\newcommand{\conclusion}[3]{
 \bc
    $ \left.  \begin{array}{r}
                    #1 \\
                    #2
                \end{array}
                    \right\}  \rightarrow #3
            $
 \ec
}
%___________________________________________________________

%-------- Update Operations --------------

\newcommand{\BriefDescription}{$\bullet$
    {\em ������� ���������:} \\}

\newcommand{\CheckedConstraint}
   {$\bullet$ {\em ���������� ������� ����������� :} \\}

\newcommand{\ExecutionAnalysis}[1]{
    $\bullet$ {\em ��������� ��������� :} \\
    \parbox[t]{2cm}{\hspace{2cm}} \ \parbox[t]{12cm}{ {\small #1} }
}

\newcommand{\SemanticAnalysis}[1]{
    $\bullet$ {\em ������������ �������������� ��������� :} \\
    \parbox[t]{2cm}{\hspace{2cm}} \ \parbox[t]{12cm}{ {\small #1} }
}

\newcommand{\namecheck}{ ������� ��������� ���� ������������ }

%--------  ---------------- --------------

\newcommand{\EFupdate}[3]{
    \item  {\scriptsize �����}: #1 {\scriptsize ��������} : $#2$

    \hspace*{1cm} \parbox[t]{12cm}{#3}
}

\newcommand{\visiblecomment}[1]{
  \underline{\centerline{ \bf ������}}\\
     \vspace{0.3cm}
     \bc
      \parbox[t]{14cm}{
            \small \em #1
      } \\
    \ec
  \underline{\centerline{ \bf ����� �������}}
}

\newcommand{\slidefigure}[3]{
% \slidefigure{#1 figure_text}
%             {#2 figure_name}
%             {#3 height_in_mm}

    \begin{figure}[ht]
                    \comment{\rule[0.2in]{\textwidth}{.01in}}
        {\Large \mbox{#1}}
        \centerline{
            \psfig{figure=#2,height=#3}
        }
    \end{figure}
}



\newcommand{\putsinglefigureR}[2]{
% \putsinglefigure{#1 figure_name}
%                 {#2 height_in_mm}

    \begin{figure}[htbp]
        \psfig{figure=#1,height=#2}
    \end{figure}
}

\newcommand{\putsinglefigure}[2]{
% \putsinglefigure{#1 figure_name}
%                 {#2 height_in_mm}

%   \begin{figure}[htbp]            <- not supported in landscape !
        \centerline{
        \psfig{figure=#1,height=#2}
        }
%   \end{figure}
}

\newcommand{\putsfigure}[2]{
        \psfig{figure=#1,height=#2}
}

\newcommand{\putfigure}[5]{
% \putfigure{#1 figure_name}
%           {#2 height_in_mm}
%           {#3 figure_label}
%           {#4 figure_title}
%           {#5 figure_text}

    \begin{figure}[htbp]
        \centerline{ \psfig{figure=#1,height=#2}}
        %\vspace*{-0.3cm} % added 4/5/2001 for EJM conf
        \caption{#4}{\vspace*{0.01cm}  \centerline{ \parbox[t]{13cm}{\footnotesize #5}}}
        \label{#3}
        %\vspace*{-0.6cm} % added 4/5/2001 for EJM conf
    \end{figure}
}



\newcommand{\putfigureVLDB}[5]{
% \putfigure{#1 figure_name}
%           {#2 height_in_mm}
%           {#3 figure_label}
%           {#4 figure_title}
%           {#5 figure_text}

    \begin{figure}[htbp]
        \centerline{ \psfig{figure=#1,height=#2}}
        \vspace*{-0.3cm} % added 4/5/2001 for EJM conf
        \caption{#4}{\vspace*{0.01cm}  \centerline{ \parbox[t]{9cm}{\footnotesize #5}}}
        \label{#3}
        \vspace*{-0.6cm} % added 4/5/2001 for EJM conf
    \end{figure}
}


\newcommand{\putfigureSV}[5]{
% \putfigurSVe{#1 figure_name}  SV = for Springer-Verlag
%           {#2 height_in_mm}
%           {#3 figure_label}
%           {#4 figure_title}
%           {#5 figure_text}

    \begin{figure}[htbp]
        \vspace*{-0.2cm}
        %\vspace*{-0.4cm}
        \centerline{ \psfig{figure=#1,height=#2}}
        \vspace*{-0.3cm}
        \caption{#4}{\vspace*{0.02cm}  \centerline{ \parbox[t]{10cm}{ #5}}}
        \label{#3}
        \vspace*{-0.8cm}
    \end{figure}
}


\newcommand{\putgeneralfigureNEW}[4]{
%           {#1 figure body}
%           {#2 figure_label}
%           {#3 figure_title}
%           {#4 figure_text}
    \begin{figure}[htbp]
        #1
        \vspace*{-0.5cm} % added 4/5/2001 for EJM conf
        \caption{#3}{\vspace*{0.001cm} \centerline{ \parbox[t]{13cm}{\footnotesize #4}}}
        %\caption{#3}{\vspace*{0.1cm} \centerline{ \parbox[t]{\textwidth}{\footnotesize #4}}}
        \label{#2}
        \vspace*{-0.4cm} % added 4/5/2001 for EJM conf
    \end{figure}
}

\newcommand{\putgeneralfigure}[4]{
%           {#1 figure body}
%           {#2 figure_label}
%           {#3 figure_title}
%           {#4 figure_text}

    \begin{figure}[htbp]
        #1
        \vspace*{-0.5cm} % added 4/5/2001 for EJM conf
        \caption{#3}{\vspace*{0.001cm} \centerline{ \parbox[t]{13cm}{\footnotesize #4}}}
        %\caption{#3}{\vspace*{0.1cm} \centerline{ \parbox[t]{\textwidth}{\footnotesize #4}}}
        \label{#2}
        \vspace*{-0.4cm} % added 4/5/2001 for EJM conf
    \end{figure}
}


\newcommand{\putTwoFiguresEngl}[6]{
    \begin{center}
    \begin{figure}
        \begin{minipage}[b]{0.5\textwidth}  %{7cm}
            \centerline{\putsfigure{#1}{#3}}
            \vspace*{3mm} \centerline{(a)}
        \end{minipage}
        \hspace*{0.5cm}
        \begin{minipage}[b]{0.5\textwidth}  %{7cm}
            \centerline{\putsfigure{#2}{#3}}
            \vspace*{3mm} \centerline{(b)}
        \end{minipage}
        \caption{#5}{\vspace*{0.1cm}
            \centerline{ \parbox[t]{13cm}{\footnotesize #6}}}
        \label{#4}
        \vspace*{-5mm} % added 15/9/2004
    \end{figure}
    \end{center}
}



\newcommand{\putTwoFiguresEnglSquare}[6]{
    %\begin{center}
    \begin{figure}
        \begin{minipage}[b]{85mm}
            \centerline{\fbox{\putsfigure{#1}{#3}}}
            \vspace*{3mm} \centerline{(a)}
        \end{minipage}
        \begin{minipage}[b]{85mm}
            \centerline{\fbox{\putsfigure{#2}{#3}}}
            \vspace*{3mm} \centerline{(b)}
        \end{minipage}
        \vspace*{-0.3cm}
        \caption{#5}{\vspace*{0.1cm}
            \centerline{ \parbox[t]{13cm}{\footnotesize #6}}}
        \label{#4}
        \vspace*{-8mm} % added 15/9/2004
    \end{figure}
    %\end{center}
}

\newcommand{\putFourFiguresEnglSquare}[8]{
    %\begin{center}
    \begin{figure}
        \begin{minipage}[b]{40mm}
            \centerline{\fbox{\putsfigure{#1}{#5}}}
            \vspace*{3mm} \centerline{(a)}
        \end{minipage}
        \begin{minipage}[b]{40mm}
            \centerline{\fbox{\putsfigure{#2}{#5}}}
            \vspace*{3mm} \centerline{(b)}
        \end{minipage}
        \begin{minipage}[b]{40mm}
            \centerline{\fbox{\putsfigure{#3}{#5}}}
            \vspace*{3mm} \centerline{(c)}
        \end{minipage}

        \begin{minipage}[b]{40mm}
            \centerline{\fbox{\putsfigure{#4}{#5}}}
            \vspace*{3mm} \centerline{(d)}
        \end{minipage}
        \caption{#7}{\vspace*{0.1cm}
            \centerline{ \parbox[t]{13cm}{\footnotesize #8}}}
        \label{#6}
        \vspace*{-1mm} % added 15/9/2004
        %\vspace*{-8mm} % added 15/9/2004
    \end{figure}
    %\end{center}
}

\newcommand{\putFigureAndText}[6]{
    \begin{figure}
        \begin{minipage}[b]{7cm}
            \centerline{\putsfigure{#1}{#2}}
            \vspace*{1mm} \centerline{(a)}
        \end{minipage}
        \begin{minipage}[b]{7cm}
            \bc
                #6
            \ec
            \vspace*{1mm} \centerline{(b)}
        \end{minipage}
        \caption{#4}{\vspace*{-2mm}
            \centerline{ \parbox[t]{13cm}{\footnotesize #5}}}
        \label{#3}
         %\vspace*{-10mm} % added 25/10/2001 for WISE conf
         %\vspace*{-5mm} % added 25/10/2001 for SETNT conf
    \end{figure}
}

% lala
\newcommand{\putFigureAndTextSV}[6]{
    \begin{figure}
        \vspace*{-0.4cm}   % SV
        \begin{minipage}[b]{6cm}
            \centerline{\putsfigure{#1}{#2}}
            \vspace*{1mm} \centerline{(a)}
        \end{minipage}
        \begin{minipage}[b]{6cm}
            \bc
                #6
            \ec
            \vspace*{1mm} \centerline{(b)}
        \end{minipage}
        \caption{#4}{\vspace*{-2mm}
            \centerline{ \parbox[t]{13cm}{\footnotesize #5}}}
        \label{#3}
        \vspace*{-8mm} %
    \end{figure}
}

\newcommand{\putFigureAndTextVLDB}[6]{
    \begin{figure}[h]
        \begin{minipage}[b]{7cm}
            \centerline{\putsfigure{#1}{#2}}
            \vspace*{1mm} \centerline{(a)}
        \end{minipage}
        \begin{minipage}[b]{7cm}
            \bc
                #6
            \ec
            \vspace*{1mm} \centerline{(b)}
        \end{minipage}
        \caption{#4}{%\vspace*{-2mm}
            \centerline{ \parbox[t]{85mm}{\footnotesize #5}}}
        \label{#3}
         %\vspace*{-10mm} % added 25/10/2001 for WISE conf
         %\vspace*{-5mm} % added 25/10/2001 for SETNT conf
    \end{figure}
}



\newcommand{\putTwoFigures}[6]{
    %\begin{center}
    \begin{figure}
        \begin{minipage}[b]{7cm}
            \centerline{\putsfigure{#1}{#3}}
            \vspace*{3mm} \centerline{(�)}
        \end{minipage}
        \begin{minipage}[b]{7cm}
            \centerline{\putsfigure{#2}{#3}}
            \vspace*{3mm} \centerline{(�)}
        \end{minipage}
        \caption{#5}{\vspace*{0.1cm}
            \centerline{ \parbox[t]{13cm}{\footnotesize #6}}}
        \label{#4}
    \end{figure}
    %\end{center}
}


\newcommand{\putTwoSliteFiguresB}[3]{
    \parbox[b]{7cm}{
            \centerline{\putsfigure{#1}{#3}}
    } \
    \parbox[b]{7cm}{
            \centerline{\putsfigure{#2}{#3}}
    }
}

\newcommand{\putTwoSliteFigures}[3]{
    \begin{figure}[h]
        \begin{minipage}[b]{7cm}
            \centerline{\putsfigure{#1}{#3}}
            %\vspace*{3mm} \centerline{(�)}
        \end{minipage}
        \begin{minipage}[b]{7cm}
            \centerline{\putsfigure{#2}{#3}}
            %\vspace*{3mm} \centerline{(�)}
        \end{minipage}
    \end{figure}
}



\newcommand{\putfigureSquare}[5]{
% \putfigure{#1 figure_name}
%           {#2 height_in_mm}
%           {#3 figure_label}
%           {#4 figure_title}
%           {#5 figure_text}

    \begin{figure}[htbp]
        \centerline{ \fbox{\psfig{figure=#1,height=#2}}}
        %\vspace*{-0.3cm} % added 4/5/2001 for EJM conf
        \vspace*{-0.3cm}
        \caption{#4}{\vspace*{0.01cm}  \centerline{ \parbox[t]{13cm}{\footnotesize #5}}}
        \label{#3}
        %\vspace*{-0.6cm} % added 4/5/2001 for EJM conf
        \vspace*{-0.5cm}
    \end{figure}
}



\newcommand{\puttable}[4]{
% \puttable {#1 table_label}
%           {#2 table_title}
%           {#3 table_text}
%           {#4 table_body}
    \begin{table}[htbp]
        \begin{center}
            #4
            \caption{#2}{\vspace*{0.1cm} \centerline{ \parbox[t]{13cm}{\footnotesize #3}}}
            \label{#1}
        \end{center}
        \vspace*{-0.8cm} % added 25/10/2001 for WISE conf
    \end{table}
}

\newcommand{\putwidetable}[4]{
% \puttable {#1 table_label}
%           {#2 table_title}
%           {#3 table_text}
%           {#4 table_body}
    \begin{table}[htbp]
        %\begin{center}
            \hspace*{5mm}
            #4
            \caption{#2}{\vspace*{0.1cm} \centerline{ \parbox[t]{16cm}{\footnotesize #3}}}
            \label{#1}
        %\end{center}
        %\vspace*{-0.8cm} % added 25/10/2001 for WISE conf
    \end{table}
}

\newcommand{\puttableSV}[4]{  % caption on top
% \puttableSV{#1 table_label}   SV = for Springer-Verlag format
%           {#2 table_title}
%           {#3 table_text}
%           {#4 table_body}
    \begin{table}[htbp]
        \begin{center}
            \vspace*{-3mm} % 8-3-04
            \caption{#2}{\vspace*{-3mm} \centerline{ \parbox[t]{10cm}{ #3}}}
            \vspace*{-3mm} % 7-11-05
            \label{#1}
            #4
        \end{center}
    \end{table}
    \vspace*{-2mm} % 8-3-04
}

\newcommand{\puttableSVold}[4]{  //old version: caption at bottom
% \puttableSV{#1 table_label}   SV = for Springer-Verlag format
%           {#2 table_title}
%           {#3 table_text}
%           {#4 table_body}
    \begin{table}[htbp]
        \begin{center}
            #4
            \caption{#2}{\vspace*{0.01cm} \centerline{ \parbox[t]{10cm}{ #3}}}
            \label{#1}
        \end{center}
    \end{table}
}

\newcommand{\puttableSIGMOD}[4]{  % caption on top
% \puttableSV{#1 table_label}   SV = for Springer-Verlag format
%           {#2 table_title}
%           {#3 table_text}
%           {#4 table_body}
    \begin{table}[htbp]
        \begin{center}
            %\vspace*{-4mm} % 8-3-04
            \caption{#2}{\centerline{ \parbox[t]{13cm}{ #3}}}
            %\vspace*{-4mm} % 8-3-04
            \label{#1}
            #4
        \end{center}
    \end{table}
    %\vspace*{-5mm} % 8-3-04
}

\newcommand{\putFourTables}[7]{
% \puttable {#1 table_label}
%           {#2 table_title}
%           {#3 table_text}
%           {#4 table_body1}
%           {#5 table_body2}
%           {#6 table_body3}
%           {#7 table_body4}
    \begin{table}[htbp]
        \begin{center}
            #4 #5 #6 #7
            \caption{#2}{\vspace*{0.1cm} \centerline{ \parbox[t]{13cm}{\footnotesize #3}}}
            \label{#1}
        \end{center}
    \end{table}
}





\newcommand{\putalgorithm}[2]{
% \putalgorithm {#1 alg_label}^M
%               {#2 alg_body}^M
    \begin{figure}[htbp]
        %\underline{\hspace*{16cm}}
        {\bf Algorithm} \label{#1}.

        #2

        %\underline{\hspace*{16cm}}
    \end{figure}
}



% Changes page margins :
%   eg \begin{changemargin}{-1cm}{-1cm}  ( widens page )
%   eg \begin{changemargin}{2cm}{2cm}    ( narrows page )
%
\newenvironment{changemargin}[2]{\begin{list}{}{
         \setlength{\topsep}{0pt}
          \setlength{\leftmargin}{0pt}
         \setlength{\rightmargin}{0pt}
         \setlength{\listparindent}{\parindent}
         \setlength{\itemindent}{\parindent}
         \setlength{\parsep}{0pt plus 1pt}
         \addtolength{\leftmargin}{#1}\addtolength{\rightmargin}{#2}
         }\item }{\end{list}
}

\newsavebox{\wholeWidthLine}
\sbox{\wholeWidthLine} {\rule[0.1in]{\textwidth}{.01in}}

\newcommand{\Figure}[5] {
%           {#1 figure_body}
%           {#2 figure_label}
%           {#3 figure_title}
%           {#4 figure_text}
%           {#5 figure_placement}
    \begin{figure}[#5]
        %\usebox{\wholeWidthLine}
        \vspace*{2mm}
        #1
        \caption{#3}{\
            \cl{\parbox[t]{\figtblfootnotewidth}{\footnotesize #4}}}
        \label{#2}
        \vspace*{2mm}
        %\usebox{\wholeWidthLine}
    \end{figure}
}


\newcommand{\putAndPlaceGeneralFigure}[5]{
%           {#1 figure body}
%           {#2 figure_label}
%           {#3 figure_title}
%           {#4 figure_text}
%           {#5 figure_placement}
    \begin{figure}[#5]
        %\usebox{\wholeWidthLine}
        \bc
            #1
        \ec
        \caption{#3}{\
            \centerline{\parbox[t]{\figtblfootnotewidth}{\footnotesize #4}}}
        \label{#2}
        %\usebox{\wholeWidthLine}
    \end{figure}
}


\newcommand{\abstrOfeq}[0]{
     \sqsubseteq
}
\newcommand{\abstrOf}[0]{
     \sqsubset
}

%\newcommand{\notisA}[0]{
     %isA \hspace*{-0.6cm} \not  \hspace*{0.6cm}
%}


\newcommand{\mysignature}{
    \parbox[t]{7cm}{
        \begin{center}
            {\normalsize ������� �. ���������} \\
            {\small MSc Computer Science} \\
            {\scriptsize ���: 093 47 13 11, (081) 256 561 }
        \end{center}
    }
}


% Frequently Used commands

\newcommand{\bq}{\begin{quote}}
\newcommand{\eq}{\end{quote}}
\newcommand{\be}{\begin{enumerate}}
\newcommand{\ee}{\end{enumerate}}
\newcommand{\bi}{\begin{itemize}}
\newcommand{\ei}{\end{itemize}}
\newcommand{\bie}{\begin{itemize}\begin{enumerate}}
\newcommand{\eie}{\end{enumerate}\end{itemize}}
\newcommand{\ba}{\begin{array}}
\newcommand{\ea}{\end{array}}
\newcommand{\btbl}{\begin{tabular}}
\newcommand{\etbl}{\end{tabular}}
\newcommand{\bequ}{\begin{displaymath}}
\newcommand{\eequ}{\end{displaymath}}
\newcommand{\bequa}{\begin{eqnarray*}}
\newcommand{\eequa}{\end{eqnarray*}}
\newcommand{\bc}{\begin{center}}
\newcommand{\ec}{\end{center}}
\newcommand{\btab}{\begin{tabbing}}
\newcommand{\etab}{\end{tabbing}}
\newcommand{\cl}{\centerline}

\newcommand{\mysection}[1]{
    \centerline{\LARGE \bf #1}
    \godown
}

\newcommand{\goright}{\hspace*{0.5cm}}
\newcommand{\goRight}{\goright \goright}
\newcommand{\goRIGHT}{\goRight \goRight}

\newcommand{\bul}{ $\bullet$ }
\newcommand{\goup}{ \vspace*{-0.5cm}}
\newcommand{\goUp}{ \goup \goup}
%\newcommand{\goUP}{ \vspace*{-0.5cm}}


\newcommand{\vparag}{ \vskip 5mm}
\newcommand{\godown}{ \vspace*{0.3cm}}
\newcommand{\goDown}{ \godown \godown}
\newcommand{\goDOWN}{ \goDown \goDown }

% set operator : not in
\newcommand{\noin}{\in \hspace{-0.33cm} / \hspace{0.3cm} }

%=========== MATH SYMBOLS ==============

\newcommand{\FALL}{\;\forall \;}
\newcommand{\mybox}[1]{\mbox{\letterspace #1 \letterspace }}
\newcommand{\card}[1]{
        \letterspace\mid
         #1
        \mid\letterspace
}


%=========== MDE WRITING ==============

% mpr : prints its argument in math mode
%       It can also be called from non math mode
\def\mpr#1{\ifmmode #1 \else #1 \fi}

\newcommand{\TB}        {\mpr{TB}}

%\newcommand{\R}         {\mpr{R}}
\newcommand{\RIN}       {\mpr{R^{in}}}
\newcommand{\RINALL}    {\mpr{R^{in}_{impl}}}
\newcommand{\RINB}      {\mpr{R^{in}_{sys}}}
\newcommand{\RINU}      {\mpr{R^{in}_{u}}}
\newcommand{\RINUtonos} {\mpr{R^{in'}_{u}}}
\newcommand{\RISA}      {\mpr{R^{isA}}}
\newcommand{\RISATR}    {\mpr{R^{isA}_{tr}}}
\newcommand{\RISAB}     {\mpr{R^{isA}_{sys}}}
\newcommand{\RISAU}     {\mpr{R^{isA}_{u}}}
\newcommand{\RISAUtonos}{\mpr{R^{isA'}_{u}}}
\newcommand{\RFROM}     {\mpr{R^{from}}}
\newcommand{\RTO}       {\mpr{R^{to}}}

%\newcommand{\TO}        {\mpr{O}}
\newcommand{\TOB}       {\mpr{O_{sys}}}
\newcommand{\TOU}       {\mpr{O_{u}}}
\newcommand{\PV}        {\mpr{PV}}
\newcommand{\Integer}   {\mpr{Integer}}
\newcommand{\Real}      {\mpr{Real}}
\newcommand{\String}    {\mpr{String}}
\newcommand{\Nat}       {\mpr{N}}
\newcommand{\LN}    {\mpr{L}}
\newcommand{\LF}    {\mpr{{\cal L}}}

\newcommand{\TI}        {\mpr{I}}
\newcommand{\TIU}       {\mpr{I_{u}}}
\newcommand{\TA}        {\mpr{A}}
\newcommand{\TC}        {\mpr{C}}
\newcommand{\TIC}       {\mpr{I_{c}}}
\newcommand{\TAC}       {\mpr{A_{c}}}

\newcommand{\type}      {\mpr{type}}
\newcommand{\level}     {\mpr{level}}
\newcommand{\typeb}     {\mpr{type_{sys}}}
\newcommand{\levelb}    {\mpr{level_{sys}}}

\newcommand{\SemT}      {\mpr{SemanticType}}
\newcommand{\SynT}      {\mpr{SyntacticType}}

\newcommand{\semT}      {\mpr{semType}}
\newcommand{\synT}      {\mpr{synType}}
\newcommand{\semTb}     {\mpr{semType_{sys}}}
\newcommand{\synTb}     {\mpr{synType_{sys}}}

\newcommand{\sem}       {\mpr{sem}}
\newcommand{\syn}       {\mpr{syn}}

\newcommand{\PBT}       {\mpr{PBT}}
\newcommand{\TP}        {\mpr{TP}}


%-- TELOS QUERIES

% writting TELOS statements
\newcommand{\telos}{Telos\letterspace}
\newcommand{\clio}{
    % $\cal{K}${\em �}$\cal{EI}${\em �} \letterspace
    {\em �����}
}
\newcommand{\tellretell}{TELL/RETELL\letterspace}
\newcommand{\export}{{\em Export} \letterspace}

\newcommand{\letterspace}{\hspace*{1mm}}

\newcommand{\ison}{\;=\;}
\newcommand{\isA}{\mbox{\bf isA} \letterspace}
\newcommand{\instOf}{\mbox{\bf instOf}}

\newcommand{\Tell}{{\bf TELL Individual} \letterspace}
\newcommand{\Retell}{{\bf RETELL Individual} \letterspace}
\newcommand{\End}{{\bf end} \letterspace}
\newcommand{\inT}{{\bf in} \letterspace}
\newcommand{\scl}{{\bf S\_Class} \letterspace}
\newcommand{\tok}{{\bf Token} \letterspace}

\newcommand{\printPQI}[1]{ \mbox{\bf #1} \letterspace }

\newcommand{\sysclass}{ \printPQI{sysclass}}
\newcommand{\scn}{      \printPQI{scn}}
\newcommand{\gfv}{      \printPQI{gfv}}
\newcommand{\gtv}{      \printPQI{gtv}}
\newcommand{\glf}{      \printPQI{glf}}
\newcommand{\gilf}{     \printPQI{gilf}}
\newcommand{\gilt}{     \printPQI{gilt}}
\newcommand{\glt}{      \printPQI{glt}}
\newcommand{\glfc}{     \printPQI{glfc}}
\newcommand{\gfnc}{     \printPQI{gfnc}}
\newcommand{\gtnc}{     \printPQI{gtnc}}
\newcommand{\gasb}{     \printPQI{gasb}}
\newcommand{\gsb}{      \printPQI{gsb}}
\newcommand{\gasc}{     \printPQI{gasc}}
\newcommand{\gc}{       \printPQI{gc}}
\newcommand{\gac}{      \printPQI{gac}}
\newcommand{\gai}{      \printPQI{gai}}
\newcommand{\gi}{       \printPQI{gi}}
\newcommand{\gsc}{      \printPQI{gsc}}
\newcommand{\gSc}{      \printPQI{gSc}}
\newcommand{\gaSc}{     \printPQI{gaSc}}
\newcommand{\gSi}{      \printPQI{gSi}}
\newcommand{\attrs}{    \printPQI{attrs}}
\newcommand{\focus}     {{\em s \letterspace}}

%---------- CHAPTER 4 -------------------
\newcommand{\BPE}[1]{ \mbox{\bf #1 \letterspace} }

\newcommand{\CreateNode}        {\BPE{CreateIndividual}}
\newcommand{\DeleteNode}        {\BPE{DeleteIndividual}}
\newcommand{\CreateAttribute}   {\BPE{CreateAttribute}}
\newcommand{\DeleteAttribute}   {\BPE{DeleteAttribute}}
\newcommand{\AddInstance}       {\BPE{AddInstance}}
\newcommand{\DeleteInstance}    {\BPE{DeleteInstance}}
\newcommand{\Rename}            {\BPE{Rename}}
\newcommand{\AddSubClass}       {\BPE{AddSubClass}}
\newcommand{\DelSubClass}       {\BPE{DeleteSubClass}}
\newcommand{\makecopy}          {\BPE{MakeCopy}}

%---tmpmaybe
\newcommand{\CreateNodeSet}     {{\em CreateNodeSet}}
\newcommand{\DeleteNodeSet}     {{\em DeleteNodeSet}}
\newcommand{\CreateAttributeSet}{{\em CreateAttributeSet}}
\newcommand{\DeleteAttributeSet}{{\em DeleteAttributeSet}}
\newcommand{\AddInstanceSet}    {{\em AddInstanceSet}}
\newcommand{\DeleteInstanceSet} {{\em DeleteInstanceSet}}
\newcommand{\RenameSet}         {{\em RenameSet}}
\newcommand{\AddSubClassSet}    {{\em AddSubClassSet}}
\newcommand{\DelSubClassSet}    {{\em DelSubClassSet}}

%---------- CHAPTER 5 -------------------
\newcommand{\EFID}[1]{ \mbox{\em \scriptsize #1 } }

\newcommand{\obj}{ \letterspace
                    \mbox{\em o}\letterspace
}
\newcommand{\aobj}{ \mbox{\em a}\letterspace}
\newcommand{\bobj}{ \mbox{\em b}\letterspace}
\newcommand{\fobj}{ \mbox{\em from}\letterspace}
\newcommand{\tobj}{ \mbox{\em to}\letterspace}
\newcommand{\sobj}{ \mbox{\em s}\letterspace}
\newcommand{\cobj}{ \mbox{\em c}\letterspace}
\newcommand{\iobj}{ \mbox{\em i}\letterspace}
\newcommand{\ObjectSet}{ \mybox{\em ObjectSet}}
\newcommand{\SysClass}{ \mbox{\em SysClass}\letterspace}

\newcommand{\CREATENODE}    {\EFID{CREATE\_INDIVIDUAL}}
\newcommand{\DELETENODE}    {\EFID{DELETE\_INDIVIDUAL}}
\newcommand{\RENAMENODE}    {\EFID{RENAME\_INDIVIDUAL}}
\newcommand{\CHANGECLASSES} {\EFID{CHANGE\_CLASSES}}
\newcommand{\CHANGESUPERCLASSES}{\EFID{CHANGE\_SUPERCLASSES}}
\newcommand{\CHANGEATTRIBUTES}  {\EFID{CHANGE\_ATTRIBUTES}}
\newcommand{\nullset}   {-\hspace*{-0.1cm}-}
%%\newcommand{\emptyset}    {0\hspace*{-0.1cm}\/}
%\newcommand{\concat}   {\letterspace\#\letterspace}
\newcommand{\concat}    {\letterspace\circ\letterspace}

\newcommand{\DC}{
        \letterspace {\bf DC}\letterspace
}
\newcommand{\DO}{
        \letterspace {\bf DO}\letterspace
}
\newcommand{\RC}{${\cal R}\;$}
\newcommand{\calA}{${\cal A}\;$}
\newcommand{\calB}{${\cal B}\;$}
\newcommand{\calE}{${\cal E}\;$}

%---------- CHAPTER 6 -------------------

\newcommand{\mmName}{ ���� }

\newcommand{\sneg}   { \mbox{\={$\! \! \! s$}} }

\newcommand{\Users}      {Users}
\newcommand{\SystemUsers}{SystemUsers}
\newcommand{\UserTypes}  {UserTypes}

\newcommand{\Ustx}       {U_{stx}}
\newcommand{\Usem}       {U_{sem}}
\newcommand{\Tasks}      {Tasks}
\newcommand{\tsem}      {t_{sem}}
\newcommand{\apon}          {applyon}
\newcommand{\aponsem}       {applyon_{sem}}
\newcommand{\aponsemt}      {applyon^{t}_{sem}}

\newcommand{\SetIds}    {ID}
\newcommand{\tids}      {S_{id}}
\newcommand{\tidsR}     {S^{-1}_{id}}

\newcommand{\pids}      {P^{expl}_{id}}
\newcommand{\nids}      {N^{expl}_{id}}
\newcommand{\pimids}    {P^{impl}_{id}}
\newcommand{\nimids}    {N^{impl}_{id}}

\newcommand{\ti}        {S_{i}}
\newcommand{\tiR}       {S^{-1}_{i}}
\newcommand{\tj}        {S_{j}}

\newcommand{\cro}       {CrObj}
\newcommand{\delo}      {DelObj}
\newcommand{\ren}       {REN}
\newcommand{\del}       {DEL}
\newcommand{\addaf}     {AddAF}
\newcommand{\addat}     {AddAT}
\newcommand{\delaf}     {DelAF}
\newcommand{\delat}     {DelAT}
\newcommand{\addin}     {AddIn}
\newcommand{\delin}     {DelIn}
\newcommand{\addsub}    {AddSub}
\newcommand{\delsub}    {DelSub}
\newcommand{\addsup}    {AddSup}
\newcommand{\delsup}    {DelSup}
\newcommand{\addclass}  {AddClass}
\newcommand{\delclass}  {DelClass}

\newcommand{\tcro}      {S_{\cro}}
\newcommand{\tcroR}     {S^-1_{\cro}}
\newcommand{\tdelo}     {S_{\delo}}
\newcommand{\tdeloR}    {S^-1_{\delo}}
\newcommand{\tren}      {S_{\ren}}
\newcommand{\tdel}      {S_{\del}}
\newcommand{\taddaf}    {S_{\addaf}}
\newcommand{\tdelaf}    {S_{\delaf}}
\newcommand{\taddat}    {S_{\addat}}
\newcommand{\tdelat}    {S_{\delat}}
\newcommand{\taddin}    {S_{\addin}}
\newcommand{\tdelin}    {S_{\delin}}
\newcommand{\taddsub}   {S_{\addsub}}
\newcommand{\tdelsub}   {S_{\delsub}}


\newcommand{\negS}      {{\scriptsize NEG}}
\newcommand{\posS}      {{\scriptsize POS}}
\newcommand{\noneS}     {{\scriptsize NONE}}

\newcommand{\QT}    {{\cal QT}}

\newcommand{\symboleq}  {\stackrel{����}{=}}

\newcommand{\qlevel}        {level}
\newcommand{\qlevelsub}     {levelsub}
\newcommand{\qsubs}         {subs}
\newcommand{\qgai}      {gai}
\newcommand{\qglf}      {glf}
\newcommand{\qglfgasb}  {glfgasb}

\newcommand{\Rules}     {{\cal R}}
\newcommand{\ARules}    {{\cal A}}
\newcommand{\Preds}     {{\cal PR}}
\newcommand{\PredTypes} {{\cal PRT}}
\newcommand{\PredSys}   {{\cal PR}_{sys}}
\newcommand{\PredIN}    {{\cal PR}_{in}}
\newcommand{\PredAIN}   {{\cal PR}_{ain}}
\newcommand{\PredASUB}  {{\cal PR}_{asub}}
\newcommand{\PredL}     {{\cal PR}_{links}}
\newcommand{\PredASUBL} {{\cal PR}_{sublinks}}

%\newcommand{\TRUE}  {{\small \mbox{TRUE}}}

%---------- CHAPTER 6  <OLDVERSION> ---------
%----- S Q L ---------

    \newcommand{\sql}[1]{{\bf #1}}

    \newcommand{\SELVIEW}[1]{
        \sql{SELECTION VIEW} {\em #1}
    }
    \newcommand{\SELVIEWslite}[1]{
        \sql{SELECTION VIEW} #1
    }

    \newcommand{\FROM}[2]{
        \hspace*{1cm} \sql{FROM} {\em #1} \sql{IN} {\em #2}
    }
    \newcommand{\FROMslite}[2]{
        \hspace*{1cm} \sql{FROM} #1 \sql{IN} #2
    }

    \newcommand{\WHERE}[1]{
        \hspace*{1cm} \sql{WHERE} {\em #1}
    }
    \newcommand{\WHEREslite}[1]{
        \hspace*{1cm} \sql{WHERE} #1
    }

%---------------

\newcommand{\oos}[1]{ \mbox{\bf #1} }

%--- UIDS
\newcommand{\printUID}[1]{ \mbox{\em #1} }


\newcommand{\SCuid}[0]      { \printUID{Schema} }

\newcommand{\INuid}[0]      { \printUID{IN} }
\newcommand{\IDuid}[0]      { \printUID{ID} }
\newcommand{\ISAuid}[0]     { \printUID{ISA} }
\newcommand{\NAMEuid}[0]    { \printUID{NAME} }
\newcommand{\ATTRSuid}[0]   { \printUID{ATTR} }
\newcommand{\INSTATTRSuid}[0]{\printUID{INST\_ATTRS} }


%---- Update Sets

\newcommand{\UT}{ UpdateTypes}
\newcommand{\US}{ UpdateState}
\newcommand{\UID}{ UpdateIDs}

\newcommand{\where}[0]{
    \mbox{ \letterspace ����  \letterspace}
}
\newcommand{\an}[0]{
    \mbox{ \letterspace ��  \letterspace}
}
\newcommand{\then}[0]{
    \mbox{ \letterspace ����  \letterspace}
}
\newcommand{\kai}[0]{
    \mbox{ \letterspace ���  \letterspace}
}
\newcommand{\whereT}[0]{
    \mbox{t $\in$ \UT}
}
\newcommand{\Change}{ \mbox{ Changable} }
\newcommand{\Invariant}{ \mbox{ Invariant} }

\newcommand{\Inv}[1]{ \mbox{ Inv[#1]} }
\newcommand{\Ch}[1]{ \mbox{ Ch[#1]} }

\newcommand{\Invv}[2]{ \mbox{ Inv[#1][#2]} }
\newcommand{\Chh}[2]{ \mbox{ Ch[#1][#2]} }

\newcommand{\InvExpl}[1]{ \Invv{Expl}{#1} }
\newcommand{\ChExpl}[1]{ \Chh{Expl}{#1} }

\newcommand{\InvSub}[1]{ \Invv{onSubs}{#1} }
\newcommand{\ChSub}[1]{ \Chh{onSubs}{#1} }

\newcommand{\InvAttr}[1]{ \Invv{onAttrsFrom}{#1} }
\newcommand{\ChAttr}[1]{ \Chh{onAttrsFrom}{#1} }

\newcommand{\InvAttrSub}[1]{ \Invv{onAttrsSubFrom}{#1} }
\newcommand{\ChAttrSub}[1]{ \Chh{onAttrsSubFrom}{#1} }

\newcommand{\Existencial}{\mbox{Existencial}}




\newtheorem{telos_rule}{�����������}
\newtheorem{my_rule}{�������}

\newcommand{\subjectCondition} {{\em ������� ��� \obj :}}
\newcommand{\argumentCondition}{{\em ������� ��� �������� :} }

\newcommand{\argset}[1]{{\em ������ ������� #1 :} }

%\newcommand{\DPATTRNAME}{{\large{{\bf ��}$_{attr}$}}}
%\newcommand{\DPISANAME}{{\large{{\bf ��}$_{isA}$}}}
%\newcommand{\DPINNAME}{{\large{{\bf ��}$_{in}$}}}
\newcommand{\DPATTRNAME}{{\large{{\bf SC}$_{attr}$}}}
\newcommand{\DPISANAME}{{\large{{\bf SC}$_{isA}$}}}
\newcommand{\DPINNAME}{{\large{{\bf SC}$_{in}$}}}

\newtheorem{DP_ATTR}{\DPATTRNAME}
\newtheorem{DP_ISA} {\DPISANAME}
\newtheorem{DP_IN}  {\DPINNAME}


%--- FOR SOURCE CODE

\newcommand{\printKey}[1]{\mbox{\bf #1}}
\newcommand{\mfunction} { \printKey{function} }
\newcommand{\mvar}      { \printKey{var} }
\newcommand{\mset}      { \printKey{SET} }
\newcommand{\mbegin}    { \printKey{begin} }
\newcommand{\mif}       { \printKey{if} }
\newcommand{\melse}     { \printKey{else} }
\newcommand{\mend}      { \printKey{end} }
\newcommand{\mreturn}   { \printKey{return} }

%%%%%%%%%%%%%



\renewcommand{\algorithmicrequire}{\textbf{Input:}}
\renewcommand{\algorithmicensure}{\textbf{Output:}}
\renewcommand{\algorithmiccomment}[1]{{\scriptsize //#1}}


% FOR WRITING GREEK %
\usepackage[greek,american]{babel}
\usepackage[libtex/iso-8859-7]{inputenc}

\newcommand{\selg}[0]{\selectlanguage{greek}}
\newcommand{\selam}[0]{\selectlanguage{american}}

%%%%%%%%%%%%%%%%%%%%%%%%%%
%\renewcommand{\baselinestretch}{1.5}
%\textheight=23cm \textwidth=16cm \voffset -1.5cm \hoffset -12.8mm
%%%%%%%%%%%%%%%%%%%%%%%%%%
\renewcommand{\baselinestretch}{1}\small\normalsize
\textwidth=390pt \oddsidemargin = 50pt \evensidemargin = 0pt
%%%%%%%%%%%%%%%%%%%%%%%%%%

\makeindex

\thispagestyle{empty}
\pagestyle{empty}
%%%%%%%%%%%%%%%%%%%%%%%%%%


%%%%%%%%%%%%%%%%%%%%%%%%%%
\long\def\symbolfootnote[#1]#2
    {\begingroup
    \def\thefootnote{\fnsymbol{footnote}}\footnote[#1]{#2}
    \endgroup}

\newcommand{\workperformedat}{\symbolfootnote[0]{This work has been
    performed at the University of Crete, School of Sciences and
Engineering, Computer Science Department.}}

\newcommand{\worksupportedby}{\symbolfootnote[0]{The work has been
    supported by the Foundation for Research and Technology - Hellas (FORTH),
    Institute of Computer Science (ICS).}}

\newcommand{\thesisdate}{March 2025 }
%%%%%%%%%%%%%%%%%%%%%%%%%%
























\begin{document}
\begin{titlepage}
\begin{center}

\LARGE \textbf{DFreSh: A Lock-Free Index for Real-Time Data Series Processing}\\[0.5cm]
\LARGE \textit{Paterakis George}\\[0.5cm]

\vfill

\normalsize{
Thesis submitted in partial fulfillment of the requirements for the\\[0.30cm]

\textit{Masters' of Science degree in Computer Science and Engineering}}\\[0.30cm]

University of Crete\\
School of Sciences and Engineering\\
Computer Science Department\\
Voutes University Campus, 700 13 Heraklion, Crete, Greece\\[0.5cm]

\vfill

\Large{Thesis Advisor: Assistant Prof. \emph{Panagiota Fatourou}}\\[0.5cm]

\vfill

\end{center}

\workperformedat{} \worksupportedby{}
\end{titlepage}

\cleardoublepage




















\thispagestyle{empty}

\newcommand{\thesistitle}{DFreSh: A Lock-Free Index for Real-Time Data Series Processing}
\newcommand{\owner}{Paterakis George}
\newcommand{\firstprof}{Panagiota Fatourou}
\newcommand{\secondprof}{Professor 2}
\newcommand{\thirdprof}{Professor 3}
\newcommand{\csdchair}{Antonios Argyros}

\begin{titlepage}

\begin{center}
\textsc{University of Crete}\\
\textsc{Computer Science Department}\\
\vspace{0.4cm}
\noindent {\textbf{\thesistitle{}}}\\
\vspace{0.4cm}
\noindent Thesis submitted by\\
\textbf{\owner{}}\\
in partial fulfillment of the requirements for the\\
Masters' of Science degree in Computer Science\\
\vspace{0.4cm} THESIS APPROVAL

\vspace{0.4cm}

\begin{tabular}{rl}
\\
Author: & \underline{\phantom{123456789012345678901234567890123456789012}}\\
    & \owner{}\\
    \\
    \\
    \\
Committee approvals: & \underline{\phantom{123456789012345678901234567890123456789012}}\\
    & \firstprof{}\\
    & {\small Assistant Professor, Thesis Supervisor}\\
    \vspace{0.2cm}
    \\
    \\
& \underline{\phantom{123456789012345678901234567890123456789012}}\\
    & \secondprof{}\\
    & {\small Professor, Committee Member}\\
    \vspace{0.2cm}
    \\
    \\
& \underline{\phantom{123456789012345678901234567890123456789012}}\\
    & \thirdprof{}\\
    & {\small  Professor, Committee Member}\\
    \vspace{0.2cm}
    \\
    \\
\hspace{1.4ex}Departmental approval: & \underline{\phantom{123456789012345678901234567890123456789012}}\\
    & \csdchair{}\\
    & {\small Professor, Director of Graduate Studies}\\
\end{tabular}
\\

\vfill Heraklion, \thesisdate{}
\end{center}

\thispagestyle{empty}

\end{titlepage}

\cleardoublepage








\thispagestyle{empty}
\begin{titlepage}
\begin{center}
\bc \Large{ \textbf{TITLE}} \ec \bc {\bf\large Abstract}\\\ec
\end{center}

Abstract

In this thesis, we present DFreSh, a real-time lock-free data series index that 
achieves high performance while maintaining robustness. DFreSh builds upon FreSh,
the current state-of-the-art in lock-free data series indexing. FreSh, in turn,
is based on Refresh, a generic framework designed to enable efficient lock-freedom
in locality-aware data series indexes. Refresh provides a systematic approach
to deriving high-performing lock-free versions of other locality-aware blocking
data structures.
Despite its strengths, FreSh has a critical limitation: 
it is a static index, incapable of supporting dynamic updates after construction.
This limitation renders FreSh unsuitable for real-time environments where data
is continuously generated and needs to be inserted-integrated into the index
without the need for complete rebuilding. Addressing this shortcoming,
DFreSh introduces dynamic update capabilities, making it a robust solution
for real-time applications while retaining the performance benefits of its predecessor.
We evaluated DFreSh through extensive experiments using both synthetic
and real-world datasets. The results demonstrate that DFreSh achieves
performance comparable to FreSh, with its lightweight mechanisms for
real-time data handling upholding the key principles of performance critical
to locality-aware data structures. 
\vfill
\end{titlepage}

\cleardoublepage


























\thispagestyle{empty}

\selg

\begin{titlepage}
\begin{center}
\bc \Large{ \textbf{������}} \ec \bc {\bf\large ��������}\\\ec
\end{center}


\vfill

\end{titlepage}


\cleardoublepage

















\bc \large{\textbf{�����������}} \ec



\cleardoublepage


\begin{flushright}
\emph{
\newline \newline \newline \newline \newline \newline \newline \newline
����� ������ ���} \\
\end{flushright}

\cleardoublepage



\selam

\pagestyle{plain}
\pagenumbering{roman}
\setcounter{page}{1}
\setcounter{tocdepth}{3}

% \tableofcontents
%     \addtocontents{toc}{\contentsline {chapter}{Table of Contents}{i}}

% \listoftables
%     \addtocontents{toc}{\contentsline {chapter}{List of Tables}{iii}}

% \listoffigures
%     \addtocontents{toc}{\contentsline {chapter}{List of Figures}{v}}

% \cleardoublepage

\pagenumbering{arabic}
%\renewcommand*\thesection{\arabic{section}}




 \pagestyle{headings}
%CHAPTER_1
\chapter{Introduction}
\label{chapter:introduction}


Processing large collections of data series is critical for a wide range of applications
~\cite{DBLP:journals/sigmod/Palpanas15,DBLP:journals/dagstuhl-reports/BagnallCPZ19,Palpanas2019}.
These applications depend on efficient similarity search, which aims to retrieve the most similar
data series to a given query $𝑄$.
Specifically, a similarity search operation returns a subset of data series from the collection 
that are closest to $𝑄$  based on a defined distance metric (or similarity function).

Beyond its standalone utility, similarity search is also a key component in various machine 
learning tasks, including clustering, classification, and outlier detection. However, 
performing similarity search at scale is computationally expensive due to two main factors 
\begin{itemize}
\item The massive size of modern data series collections, which often reach terabyte or petabyte scales.
\item The high dimensionality of data series (i.e. length) that modern applications need to analyze.
\end{itemize}

To overcome these challenges, state-of-the-art data series indexes
~\cite{DBLP:journals/pvldb/EchihabiZPB18,isax2plus,wang2013data,peng2018paris,parisplus,peng2020messi,PFP21-I,PFP21-II,hercules,dumpy}
employ data series summarization. These indexes build tree-based structures where each node 
stores summarized representations of data series, enabling efficient pruning of the dataset.
By filtering out irrelevant series early, these indexes significantly reduce the number of costly 
distance computations, improving query performance.
State-of-the-art such indexes~\cite{peng2020messi,parisplus,peng2021sing} 
use iSAX summaries~\cite{isax2plus,isaxfamily},thus we call them iSAX-based indexes.  
%
Due to their efficiency and scalability, iSAX-based indexes have been widely adopted across 
diverse application domains. These include operational health monitoring in data centers, 
vehicles, and manufacturing processes; Internet of Things (IoT) data analysis; environmental 
and climate monitoring; energy consumption analysis; financial market decision-making; 
telecommunications traffic analysis; medical diagnostics; web search optimization; 
and even agriculture, where they help identify pest infestations in crops.
%
To efficiently support these real-world use cases, state-of-the-art data series indexes
~\cite{DBLP:journals/pvldb/EchihabiZPB18,isaxfamily,peng2018paris,peng2020messi,PFP21-I,PFP21-II,hercules} 
leverage modern multicore architectures to achieve parallelism. However, most existing 
approaches rely on lock-based synchronization mechanisms, which introduce blocking behavior:
if a thread holding a lock fails or is delayed, other threads waiting on the lock are forced to stall,
leading to degraded performance.
While thread failures are relatively rare, they do occur in practice—for example, due to 
software bugs in applications using the index or issues in memory management systems. 
This is especially problematic for mission-critical applications, such as nuclear plant 
monitoring or gravitational-wave detection in astrophysics, where any delay in data processing 
can have significant consequences. Such latency-sensitive applications would greatly benefit
from our proposed approach, which eliminates blocking and ensures continuous progress.
%
While locks provide synchronization, they introduce well-known performance issues, 
such as priority inversion and convoying~\cite{F04}. Additionally, depending on how they
are used, locks can limit parallelism, leading to suboptimal performance

An alternative approach is lock-free synchronization, a well-studied technique in concurrent 
tree structures~\cite{EFRB10,EFHR14,FR2018,ABF+22} and other data structures~\cite{F04,HS08,FKR18}. 
Lock-free algorithms eliminate the need for locks, ensuring that the system as a whole 
always makes progress, regardless of thread delays or failures. This property is particularly 
beneficial for highly parallel workloads, where blocking mechanisms can create bottlenecks. 
%
In this work, we introduce FreSh, a lock-free data series index that maintains the 
correctness and strong performance of state-of-the-art lock-based indexes while avoiding 
their synchronization overheads. Additionally, we present DFreSh, a dynamic version of 
FreSh that extends its capabilities to support batches of updates while simultaneously 
answering queries. This enables real-time index updates, making it well-suited for streaming
data applications.
%
The relative performance of lock-free vs. lock-based approaches depends on the execution environment.
When threads are pinned to distinct cores, algorithms with fine-grained locks perform well. 
However, in oversubscribed settings (i.e., when there are more threads than available cores),
lock-based solutions can suffer performance degradation, as threads holding locks may be
de-scheduled, blocking others. Lock-free algorithms avoid such issues but can be more complex and,
in some cases, may not always outperform lock-based approaches when delays and failures are absent.

\noindent{\bf Challenges.} 
Achieving lock-freedom in data series indexes often requires the use of a {\em helping} mechanism,
where threads are made aware of the work other threads perform and can assist them when
needed—such as during delays or load imbalances. However, conventional helping mechanisms
tend to be costly and introduce significant overhead~\cite{F04,HS08,7515610,Williams2012CCI}.
This is one of the reasons the vast majority of the software stack still relies on locks.
Ensuring lock-freedom while maintaining the good performance of existing data series indexes 
remains a significant challenge.
%
SState-of-the-art data series indexes are designed to (a) maintain data locality
and (b) minimize synchronization, aiming for high parallelism. 
For example, they often partition the data into disjoint sets and assign a distinct thread to 
handle each set's data~\cite{parisplus,PFP21-I,PFP21-II}.
This design enables parallel and independent thread operations, reducing synchronization
and communication costs. These goals are easier to achieve with locks or barriers
~\cite{peng2018paris,peng2020messi,PFP21-I,PFP21-II}.
%
However, the way helping functions in conventional lock-free data structures is inherently
incompatible with these principles. Implementing helping in such indexes without compromising
these principles is a challenge.
%
Moreover, ensuring lock-freedom while maintaining load balancing and effective data pruning
are additional hurdles. State-of-the-art data series indexes involve multiple processing phases,
each using different data structures to optimize performance. Designing lock-based versions of
these structures is feasible, but creating lock-free versions that adhere to the communication
and synchronization principles of existing indexes is more difficult.
%
To develop a generalized approach for supporting lock-freedom in data series indexes systematically,
we must first analyze the design decisions of existing state-of-the-art indexes and the performance
principles they follow. We need to establish appropriate abstractions for the data series processing
stages, as well as design principles that ensure efficiency. Achieving these goals introduces further
challenges.
%
Lastly, a significant challenge in the context of dynamic updates, is transforming
a static index, like FreSh, into a dynamic one that supports batch insertions of updates while 
maintaining the same high performance. This dynamic index must ensure that queries continue to 
return consistent and correct results, while also guaranteeing that each query retrieves all the 
data seen by previous queries, including any updates made since the last query.

\noindent
{\bf Our approach.}
We introduce FreSh, a novel static lock-free data series index, and DFreSh, 
its dynamic counterpart that efficiently addresses all of the above challenges. 

Our experimental analysis demonstrate that FreSh matches the performance of 
MESSI~\cite{PFP21-I}, the state-of-the-art concurrent data-series index, which is lock-based. 
This confirms the efficiency of the helping mechanism we employ in FreSh. In many cases, 
FreSh even outperforms MESSI by allowing for higher parallelism during the construction of the
tree index.
It's important to note that if threads crash, MESSI (and other lock-based approaches
~\cite{peng2018paris,PFP21-I,PFP21-II,hercules}) will not terminate, which is 
why we did not provide experiments for this scenario. In contrast, 
FreSh always terminates successfully and correctly.
%
To achieve FreSh, we developed ReFreSh, a generic approach that can be applied to a
range of state-of-the-art blocking indexes to provide lock-freedom without 
incurring additional costs. 
%
ReFreSh introduces the concept of locality-aware lock-freedom, which integrates
key properties of data locality, high parallelism, low synchronization costs,
and load balancing—qualities found in many existing parallel data series indexes.
None of the conventional lock-free techniques we are aware of have been designed with
these properties in mind, and our experiments show that using traditional lock-free
techniques results in a significant performance drop (Graduate Thesis).
%
ReFreSh ensures that the underlying workload and data separation of the original data series
index are preserved, thus not compromising parallelism or load balancing. It provides a mechanism
for threads to determine whether a specific workload part has been processed, performing helping
only when necessary. ReFreSh operates in two modes:
\begin{itemize}
    \item Expeditive mode with minimal synchronization costs, which avoids any helping.
    \item Standard mode, where synchronization is necessary for helpin
\end{itemize}
%
A thread initially processes its workload in expeditive mode, and helping is only performed
after completing its assigned task. After that, threads synchronize to switch to standard mode.
This approach ensures that synchronization and communication costs remain low, similar to the
original index.
%
ReFreSh can be applied to any locality-aware algorithm (Section~\ref{sec:llf}) to create its
lock-free version. FreSh (Section~\ref{sec:fresh}) follows the design principles of locality-aware
iSAX-based indexes~\cite{isaxfamily,PFP21-I}, but requires replacing the original index’s data structures
with lock-free versions that support both expeditive and standard modes. We introduce lock-free
implementations of several concurrent data structures, including binary trees and priority queues
(Section~\ref{sec:fresh}), which provide enhanced parallelism compared to existing solutions.
%
We believe that these lock-free implementations and the ReFreSh framework can be applied to
achieve high-performance lock-free versions of various big data processing systems.
%
To enable the systematic application of ReFreSh across all processing stages of an iSAX-based index,
we introduce the abstraction of a traverse object (Section~\ref{sec:traverse}). The traverse object
is an abstract data type that offers a generic methodology for designing iSAX-based indexes modularly,
abstracting the key operations involved in the index’s function. This allows us to provide a
formal and flexible approach to designing iSAX-based indexes.
Specifically, the iSAX-based index can be implemented as a sequence of operations on traverse objects. 
%
The novelties of this work go beyond the new techniques we introduce to design the
lock-free index tree; in summary:
\begin{itemize}
\item Traverse-object abstraction: Provides a mechanism to apply the ReFreSh framework repeatedly to achieve lock-free data 
    series indexes..
\item ReFreSh framework: A generic approach that can be applied on top of any locality-aware blocking data structure to make it lock-free.
\item FreSh design: first lock-free data series index.
\item FreSh implementation: Introduces new lock-free tree and traverse-object implementations to achieve lock-freedom efficiently
\item The first lock-free data series index supporting dynamic batch insertions of updates.
\item Extends the FreSh implementation to support dynamic updates, addressing the challenge of maintaining lock-freedom during concurrent insertions..
\end{itemize}
The proposed lock-free tree incorporates several novel ideas that go beyond just engineering 
improvements. Previous approaches required copying and locally updating a leaf node L upon insertion,
followed by replacing it in the shared tree, which results in poor performance. We designed a novel
algorithm for updating leaves that employs concurrent counters, announcements, handshaking, and
other techniques to improve efficiency. The expeditive-standard mode transition also posed a
challenge, as it requires synchronization when switching between modes.


\noindent{\bf Contributions.} 
Our contributions are summarized as follows.

\noindent{$\bullet$  We develop a comprehensive theoretical framework for supporting lock-freedom 
systematically in highly-efficient data series indexes. This includes introducing key concepts
such as the traverse object, locality-awareness, and locality-aware lock-freedom, as well as
identifying the desirable principles and properties for achieving optimal performance.
This framework forms the basis for ReFreSh, a generic approach that can be applied to any 
locality-aware data series algorithm to ensure lock-freedom
} 

\noindent{$\bullet$ Based on ReFreSh, we develop FreSh, the first lock-free, efficient data series
index. We present new lock-free implementations of several data structures necessary to support the
required functionality, demonstrating the feasibility and performance of a lock-free approach.}

\noindent{$\bullet$ Building on the design of FreSh, we extend it to DFreSh, a lock-free data series
index that supports dynamic batch insertions of updates. This is the first solution to maintain
lock-freedom while allowing real-time updates, ensuring that queries always return consistent results,
including data from previous updates}

\noindent{$\bullet$ Our experiments, using large synthetic and real datasets, show that FreSh
performs as well as the state-of-the-art blocking index, without sacrificing performance for
lock-freedom. In many cases, it even outperforms existing solutions. Additionally, FreSh
significantly outperforms several lock-free baselines, which rely on conventional lock-free techniques,
due to its ability to preserve locality-awareness.}. 

\noindent{$\bullet$ We introduce a theoretical framework using the traverse object, enabling the
development of locality-aware data series indexes in a modular way. This framework allows for a
systematic application of ReFreSh to transform locality-aware blocking algorithms into efficient
lock-free versions.}

\noindent{$\bullet$ TODO: ADD DFreSh contributions.}

\chapter{Prelemenaries}
\label{chapter:prelemenaries}












% %BIB
\bibliographystyle{plain}
\cleardoublepage
\addcontentsline{toc}{chapter}{Bibliography}
\bibliography{bib/thesis}



\end{document}
