\documentclass[a4paper,11pt,twoside,openany]{book}

\usepackage{tipx}
\usepackage{algo}
% \usepackage[named]{libtex/algo}
\usepackage{algorithm}
\usepackage[noend]{algpseudocode}
\usepackage{amssymb}
\usepackage{amsthm}
\setcounter{tocdepth}{3}
\usepackage{graphicx}
\usepackage{amsfonts}
\usepackage{epsfig}
\usepackage{multirow}
\usepackage{amsfonts,amsmath}
\usepackage{srcltx}
\usepackage{multirow}
\usepackage{dsfont}
\usepackage{booktabs}
\usepackage[countmax]{subfloat}
\usepackage{subcaption}
\usepackage{color}
\usepackage[table]{xcolor}
\usepackage{tabularx}
% \usepackage{subfig}
\usepackage{url}
\usepackage{makeidx}
\usepackage{graphicx}
\usepackage{libtex/thesisStyle}
\usepackage{float}
\usepackage{enumitem} % To control list spacing





%%%%%%%%%%%%%
\input{libtex/procs}
\input{libtex/alias}
%%%%%%%%%%%%%


%%%%%%%%%%%%%%%%%Theorems%%%%%%%%%%%%%%%%%%%%%%%%%%%%
\newtheorem{definition}{Definition}
\newtheorem{theorem}{Theorem}[section]
\newtheorem{lemma}{Lemma}[section]

%%%%%%%%%%%%%%%%%%%%%%%Colors%%%%%%%%%%%%%%%%%%%%%%%%
\newcommand{\rb}[1]{{\color{red} #1}\normalcolor}
\newcommand{\bl}[1]{{\color{blue} #1}\normalcolor}
\newcommand{\tl}[1]{{\color{teal} #1}\normalcolor}
\newcommand{\br}[1]{{\color{brown} #1}\normalcolor}

\newcommand{\CAS}{\mbox{\textit{CAS}}}
\newcommand{\FAI}{\mbox{\textit{FAI}}}
\newcommand{\NULL}{\mbox{\sc null}}
\newcommand{\True}{\mbox{\texttt{True}}}
\newcommand{\False}{\mbox{\texttt{False}}}
\newcommand{\Boolean}{\mbox{boolean}}
\newcommand{\Int}{\mbox{int}}

\newcommand{\Process}{\mbox{\sc Process}}
\newcommand{\Insert}{\mbox{\sc Insert}}
\newcommand{\TreeInsert}{\mbox{\sc TreeInsert}}
\newcommand{\RPQHeapInsert}{\mbox{\sc RHeapInsert}}
\newcommand{\RPQHeapDeleteMin}{\mbox{\sc RHeapDeleteMin}}
\newcommand{\PQSAInsert}{\mbox{\sc InsertSA}}
\newcommand{\PQSADeleteMin}{\mbox{\sc DeleteMinSA}}
\newcommand{\SBInsert}{\mbox{\sc SBInsert}}
\newcommand{\MBInsert}{\mbox{\sc MBInsert}}
\newcommand{\Put}{\mbox{\sc Put}}
\newcommand{\Transform}{\mbox{\sc Transform}}
\newcommand{\Delete}{\mbox{\sc Delete}}
\newcommand{\Search}{\mbox{\sc Search}}
\newcommand{\Traverse}{\mbox{\sc Traverse}}
\newcommand{\Clean}{\mbox{\sc Clean}}
\newcommand{\SplitLeaf}{\mbox{\sc SplitLeaf}}
\newcommand{\DeleteMin}{\mbox{\sc DeleteMin}}
\newcommand{\GetMin}{\mbox{\sc GetMinHeap}}
\newcommand{\InitDeletePhaseHeap}{\mbox{\sc InitDeletePhaseHeap}}
\newcommand{\InitDeletePhaseSA}{\mbox{\sc InitDeletePhaseSA}}
\newcommand{\InitDeletePhase}{\mbox{\sc InitDeletePhase}}
\newcommand{\HeapDeleteMin}{\mbox{\sc DeleteMinHeap}}
\newcommand{\DeleteMinSA}{\mbox{\sc DeleteMinSA}}
\newcommand{\BufferCreation}{\mbox{\sc BufferCreation}}
\newcommand{\TreePopulation}{\mbox{\sc TreePopulation}}
\newcommand{\Prunning}{\mbox{\sc Prunning}}
\newcommand{\Refinement}{\mbox{\sc Refinement}}
\newcommand{\QueryAnswering}{\mbox{\sc QueryAnswering}}
\newcommand{\UpdateBSF}{\mbox{\sc UpdateBSF}}
\newcommand{\FindNode}{\mbox{\sc FindNode}}
\newcommand{\TotalNodes}{\mbox{\sc TotalNodes}}
\newcommand{\HelpTree}{\mbox{\sc HelpTree}}
\newcommand{\NextIndex}{\mbox{\sc NextIndex}}
\newcommand{\BC}{\mbox{$\mathit{BC}$}}
\newcommand{\PS}{\mbox{$\mathit{PS}$}}
\newcommand{\RS}{\mbox{$\mathit{RS}$}}
\newcommand{\RawData}{\mbox{$\mathit{RawData}$}}
\newcommand{\Refresh}{\mbox{$\mathit{ReFreSh}$}}
\newcommand{\Fresh}{\mbox{$\mathit{FreSh}$}}
\newcommand{\MESSI}{\mbox{$\mathit{MESSI}$}}
\newcommand{\MESSIenh}{\mbox{$\mathit{MESSIenh}$}}
%BASELINES
\newcommand{\DoAll}{\mbox{\sf DoAll}}
\newcommand{\FI}{\mbox{\sf FI-Based}}
\newcommand{\FINoSum}{\mbox{\sf FI-Based-NoSum}}
\newcommand{\DoAllSplit}{\mbox{\sf DoAll-Split}}
\newcommand{\DoAllSplitNoSum}{\mbox{\sf DoAll-Split-NoSum}}
\newcommand{\CASBased}{\mbox{\sf CAS-Based}}
%Evaluation
\newcommand{\FreshSub}{\mbox{Subtree}}
\newcommand{\FreshSTD}{\mbox{Standard}}
\newcommand{\FreshTreeCopy}{\mbox{TreeCopy}}
%CLEAN
\newcommand{\TPCLEAN}{\mbox{$\mathit{TP.CLEAN}$}}



% FOR WRITING GREEK %
\usepackage[greek,american]{babel}
\usepackage[libtex/iso-8859-7]{inputenc}

\newcommand{\selg}[0]{\selectlanguage{greek}}
\newcommand{\selam}[0]{\selectlanguage{american}}

%%%%%%%%%%%%%%%%%%%%%%%%%%
%\renewcommand{\baselinestretch}{1.5}
%\textheight=23cm \textwidth=16cm \voffset -1.5cm \hoffset -12.8mm
%%%%%%%%%%%%%%%%%%%%%%%%%%
\renewcommand{\baselinestretch}{1}\small\normalsize
\textwidth=390pt \oddsidemargin = 50pt \evensidemargin = 0pt
%%%%%%%%%%%%%%%%%%%%%%%%%%

\makeindex

\thispagestyle{empty}
\pagestyle{empty}

%%%%%%%%%%%%%%%%%%%%%%%%%%
\long\def\symbolfootnote[#1]#2
    {\begingroup
    \def\thefootnote{\fnsymbol{footnote}}\footnote[#1]{#2}
    \endgroup}

\newcommand{\workperformedat}{\symbolfootnote[0]{This work has been
    performed at the University of Crete, School of Sciences and
Engineering, Computer Science Department.}}

\newcommand{\worksupportedby}{\symbolfootnote[0]{The work has been
    supported by the Foundation for Research and Technology - Hellas (FORTH),
    Institute of Computer Science (ICS).}}

\newcommand{\thesisdate}{March 2025 }
%%%%%%%%%%%%%%%%%%%%%%%%%%


\overfullrule=5pt

\setlength{\baselineskip}{6pt}
\setlength{\itemsep}{0pt} % Reduce space between list items


\begin{document}
\begin{titlepage}
\begin{center}

\LARGE \textbf{DFreSh: A Lock-Free Index for Real-Time Data Series Processing}\\[0.5cm]
\LARGE \textit{Paterakis George}\\[0.5cm]

\vfill

\normalsize{
Thesis submitted in partial fulfillment of the requirements for the\\[0.30cm]

\textit{Masters' of Science degree in Computer Science and Engineering}}\\[0.30cm]

University of Crete\\
School of Sciences and Engineering\\
Computer Science Department\\
Voutes University Campus, 700 13 Heraklion, Crete, Greece\\[0.5cm]

\vfill

\Large{Thesis Advisor: Assistant Prof. \emph{Panagiota Fatourou}}\\[0.5cm]

\vfill

\end{center}

\workperformedat{} \worksupportedby{}
\end{titlepage}

\cleardoublepage





\thispagestyle{empty}

\newcommand{\thesistitle}{Lock-Free Indexes For Data Series Processing}
\newcommand{\owner}{Paterakis George}
\newcommand{\firstprof}{Panagiota Fatourou}
\newcommand{\secondprof}{Themis Palpanas}
\newcommand{\thirdprof}{Kostas Magoutis}
\newcommand{\csdchair}{Kostas Magoutis}

\begin{titlepage}

\begin{center}
\textsc{University of Crete}\\
\textsc{Computer Science Department}\\
\vspace{0.4cm}
\noindent {\textbf{\thesistitle{}}}\\
\vspace{0.4cm}
\noindent Thesis submitted by\\
\textbf{\owner{}}\\
in partial fulfillment of the requirements for the\\
Masters' of Science degree in Computer Science\\
\vspace{0.4cm} THESIS APPROVAL

\vspace{0.4cm}

\begin{tabular}{rl}
\\
Author: & \underline{\phantom{123456789012345678901234567890123456789012}}\\
    & \owner{}\\
    \\
    \\
    \\
Committee approvals: & \underline{\phantom{123456789012345678901234567890123456789012}}\\
    & \firstprof{}\\
    & {\small Assistant Professor, Thesis Supervisor}\\
    \vspace{0.2cm}
    \\
    \\
& \underline{\phantom{123456789012345678901234567890123456789012}}\\
    & \secondprof{}\\
    & {\small Professor, Committee Member}\\
    \vspace{0.2cm}
    \\
    \\
& \underline{\phantom{123456789012345678901234567890123456789012}}\\
    & \thirdprof{}\\
    & {\small  Professor, Committee Member}\\
    \vspace{0.2cm}
    \\
    \\
\hspace{1.4ex}Departmental approval: & \underline{\phantom{123456789012345678901234567890123456789012}}\\
    & \csdchair{}\\
    & {\small Professor, Director of Graduate Studies}\\
\end{tabular}
\\

\vfill Heraklion, \thesisdate{}
\end{center}

\thispagestyle{empty}

\end{titlepage}

\cleardoublepage








\thispagestyle{empty}
\begin{titlepage}
\begin{center}
\bc \Large{ \textbf{Lock-Free Indexes For Data Series Processing}} \ec \bc {\bf\large Abstract}\\\ec
\end{center}

In this thesis, we study lock-free data series indexes that achieve high performance
while maintaining robustness. First, we present FreSh, a lock-free data series index
based on Refresh, a generic approach we developed for efficiently supporting
lock-freedom on top of any locality-aware data series index.
We believe Refresh is of independent interest and can be used to derive 
high-performance lock-free versions of other locality-aware blocking data structures.
% 
To develop FreSh, we first conducted an in-depth study of the design decisions 
behind state-of-the-art data series indexes and the principles governing their 
performance. This led to a theoretical framework that enables the modular development 
and analysis of data series indexes. FreSh is designed for static data.
Experiments with several synthetic and real datasets illustrate that FreSh 
achieves performance comparable to that of the state-of-the-art blocking 
in-memory data series indexes. This demonstrates that the helping mechanisms of FreSh 
are lightweight and respect key principles that are crucial for performance 
in locality-aware data structures.
% \textsuperscript{*}
% 
A second contribution of this thesis is an extension of FreSh that works
in dynamic settings, which we call DFreSh. DFreSh handles batches of data that arrive
in the system dynamically, making it a robust solution for real-time applications while
retaining the performance benefits of FreSh. We follow two models of dynamicity coming
from two research fields and study the performance properties of each of these settings
using several synthetic and real datasets.




\vspace{1em}
\begin{center}
    \textsuperscript{*}\textit{FreSh has been published in the 42nd International \textbf{Symposium on Reliable Distributed Systems}, 2023.}
\end{center}
\vspace{1em}

\vfill
\end{titlepage}


\selam

\pagestyle{plain}
\pagenumbering{roman}
\setcounter{page}{1}
\setcounter{tocdepth}{3}

\tableofcontents

%\listoftables
%     \addtocontents{toc}{\contentsline {chapter}{List of Tables}{iii}}

% \listoffigures
%     \addtocontents{toc}{\contentsline {chapter}{List of Figures}{v}}

% \cleardoublepage

\pagenumbering{arabic}
%\renewcommand*\thesection{\arabic{section}}




 \pagestyle{headings}
%%%%%%%%%%%%%%%%%%%%%%% Introduction   %%%%%%%%%%%%%%%%%%%%%%%%%%%%%%%%%%%%%%%%%%%%%%%%%%%%%%%%%%
\chapter{Introduction}
\label{chapter:introduction}

Processing big collections of data series is of paramount importance 
for a wide spectrum of applications, across many domains, such as: operation health
monitoring in data centers, vehicles and manufacturing processes, internet of things data
analysis, environmental and climate monitoring, energy consumption analysis, decision taking
in financial markets, telecommunications traffic analysis, detection of medical and health problems,
improvement of web-search results, identification of pests invading agricultural crops,
etc.~\cite{DBLP:journals/sigmod/Palpanas15,DBLP:journals/dagstuhl-reports/BagnallCPZ19,Palpanas2019}.
In the heart of analyzing such collections 
lies the process of similarity search. 
Given a query series $Q$, {\em similarity search} 
returns a set of data series from the collection that have the closest 
distance to $Q$. Similarity search comes at considerable cost, 
due to very large size of data series collections and the
high dimensionality (i.e., length) of the data series that modern applications need to analyze. 
To address these challenges, current state-of-the-art data series 
indexes~\cite{DBLP:journals/pvldb/EchihabiZPB18,isax2plus,wang2013data,peng2018paris,parisplus,peng2020messi,PFP21-I,PFP21-II,hercules,dumpy}
are based on data series summarization. They develop a tree index containing
data series summaries used to prune the series collection in order to
restrict the execution of costly computations only to a small subset of it.

State-of-the-art data series indexes~\cite{DBLP:journals/pvldb/EchihabiZPB18,isaxfamily,peng2018paris,peng2020messi,PFP21-I,PFP21-II,hercules} 
exploit the parallelism supportedby modern multicore machines,  
but are largely lock-based to achieve synchronization. 
Using locks results in {\em blocking} implementations: if a thread that
holds a lock delays, other threads block, without making any progress,  
until the lock is released. Such thread delays  can degrade performance. 
Some applications (eg, operation health-monitoring in nuclear plants, or gravitational-wave
detection in astrophysics), are sensitive to delays, and would benefit from our approach.
Locks may also result in known problems, such as deadlock, priority inversion,
and lock convoying~\cite{F04}.

Lock-freedom~\cite{HS08} is a widely studied property when designing
concurrent trees~\cite{EFRB10,EFHR14,FR2018,ABF+22} and other data structures~\cite{F04,HS08,FKR18}. 
It avoids the use of locks, ensuring that the system, as a whole, 
makes progress, independently of delays (or failures) of threads. 
The relative performance of lock-free vs. lock-based algorithms depends on the setting:
when threads are pinned to distinct cores, algorithms using fine-grained locks do well. However, in
oversubscribed settings (more threads than cores), lock-based algorithms can suffer due to threads
getting de-scheduled while holding a lock. Lock-free algorithms address these issues, but may be
complicated and result in worse performance in settings with no delays/failures.
Designing lock-free data series indexes, which exhibit good performance, is
the focus of this thesis: we develop a lock-free data-series index, which always
produces the correct output, while maintaining the good performance of state-of-the-art
lock-based indexes.

\noindent{\bf Challenges.} 
To achieve lock-freedom, some form of helping is usually employed. 
That is, appropriate mechanisms are provided to make threads aware 
of the work that other threads perform, so that a thread may help others to
complete their work whenever needed. 
Unfortunately, conventional helping mechanisms are rather expensive and often
introduce high overheads~\cite{F04,HS08,7515610,Williams2012CCI}.
For this reason, the vast majority of the software stack is still based on locks. 
Ensuring lock-freedom while maintaining the good performance of 
existing data series indexes is a major challenge. 

State-of-the-art indexes are designed to (a) maintain some form of {\em data locality},
and (b) avoid synchronization as much as possible.
For instance, they often separate the data into {\em disjoint sets}, and have a
distinct thread manipulate the data of each set~\cite{parisplus,PFP21-I,PFP21-II}.
This processing pattern enables threads to work in parallel and independently from each other,
resulting in reduced synchronization and communication costs.  
These principles for reduced communication and synchronization are easily achieved when locks 
(or barriers) are utilized~\cite{peng2018paris,peng2020messi,PFP21-I,PFP21-II}.
However, the way helping works in conventional lock-free data structures is inherently
incompatible to these principles, thus making it challenging to implement helping on top
of such indexes without sacrificing them. Providing lock-freedom while maintaining load-balancing
among threads and ensuring good data pruning are further challenges to address. 

State-of-the-art data series indexes encompass several data processing phases, 
which often employ different data structures to accomplish their efficient processing. 
Coming up with lock-free versions of these data structures, while respecting
the communication and synchronization cost principles that govern existing 
indexes, is another major challenge to address.

In order to develop a generalized approach for supporting lock-freedom 
on top of data series indexes in a {\em systematic way}, we need to study
and understand the design decisions of state-of-the-art indexes and the performance
principles that govern them. Then, we need appropriate abstractions for the data series
processing stages and their properties, as well as a set of design principles that need
to be respected for efficiency. Accomplishing these goals leads to additional challenges. 

Existing indexes are designed primarily for static datasets, where all data is known in advance,
and updates are not a primary concern. However, many modern applications require the ability to
ingest new data continuously while ensuring that queries reflect the most recent updates.
This introduces significant challenges. For example, one of the most important challenges is
achieving efficient synchronization between threads that perform updates and threads that perform
queries, which now run concurrently. This synchronization should be achieved without sacrificing
the high performance of static indexes.


\noindent
{\bf Our approach.}
We introduce FreSh, a novel static lock-free data series index, and DFreSh, its dynamic
counterpart (i.e., a version of FreSh that handles dynamic data). These algorithms address
all the challenges mentioned above.
Our experimental analysis demonstrate that FreSh matches the performance of 
MESSI~\cite{PFP21-I}, the state-of-the-art concurrent data-series index, which is lock-based. 
This confirms the efficiency of the helping mechanism we employ in FreSh. In many cases, 
FreSh even outperforms MESSI by allowing for higher parallelism during the construction of the
tree index.
It's important to note that if threads crash, MESSI (and other lock-based approaches
~\cite{peng2018paris,PFP21-I,PFP21-II,hercules}) will not terminate, which is 
why we did not include MESSI in the experiments for this scenario. In contrast, 
FreSh always terminates successfully and correctly (even in the precessence of failures).

To achieve FreSh, we developed ReFreSh, a generic approach that can be applied to a
range of state-of-the-art blocking indexes to provide lock-freedom without 
incurring additional costs. 
%
ReFresh introduces the concept of {\em locality-aware lock-freedom} which encompasses
the properties of data locality, high parallelism, low synchronization cost,
and load balancing met in the designs of many existing parallel data series indexes.
None of the conventional lock-free techniques we are aware of has been designed with
the goal of respecting these properties. Indeed, our experiments show that such
conventional techniques result in significantly lower performance.

ReFresh respects the workload and data separation of the underlying data series index, 
in order to not hurt the degrees of parallelism and load balancing of the index.
Moreover, it provides a mechanism for threads to determine whether a specific part
of a workload has been processed, and help only whenever necessary.
Refresh introduces two modes of execution for each thread: 
(i) {\em expeditive} and (ii) {\em standard}.
%
A thread executing in {\em standard} mode may incur synchronization overhead,
as it needs to synchronize with helper threads; a thread executing in {expeditive}
mode executes a code that avoids synchronization altogether. 
A thread starts by processing its assigned workload in expeditive mode. 
Helping is performed only after a thread has finished processing its own workload.
Then, threads have to synchronize to execute on standard mode. 
This way, \Refresh\ maintains the synchronization and communication costs as low as
that of the underlying index.

Refresh can be applied (Chapter~\ref{chapter:Locality-aware}) on top of any locality-aware
algorithm to get a lock-free version of it. 
\Fresh\ (Chapter~\ref{chapter:FreSh}) follows the design decisions of locality-aware
iSAX-based indexes~\cite{isaxfamily,PFP21-I} (see Chapter~\ref{chapter:prelemenaries}).
However, to develop \Fresh, we had to replace all data structures 
of the original index~\cite{PFP21-I} with corresponding locality-aware lock-free versions;
we present lock-free implementations of several concurrent data structures, such as  
binary trees and priority queues (Chapter~\ref{chapter:FreSh}).
The proposed lock-free tree contains several new ideas. 
Previous solutions~\cite{EFRB10,DBLP:conf/podc/EllenFHR13,FR2018,NRM20} require that
when a key is inserted in a leaf $\mathit{\ell}$, 
$\mathit{\ell}$ is copied and updated locally, and then
replaced in the shared tree. This results in bad performance. 
Instead, we designed a novel algorithm for updating leaves, 
which provides enhanced parallelism compared to existing algorithms.
Another novelty, is the support of the new implementations for the expeditive and standard
execution modes, which was a challenge by itself, as synchronization is
needed to transfer from one mode to the other.
We believe that these implementations, as well as \Refresh, could be employed to get
highly-efficient lock-free versions of several other big data-processing solutions.

To be able to apply ReFresh in a systematic way throughout all processing stages of an
iSAX-based index, we introduce the abstraction of a {\em traverse object}
(Chapter~\ref{chapter:traverse-object}). The traverse object is an abstract data type that
leads to a generic methodology for designing an iSAX-based index in a modular way. It abstracts
the main processing pattern used during the operation of iSAX-based indexes.
Specifically, the iSAX-based index can be implemented via traverse object operations.
The introduction of the traverse object is one of the novelties of our work.

Many applications today require dynamicity, meaning that data arrives in a dynamic manner.
To meet the needs of such applications, we develop a dynamic version of FreSh called DFreSh,
where queries can be executed while data dynamically arrives in the index. We explore two dynamicity
models inspired by different research domains and analyze their performance characteristics across
various synthetic and real datasets.



\noindent{\bf Contributions.} 
Our contributions are summarized as follows.

\noindent{$\bullet$ We develop a theoretical framework 
for supporting lock-freedom in a systematic way on top of highly-efficient data series indexes.
In particular, we present \Refresh, a novel {\em generic approach} that can be applied on top of
any locality-aware data series algorithm to ensure lock-freedom.} 

\noindent{$\bullet$ Based on \Refresh, we develop \Fresh, the first {\em lock-free},
efficient iSAX-based data series index. To get \Fresh, we present new lock-free implementations
of several data structures which support the needed functionality.}

\noindent{$\bullet$ Our experiments, with large synthetic and real datasets, demonstrate 
that \Fresh\ performs as good as the state-of-the-art {\em blocking} index,
thus paying no penalty for providing lock-freedom (and in many cases achieves better performance). 
}

\noindent{$\bullet$ Experiments show that by providing lock-freedom without
jeopardizing locality-awareness, \Fresh\ outperforms by far several lock-free
baselines we have designed, based on conventional approaches for ensuring lock-freedom.
}

\noindent{$\bullet$ We present a theoretical framework, which introduces the traverse object,
and utilizes it to enable the development of locality-aware data series indexes in a modular way.
}

\noindent{$\bullet$ Building on FreSh and incorporating ideas from both the database and
concurrent computing domains, we designed and implemented two versions of DFreSh, each aligned
with each of these fields. DFreSh is the first lock-free data series index that supports dynamic
batch insertions while concurrently answering queries. It preserves the key innovations of FreSh
and maintains high performance similar to FreSh, despite its dynamic nature.
}

\section{Other Related Work}
Numerous tree-based techniques for efficient and scalable data series similarity search
have been proposed~\cite{DBLP:journals/pvldb/EchihabiZPB18, DBLP:journals/pvldb/EchihabiZPB19,
DBLP:conf/edbt/EchihabiZP21, DBLP:journals/pvldb/EchihabiPZ21},
including approximate~\cite{DBLP:journals/pvldb/AziziEP23, 
DBLP:journals/kais/LevchenkoKYAMPS21} and 
progressive~\cite{DBLP:conf/sigmod/GogolouTEBP20, DBLP:journals/tvcg/JoSF20,
DBLP:conf/sigmod/LiZAH20, DBLP:journals/vldb/EchihabiTGBP23} solutions.
Among these, iSAX-based indexes~\cite{isaxfamily} have proven to be particularly
competitive in terms of both index construction and query performance~\cite{DBLP:journals/pvldb/EchihabiZPB18,
DBLP:journals/pvldb/EchihabiZPB19, hercules, odyssey, dumpy}. These indexes also include
parallel and distributed solutions that leverage modern hardware (e.g., SIMD, multi-core,
multi-socket, GPU), such as ParIS+~\cite{parisplus}, MESSI~\cite{PFP21-I}, and SING~\cite{PFP21-II},
as well as distributed approaches like DPiSAX~\cite{dpisax, dpisaxjournal} and
Odyssey~\cite{odyssey}. A summary of how a conventional iSAX based index work is provided in 
Chapter~\ref{chapter:prelemenaries}.
 
The first lock-free concurrent search tree implementation was proposed in~\cite{EFRB10}.
Building on the ideas from that paper, we develop a baseline algorithm, which we discuss
and compare experimentally with \textit{FreSh} in Chapter~\ref{chapter:Evaluation}.
Several other non-blocking concurrent search trees have been introduced in the 
literature~\cite{BER14, HL16, ABF20, HJ12, NRM20, CNT14, BP12, EFHR14, FR2018, ABF+22}.
The key novelty of our tree implementation, presented in chapter~\ref{chapter:FreSh}, is its
ability to concurrently perform multiple insert operations in a lock-free manner to
update the array in a (fat) leaf. Additionally, it supports the expeditive-standard mode
of execution. These advancements result in improved parallelism and better performance.
Our approach focuses solely on the functionality required for implementing traversal objects,
whereas the aforementioned implementations support a variety of other features or are designed
for different contexts.
 
Concurrent priority queues have been explored in~\cite{AK15-I, RT21, WG15, SUNDELL2005609,
tamir_et_al, LJ13}, none of which are based on sorted arrays or support different
execution modes. In our baseline lock-free implementations, we use a skip-list-based
priority queue~\cite{LJ13}, which has been shown to perform well. However, our experiments
indicate that the priority queue design we implemented for \textit{FreSh} significantly
outperforms this approach (Chapter~\ref{chapter:Evaluation}).
 
Universal constructions~\cite{FK11spaa, FK12ppopp, FK14, FK17opodis, FK09, FK20,
FKK18, EF+16, FKK22} can provide wait-free or non-blocking concurrent versions of any
sequential data structure. However, due to their general nature, they are often less
efficient than implementations tailored to specific data structures. The algorithms
in~\cite{FK11spaa, FK14, FKK22} are highly efficient for small shared objects
(e.g., stacks and queues) but are not suitable for our application.
 
The concept of transforming algorithms to achieve different progress guarantees is not
new, as seen in~\cite{SP14, ELM05, GKK06}, though these transformations address different
problems. The technique used in \textit{FreSh}, called \textit{ReFreSh}  departs from all these
prior approaches.

The use of timestamps for handling dynamic data is common in data-based systems.
For example, Apache Kafka and Google Spanner employ such mechanisms.
The first approach we use for dynamicity is inspired by this timestamp model ~\cite{babcock2002}.
Additionally, we introduce a second approach to designing a dynamic index that is linearizable \cite{herlihyWingLinearizability},
 a widely used correctness condition in concurrent data structures.
%%%%%%%%%%%%%%%%%%%%%%% PRELIMINARIES %%%%%%%%%%%%%%%%%%%%%%%%%%%%%%%%%%%%%%%%%%%%%%%%%%%%%%%%%%
\chapter{Preliminaries}
\label{chapter:prelemenaries}

\section{Data Series And Similarity Search}

A \textbf{data series (DS)} of size (or dimensionality) \( n \) is a sequence 
of \( n \) pairs, where each pair consists of a real value and its corresponding 
dimension.
% 
The \textbf{Piecewise Aggregate Approximation (PAA)}~\cite{DBLP:journals/kais/KeoghCPM01} 
provides a compact representation of a data series by dividing the x-axis into 
\( w \) equal segments. Each segment is then represented by the \textbf{mean value} 
of the corresponding points, as shown by the black horizontal lines in 
Figure~\ref{fig:PAA}. 
% 
\bl{
The PAA representation of a data series \( T = (t_1, t_2, \dots, t_n) \) is computed by 
dividing it into \( w \) segments, each containing \( \frac{n}{w} \) data points. The 
\( i \)-th PAA coefficient \( \overline{t}_i \) is given by:  

\begin{equation}
\overline{t}_i = \frac{w}{n} \sum_{j=\frac{n}{w} (i-1) + 1}^{\frac{n}{w} i} t_j
\label{eq:paa-transformation}
\end{equation}
% 
where the summation aggregates all values in the segment, and the factor \( \frac{w}{n} \) 
ensures proper scaling.  
}
% 
To compute the \textbf{iSAX summary}~\cite{shieh2008sax}, the y-axis is partitioned 
into \( c \) discrete regions, where \( c \) represents the \textbf{cardinality} 
of the representation.
%
Each region is assigned a unique bit pattern, and instead of storing the raw PAA values, 
iSAX encodes each segment using the bit representation of the region it falls into. 
This forms a \textbf{symbolic representation} of the series, 
such as the word \( 10_2 00_2 11_2 \) in Figure~\ref{fig:iSAXSummary} 
(where subscripts indicate the number of bits used per segment).
% 
The number of bits used per region can vary, enabling the construction of a 
\textbf{hierarchical tree index}, known as an \textbf{iSAX-based tree index}, as 
illustrated in Figure~\ref{fig:iSAXTree}.
The index is a \textbf{leaf-oriented tree}, where each leaf stores up to \( M \) keys.
% 
During insertion, if the target leaf \( \ell \) has available space, the new key 
is simply added. However, if \( \ell \) is full, it undergoes a \textbf{split operation},
where it is replaced by a small subtree consisting of an \textbf{internal node} and two 
new leaves, which inherit  the keys from \( \ell \). If one of the newly created leaves
remains empty, the splitting process continues recursively.
For further details on iSAX-based indexes, refer to~\cite{isaxfamily}.


\begin{figure}[htbp]
    \centering
    \begin{subfigure}[b]{0.35\textwidth}
        \centering
        \includegraphics[width=\textwidth]{figures/prelem/timeseries.png}
        \caption{Data Series}
        \label{fig:dataseries}
    \end{subfigure}
    \hfill
    \begin{subfigure}[b]{0.35\textwidth}
        \centering
        \includegraphics[width=\textwidth]{figures/prelem/PAA.png}
        \caption{PAA Summary}
        \label{fig:PAA}
    \end{subfigure}
    \hfill
    \begin{subfigure}[b]{0.45\textwidth}
        \centering
        \includegraphics[width=\textwidth]{figures/prelem/isax.png}
        \caption{iSAX Summary}
        \label{fig:iSAXSummary}
    \end{subfigure}
    \hfill
    \begin{subfigure}[b]{0.8\textwidth}
        \centering
        \includegraphics[width=\textwidth]{figures/prelem/isaxTreeCustom.png}
        \caption{iSAX Tree}
        \label{fig:iSAXTree}
    \end{subfigure}
    
    \caption{From data series to iSAX index}
    \label{fig:from_ds_to_iSAX}
\end{figure}

\noindent{\bf Similarity Search}  
We focus on \textit{exact similarity search} (also known as exact \textit{1-NN}),  
which retrieves the data series from a collection that is most similar to a given 
query series. Similarity is typically measured using \textbf{Euclidean Distance (ED)},
but our techniques are general enough to support other widely used 
\textit{similarity measures}, such as Dynamic Time Warping (DTW)~\cite{rakthanmanon2012searching}.  
% 
The \textbf{Euclidean distance} between two time series  
\( T = \{t_1, ... , t_n\} \) and \( T' = \{t'_1, ... , t'_n\} \)  
is defined as:  
\begin{equation}
ED(T, T') = \sqrt{\sum_{i=1}^{n} (t_i - t'_i)^2}
\label{eq:Euclidean-distance}
\end{equation} 
% 
We refer to the distance between the \textit{iSAX summaries} of two data series  
as the \textbf{lower-bound distance}. \bl{If we transform the original subsequences into PAA
representations, $\overline{T}$ and $\overline{T'}$, using Eq.~\ref{eq:paa-transformation}
(according to~\cite{Lin2007SAX}),
we can then obtain a lower bounding approximation of the Euclidean distance
between the original subsequences by:

\begin{equation}
DR(\overline{T}, \overline{T'}) \equiv \sqrt{\frac{n}{w}} \sqrt{\sum_{i=1}^{w} (\overline{t}_{i} - \overline{t'}_{i})^2}
\label{eq:lower-bound-distance}
\end{equation}
}
% 
The calculation of this distance guarantees the \textbf{pruning property}:  
the lower-bound distance between two data series is always less than or equal to  
their Euclidean distance, which we refer to as the \textbf{real distance}.  
% 
This property enables efficient pruning of candidates during query processing:  
a data series can be \textbf{pruned} if its lower-bound distance to the query series
\( Q \) is greater than the real distance of any other data series in the collection 
from \( Q \).

\noindent
{\bf Leaf-Oriented Trees.}  
In a \textit{leaf-oriented tree}, all data are stored in the leaves, with each leaf
capable of holding up to \( M \) keys.  
% 
During an insertion, if the target leaf \( \ell \) has available space,  
the new key is simply added to \( \ell \).  
However, if \( \ell \) is full, it undergoes a \textbf{split}: it is replaced 
by a subtree consisting of an internal node and two new leaves.  
The keys from \( \ell \) are then redistributed between the new leaves based on
their values.  
% 
If one of the newly created leaves remains empty after redistribution,  
the splitting process is repeated until both leaves contain keys.  

\section{iSAX-Based Indexing}

Concurrent iSAX-based indexes~\cite{peng2018paris,parisplus,peng2020messi,  
PFP21-I,PFP21-II} operate in two main phases:  
the \textit{tree index construction phase} and the \textit{query answering phase},  
each utilizing distinct data structures. 

\noindent\textbf{Tree Index Construction.} During this phase a set of \textit{worker threads}  
processes a collection of input data series (i.e., \textit{raw data}).  
Each series is summarized using an iSAX representation and inserted into a 
\textit{tree index} as a pair of an iSAX summary and a pointer to the corresponding 
data series.  
% 
The process is divided into two main stages:
\begin{itemize} [noitemsep, topsep=3pt, partopsep=0pt, parsep=0pt]
    \item \textbf{Buffers Creation:} iSAX summary pairs are first stored in array-based  
    \textit{summarization buffers}.  
    \item \textbf{Tree Population:} Worker threads traverse these buffers and insert 
    their entries into the tree index.  
\end{itemize}  
Data series with similar iSAX representations are placed in the same buffer and later  
within the same subtree of the index tree.  
This approach ensures \textbf{high parallelism}, \textbf{good locality}, and  
\textbf{low synchronization overhead} during index construction.

\noindent\textbf{Query Answering.}  
Given a query data series \( Q \), the system follows these steps
to find the data series with the smallest distance to \( Q \) :  
\vspace{-5pt} % Reduce space before the enumerate list
\begin{enumerate}[noitemsep, topsep=1pt, partopsep=0pt, parsep=0pt]
    \item The iSAX summary of \( Q \) is computed and used to traverse the index tree,  
    leading to a leaf \( \ell \).  
    \item The \textit{real distance} between \( Q \) and each data series in \( \ell \) is computed.  
    \item The smallest distance found so far is stored in the \textbf{Best-So-Far (BSF)} variable,  
    serving as an initial approximate answer.  
\end{enumerate}  
% 
Then query answering proceeds by exexuting the following two stages
namely the \textit{Prunning Stage} and the \textit{Refinement Stage}:

\noindent\textbf{Pruning Stage:} In this stage, a set of \textit{query answering threads}
traverse the tree. If the \textit{lower bound distance} from a node to $Q$
exceeds the Best So Far (BSF) value, the node is \textit{pruned} and excluded from
further processing. This pruning ensures that no pruned data series can contribute
to the final answer.

\noindent\textbf{Refinement Stage:} During refinement, the non pruned data series 
are the candidate series and are stored in one or more \textit{priority queues}~\cite{parisplus,PFP21-I,PFP21-II}.
Multiple threads process these candidate data series, calculating their exact
distances to Q. The BSF is updated whenever a smaller distance is encountered.

At the end of this phase, the final answer is contained in BSF.  
\textit{Barriers} synchronize threads at the end of each stage, ensuring correctness,  
while \textit{locks} handle concurrent access to shared data structures.  
Figures~\ref{fig:example} and~\ref{fig:example2} summarize the indexing process.  

\begin{figure}[H]
    \centering
    \includegraphics[width=0.85\textwidth]{figures/prelem/iSAX-index.pdf}
    \caption{Similarity search with the use of a data series index.}
    \label{fig:example}
    \vspace{-0.5cm} % Reduces space after the figure
\end{figure}

\begin{figure}[H]
    \centering
    \includegraphics[width=0.8\textwidth]{figures/prelem/flowchart2.pdf}
    \caption{Index building and query answering flowchart for the MESSI data series index.}
    \label{fig:example2}
    \vspace{-0.5cm} % Reduces space after the figure
\end{figure}


\noindent{\bf MESSI as an example.} 
MESSI~\cite{peng2020messi} is a state-of-the-art, in-memory iSAX-based index.
It uses an array, referred to as \textit{RawData}, to store the raw data. During the 
{\em buffers creation stage}, this array is split into a number of fixed-size chunks 
containing consecutive raw data series. A Worker thread repeatedly {\em acquires} and
processes a chunk, storing the calculated iSAX summaries in the appropriate 
summarization buffers. Threads determine which chunks to work on using a \FAI\ object.
Each thread is allocated its own space in each summary buffer to avoid collisions when
adding elements, ensuring thread safety. This process continues until all data series in
\textit{RawData} have been processed.
% 
During the {\em tree population stage}, worker threads again use \FAI\ to {\em acquire}
iSAX buffers to work on. Each subtree of the index tree is a binary leaf-oriented tree
with fat leaves.
% 
In the {\em query answering phase}, a query answering worker repeatedly {\em acquires}
a subtree (using \FAI) and {\em traverses} it by calculating the {\em lower-bound
distance} between the query series \( Q \) and the iSAX summary of each node
encountered. If the lower-bound distance of a leaf is smaller than the Best So Far (BSF)
distance, the leaf is inserted into a {\em priority queue}, with the distance
used as the priority. The algorithm uses a set of {\em priority queues} and threads 
insert elements into these queues in a round-robin fashion.
Since a priority queue may be concurrently accessed by multiple threads, 
MESSI employs a coarse-grain {\em lock} for each queue for synchronization.
% 
During the {\em refinement phase}, each query answering thread \( t \) is assigned a 
priority queue \( \mathit{PQ} \) to process. It repeatedly removes the leaf with the
minimum priority from \( \mathit{PQ} \) and compares its iSAX summary to the BSF.
If the leaf's summary is smaller, the real distances between the data series stored
in the leaf and the query series are computed. Otherwise, the leaf and all remaining
nodes in the queue are pruned. Since multiple threads may process a priority queue
concurrently, it is protected by a coarse-grain lock. Once processing of a priority
queue is complete, thread \( t \) moves on to the next priority queue in a round-robin
manner.
% 
{\em Barriers} are used among threads at the end of each stage and before the start of 
the next to ensure correct synchronization and maintain the integrity of the process.

\section{System And Dynamic Data}
\noindent{\bf System:}
We consider a shared-memory system with \( N \) threads that execute {\em concurrently
and asynchronously} while communicating by accessing shared objects. A shared object
\( O \) can be atomically read or written. Additionally, the operation \FAI(O, v)
atomically reads the current value of \( O \), adds the value \( v \) to it, and
returns the value that was read. The operation \CAS($O, u, v$) reads the value of
\( O \), and if it is equal to \( u \), it changes it to \( v \) and returns
\textit{True}; otherwise, \( O \) remains unchanged and \textit{False} is returned.
% 
Threads may experience delays (e.g., due to page faults, power consumption issues,
or overheating~\cite{inteloverheating}), or they may fail by crashing (e.g., due to
software errors). An algorithm is said to be {\em blocking} if a thread must wait for
actions to be performed by other threads in order to make progress. {\em Lock-freedom}
guarantees that some thread makes progress at each point in time, thus 
the system as a whole continues to make progress independently of the speed of threads
or their failures. Lock-Free algorithms do not utilize locks, MESSI is not lock-free;
it is blocking.


 
\noindent{\bf Dynamic Data:} To simulate dynamicity in data we assume that all data is initially stored in memory, but
only a small subset is used to create the initial index. The rest of the data are organized into batches.
After the initial index is built, new data arrives dynamically in batches, with the batch size
configurable by the user. To simulate real-time data ingestion, we introduce a delay interval
between consecutive batches. This delay, which can be defined by the user,
begins as soon as a batch arrives. By appropriately tuning with the delay we simulate different
dynamicity patterns in data. 
% 
The behavior of the system depends on the interplay between batch size and delay interval.
If the delay is determined to be long enough, index workers (i.e., threads responsible for building the index)
may fully integrate each batch into the index before the next batch arrives, remaining idle for
the rest of the time. Conversely, if the delay is determined to be short or the batch size is large, new batches
may arrive before previous ones have been fully processed.
% 
While index workers are responsible for inserting new data, query workers (i.e., threads
responsible for answering queries) simultaneously access this data to perform
exact similarity search. Ensuring correctness in such a system requires a robust mechanism to
manage concurrent updates and queries. To achieve this, we employ two models inspired by different
domains of computer science: a \textit{timestamp-based model} and a \textit{consistency model based
on linearizability}.  

\subsection{Timestamp-Based Ordering And Linearizability}  

A well-known approach for handling dynamic data is the use of timestamps to enforce ordering 
constraints. This method is widely used in database systems to ensure that data is processed
in a meaningful manner~\cite{babcock2002}. Notable systems employing timestamp-based mechanisms
include Apache Kafka and Google Spanner~\cite{kafka2021,spanner2013}.  
% 
In our approach, timestamps are implicitly generated using the system's clock. Specifically,
every batch is assigned a timestamp corresponding to the time when the processing of the batch
starts. This timestamp plays a crucial role in  how queries behave. Specifficaly,   

\begin{itemize} [noitemsep, topsep=3pt, partopsep=0pt, parsep=0pt]
    \item Queries must see all data that previous queries have seen.  
    \item If a query has a later timestamp than an ongoing batch, it must wait for that batch to
     finish processing before continuing.  
\end{itemize}  
% 
By enforcing this ordering, we ensure that queries always operate on a consistent view of the data.
Our approach is inspired by the principles outlined in \textit{Models and Issues in Data Stream
Systems}~\cite{babcock2002}, which guided the design and implementation of timestamps in our dynamic system.  

The second approach we employ is inspired by \textit{linearizability}
\cite{herlihyWingLinearizability}, a key concept in concurrent
and distributed systems. Linearizability provides a strong consistency guarantee by ensuring
that all operations appear to execute atomically in a single, consistent order.  

Formally, a system execution is \textbf{linearizable} if:  
\begin{itemize}  [noitemsep, topsep=3pt, partopsep=0pt, parsep=0pt]
    \item For each completed operation p, to insert a
    linearization point *p somewhere between p's
    invocation and response in a.
    \item To select a subset F of the incomplete operations,
    and for each operation r in F:
    to select a response, and
    to insert a linearization point *r somewhere after r's invocation in a.
    \item These operations and responses should be selected and these 
    linearization points should be inserted, so
    that, in the sequential execution constructed by
    serially executing each operation at the point that
    its linearization point has been inserted, the
    response of each operation is the same as that in a
\end{itemize} 
% 
By leveraging the principles of linearizability, we ensure that concurrent updates and queries
interact correctly, preserving system correctness while maintaining high concurrency. 
% 
The Timestamp and Linearizability models complement each other by addressing different
aspects of correctness in a dynamic, concurrent system, reflecting the principles of two
distinct domains: the database world and the concurrent computing world.
The timestamp model requires queries to wait for ongoing batches to finish before starting
their processing. This ensures that any batch arriving before or concurrently with the query
will be considered. This approach is generally preferred in the field of databases, as it
respects the temporal sequence of batch arrivals and processing.
On the other hand, linearizability offers more concurrency, as there is no waiting time.
While data arrives, the algorithm is responsible for handling it, ensuring that the execution
remains linearizable. This allows for concurrent operation throughout the entire process of
insertions. 

%%%%%%%%%%%%%%%%%%%%%%% Traverse Object %%%%%%%%%%%%%%%%%%%%%%%%%%%%%%%%%%%%%%%%%%%%%%%%%%%%%%%%
\chapter{Traverse Object}
\label{chapter:traverse-object}
\vspace{-0.5cm}

\section{Traverse Object}
In this section, we introduce the \textbf{traverse object}, an abstract data type
that enables a modular design for iSAX-based indexes. The processing pipeline of
an iSAX-based index consists of multiple stages, where each stage processes data
produced by the previous one. The first stage handles the original dataset, while
subsequent stages refine and structure the data progressively. This structured
processing inspired the definition of the \textbf{traverse object}, whose
sequential specification is given below.
% 
\begin{definition}
\label{def:traverse}
Let $U$ be a universe of elements. A \textbf{traverse object} $S$ stores elements
of $U$ (which may not be distinct) and supports the following operations:

\begin{itemize}
    \item \textit{Put}($S,e,\mathit{param}$): Adds an element $e \in U$ to $S$.
    The optional parameter $\mathit{param}$ allows implementations to pass
    additional arguments.
    
    \item \textit{Traverse}($S,f,\mathit{fparam},del$): Traverses $S$, applying
    the function $f$ with parameters $\mathit{fparam}$ to each element. If the
    $del$ flag is set, the elements are deleted from $S$ after being processed.
    The parameter $\mathit{fparam}$ serves the same purpose as in \textit{Put}.

    \item \textit{Clean}($S$): Resets $S$ to be the empty traverse
    object.
\end{itemize}
% 
A traverse object $S$ satisfies the \textbf{traversing property}:  
Each instance of \textit{Traverse} in a sequential execution applies $f$ at
least once to all distinct elements added to $S$ that have not been deleted
in a previous invocation of \textit{Traverse}.
\end{definition}

\subsection{Traverse Objects in iSAX-Based Indexing}

\begin{algorithm}[htbp]
    \footnotesize
    \vspace*{2mm}
    
    \begin{algorithmic}[1]
    
    \Procedure{InitializeSharedObjects}{}
        \State \BC $\gets$ initially contains all raw data series
        \State \TP, \PS, \RS $\gets$ initially empty
        \State $\mathit{BSF} \gets$ some initial value
    \EndProcedure
    
    \vspace*{1mm}
    \State{Code for thread $t_i$, $i \in \{0, \ldots, n-1\}$:}		
    \vspace*{1mm}

    \Procedure{\normalfont QueryAnswering(QuerySeriesSet $\mathit{SQ}$) \textbf{returns} int}{}

        \State \BC.\Traverse(\&BufferCreation(), $\mathit{BCParam}$, \False)
        \State \TP.\Traverse(\&TreePopulation(), $\mathit{TPParam}$, \False)
        \State \PS.\Traverse(\&Prunning(), $\mathit{PSParam}$, \False)
        \State \RS.\Traverse(\&Refinement(), $\mathit{RSParam}$, \True)
        \State \Return $\mathit{BSF}$
    \EndProcedure
    
    \vspace*{1mm}
    \Procedure{\BufferCreation(DataSeries $\mathit{ds}$)}{}
        \State iSAXSummary $\mathit{iSAX} \gets$ Calculate the iSAX summary for $\mathit{ds}$
        \State Index $\mathit{bind} \gets$ index to appropriate buffer based on $\mathit{iSAX}$
        \State \TP.\Put($\langle \mathit{iSAX}$, index of $\mathit{ds}$ $\rangle$, $\mathit{bind}$)
    \EndProcedure
    
    \vspace*{1mm}
    \Procedure{\TreePopulation(Summary $\mathit{iSAX}$, Index $\mathit{ind}$,
         Index $\mathit{bind}$, Boolean $\mathit{flag}$)}{}
        \State \PS.\Put($\langle \mathit{iSAX}$, $\mathit{ind} \rangle$, $\mathit{bind}$, $\mathit{flag}$)
    \EndProcedure
    
    \vspace*{1mm}
    \Procedure{\Prunning(DataSeries $Q$, DataSeriesSet $\mathit{E}$,  
     \Statex \normalfont Boolean $\mathit{flag}$) \textbf{returns} boolean}{}
            \State iSAXSummary $\mathit{iSAX} \gets$ Calculate the iSAX summary for $\mathit{E}$
            \State int $\mathit{lbDist} \gets$ lower bound distance between $\mathit{iSAX}$ and $Q$
            \If{$\mathit{lbDist} < \mathit{BSF}$}
                \State \RS.\Put($\langle \mathit{E, iSAX} \rangle$, $\mathit{flag}$)
                \Return \True
            \EndIf
            \Return \False
    \EndProcedure
    
    \vspace*{1mm}
    \Procedure{\Refinement(DataSeries $Q$, DataSeriesSet $E$, Summary $\mathit{iSAX}$,
     \normalfont Function *\UpdateBSF) \textbf{returns} Boolean}{}
        \State int $\mathit{lbDist}$, $\mathit{rDist}$
        \State $\mathit{lbDist} \gets$ lower bound distance between $\mathit{iSAX}$ and $Q$
        \If{$\mathit{lbDist} < \mathit{BSF}$}
            \ForAll{pair $\langle \mathit{iSAX_{ds}, ind_{ds}} \rangle$ in $E$}
                \State $\mathit{lbDist} \gets$ lower bound distance between $\mathit{iSAX_{ds}}$ and $Q$
                \If{$\mathit{lbDist} < \mathit{BSF}$}
                    \State $\mathit{rDist} \gets$ real distance between $\mathit{ds}$ and $Q$
                    \If{$\mathit{rDist} < \mathit{BSF}$}
                        \State *\UpdateBSF($\mathit{BSF},\mathit{rDist}$) \Comment{user-provided routine}
                    \EndIf
                \EndIf
            \EndFor
            \Return \True
        \Else
            \Return \False
        \EndIf
    \EndProcedure
    
    \end{algorithmic}
    
    \caption{Implementation of an iSAX-based index using the traverse objects \BC, \TP, \PS, \RS.}
    \label{alg:iSAXTraverse}
    \end{algorithm}
    
    

We use four instances of a traverse object to implement the four stages of an iSAX-based
index. We call BC, TP, PS, and RS, the traverse objects that implement the buffer 
creation,tree population, prunning, and refinement stages, respectively.
% 
The buffers creation stage uses an array RawData to store the raw data series, thus, the
elements of BC are stored in RawData. The tree population stage uses a set of arrays
(summarization buffers) where the pairs of iSAX summaries and pointers to data series
are initially stored. TP stores these pairs. The prunning stage employs a leaf-oriented
tree to store these pairs. Thus, PS organizes the pairs into as many sets as the leaf
nodes of the tree. Each set contains the pairs stored in each leaf. Finally, the
refinement stage uses priority queues to store those tree leaves containing candidate
series.
% 
Answering a query is now comprised of a sequence of four invocations of TRAVERSE on the
different traverse objects. Algorithm 1 provides pseudocode for the implementation of
an iSAX-based index using traverse objects.
 
\noindent{\bf Static case.}
A static data series index where no data arrive in the system after the creation of the
index is comprised of four stages which do not overlap with one another. This is
usually ensured with the use of synchronization barriers. In the scheme of Algorithm 1,
the barriers, if needed, (as well as multithreading processing) should be incorporated
in the implementation of PUT and TRAVERSE. Thus, a static iSAX-based
index satisfies the following property:
% 
\begin{definition}[\textbf{Non-Overlapping Property}]
    \label{def:non-overlapping}
    In every concurrent execution of the index, for every traverse object $S$, an
    instance of \textit{Traverse} on $S$ is performed only after all \textit{Put}
    operations that add distinct elements to $S$ have completed.
\end{definition}
% 
Assume that the non-overlapping property holds for BC, TP, PS, and RS and that RawData
initially stores all raw data series. The traversing property implies that the
\textit{\BufferCreation} function is invoked at least once for each data series ds in RawData, so
at least one appropriate pair is added for it in TP, i.e., the summarization buffers
are populated appropriately. By the non-overlapping and the traversing properties,
\textit{\BufferCreation} is invoked for all these pairs. Since \textit{\BufferCreation}
invokes \textit{\Put} on PS, it follows that at least one pair for each of the data series of the collection
is added in PS (i.e. in the tree index). By the traversing property, all elements of PS
are traversed and \textit{\Prunning} is called on them. Thus, all tree leaves that cannot be
pruned are added in RS. Note that \textit{\Traverse} on RS is invoked with the del flag being True.
This allows to use (one or more) priority queues for implementing RS, and to employ
\textit{\DeleteMin} to delete each traversed element during \textit{\Traverse}. \textit{\Refinement}
will be applied on every traversed element of RS. Therefore, those leaves that cannot be pruned
will be further processed by calculating real distances and for the data series they store,
and by updating BSF whenever needed. Implementations for \textit{\Put} and \textit{\Traverse} for BC, TP,
PS, and RS in FreSh are presented in Chapter~\ref{chapter:FreSh}.


\noindent{\bf Dynamic case.}    
In a dynamic environment, data arrives in batches, each with a unique ID. For every
batch, the same sequence of events as in an iSAX-based index is executed.
During the Index Construction phase, workers compute summaries for each data series and
place them into the appropriate summarization buffers. Each summarization buffer is a 
two-dimensional array, where the number of rows corresponds to the number of index workers,
ensuring that each worker has its own dedicated part in the summarization buffer.
% 
Once all summaries for a batch have been generated, workers begin inserting, into the iSAX 
tree, pairs of summaries and pointers to the positions of the corresponding data series.
% 
Since this process repeats for every batch, both the iSAX tree and the
summarization buffers must accommodate continuous growth. However, while the iSAX tree
must retain all data, the summarization buffers only store temporary information that
is not needed anymore after a batch is processed. To avoid unnecessary memory growth,
we do not expand the summarization buffers with each new batch that arrives.
Instead, we clear and reuse them for each batch, ensuring efficient memory utilization. 
% Since the sequence of events
% is implemented by the traverse objects \BC\ and \TP\ they must be reinitialized
% before processing a new batch.
\newline \noindent{\bf Reusing The Summarization Buffers.} 
During \TP, workers traverse the summarization buffers to build the iSAX tree.
Specifically, each worker acquires a summarization buffer and proceeds with
constructing a subtree of the iSAX tree. Since summarization buffers are reused
across batches, a worker must determine whether the data stored by other workers
is still valid or belongs to a previous batch that has not yet been cleared.
To distinguish which batch the data in a buffer belongs to at any given time, we
associate with each worker's part within a summarization buffer with a
sequence number, which stores the id of the batch.
The reinitialization of summarization buffers is performed by invoking the 
\textit{\TPCLEAN}.  
During this process, for each summarization buffer, each worker is responsible for:
\begin{enumerate} [noitemsep, topsep=3pt, partopsep=0pt, parsep=0pt]
    \item Resetting the size of his part to zero and.
    \item Updating the sequence number of his part to match the ID of the next batch.
\end{enumerate}

Our system is designed to be lock-free. Because of this it is ensured that when an index worker
reaches the \TP\ stage, it indicates that the \BC\ stage has been completed successfully and
that all data series for the current batch have been stored in the summarization buffers, even
if some workers are slow or have experienced failures.
% 
During the execution of TP, each worker first checks the size of its part.
If the size is zero indicating that no data has been inserted into the currently processing part,
the worker moves on to the next part in this buffer,
If the size is nonzero, meaning there is data in that part, the worker proceeds by
reading the sequence number. If the sequence number differs from the current batch ID,
it indicates that the data is obsolete.

For instance a worker with ID $1$ may read a nonzero size in the buffer part of a worker
that contains data from the batch $X$. It may then become slow. Meanwhile, the worker 
responsible for that part updates the size to zero and sets the sequence number to
the current batch ID. To avoid situations where worker $1$ wakes up, reads the sequence number
as the current batch ID and starts processing a batch with actual size zero, workers verify that
the values they have read are still the same. This is done by rereading them and checking whether
their values have changed. If this is the case, the worker skips the work it was supposed to do.

% 
In a dynamic setting, threads may concurrently execute two distinct phases of the index:
\textit{Index Construction} and \textit{Query Answering}. This overlap introduces challenges
in programming such a system, as batch insertions occur simultaneously with query processing. 
Because of this concurrent execution, the Non-Overlapping
Property~\ref{def:non-overlapping}, which holds in the static case, does not apply
in the dynamic setting. However, this property still applies separately to the two
phases, Index Construction and Query Answering, but not to the system as a whole.

In Algorithm~\ref{alg:DynamiciTraverseObject} we present 
\textit{Index Construction} and \textit{Query Answering} as two separate procedures that can
be executed concurrently by threads. 
\newline \noindent{\bf IndexCreation. }Each thread has a local counter called
called $currentBatch$ which tracks the batch the worker is currently processing.
If the Timestamp-based model (referred to as SystemTS) is used, the system assigns a
timestamp to the entire batch. The process then proceeds by executing the \Traverse methods 
of \BC\ and \TP\ with no overalap to insert the batch into the iSAX tree.
% 
As mentioned earlier, before moving to the next batch, the summarization buffers must
be cleared to ensure they are ready for reuse (line~\ref{alg:reuse}).
Finally, the algorithm calculates the time it took to insert the last batch
into the index. If this time is less than the configured $DELAY$ value set by the user,
the index workers will be delayed for the remaining time before processing the next batch.
% 
\newline \noindent{\bf QueryAnswering. } 
If the system follows the Timestamp-based model, it assigns a
timestamp to each query. If this timestamp is greater than or equal to the timestamp
of the ongoing batch, the query must wait for the batch to complete.
In contrast, if the system uses the Linearizability model,
queries are responsible for incrementing a global counter, $GlobalSeq$, which serves
as a shared sequence number for both the index construction and Query Answering workers.
This is achieved using a CAS (Compare-and-Swap) instruction, where query workers attempt
to update $GlobalSeq$ using their local counter, $localSeq$. The procedure then proceeds by
invoking first the \Traverse method of \PS\ and then the traverse method of \RS\. Finally,
it invokes the the Clean method of \RS to prepare the priority queues for the next query.
followed by clearing the buffers of the \RS\ object (line~\ref{alg:reuseRS}).

\begin{algorithm}[htbp]
    \footnotesize
    \vspace*{2mm}
    
    \begin{algorithmic}[1]
    
    \Procedure{InitializeSharedObjects}{}
        \State \BC $\gets$ initially contains all raw data series
        \State \TP, \PS, \RS $\gets$ initially empty
        \State $\mathit{BSF} \gets$ some initial value
        \State $\mathit{GlobalSeq} \gets 0$
    \EndProcedure
    
    \vspace*{1mm}
    \State{Code for thread $t_i$, $i \in \{0, \ldots, k-1\}$:}    
    \vspace*{1mm}
    
    \Procedure{IndexCreation(int $\mathit{TotalUpdates}$) \textbf{returns} int}{}  \label{alg:IndexCreation}
        \State int $\mathit{currentBatch} \gets 1$
        \While{$\mathit{currentBatch} < \mathit{TotalBatches}$}
            \If{$\mathit{model} == \mathit{Timestamps}$ }
                \State batchTS $\gets$ getTS()
            \EndIf
            \State \BC.\Traverse(\&BufferCreation(), $\mathit{BCParam}$, \False)
            \State \TP.\Traverse(\&TreePopulation(), $\mathit{TPParam}$, \False)
            \State \TP.\Clean() \label{alg:reuse}
            % \ForAll{summarization buffers that have data}
            %     \State ReuseBuffers($\mathit{summBuffer}, \mathit{currentBatch}$)
            % \EndFor
            \If{$\mathit{DELAY} >= \mathit{BatchProcessingTime}$}
                \State sleep($\mathit{DELAY}-\mathit{BatchProcessingTime}$)
            \EndIf
            \State $\mathit{currentBatch} \gets \mathit{currentBatch} + 1$
        \EndWhile
    \EndProcedure
    
    \vspace*{1mm}
    
    \Procedure{QueryAnswering(QuerySeriesSet $\mathit{SQ}$) \textbf{returns} int}{} \label{alg:QueryAnswering}
        \State int $\mathit{localTS} \gets 0$
        \While{$\mathit{!SQ.Empty}$}
            \If{$\mathit{model} == \mathit{Linearizable}$}
                \If{$\mathit{localSeq} == \mathit{GlobalSeq}$}
                    \State \CAS(\&GlobalSeq, localSeq, localSeq + 1)
                \EndIf
            \ElsIf{$\mathit{model} == \mathit{Timestamps}$}
                \State queryTS $\gets$ getTS()
                \While {$\mathit{queryTS} >= \mathit{batchTS}$}
                    \State backoff()
                \EndWhile
            \EndIf
            \State \PS.\Traverse(\&Pruning(), $\mathit{PSParam}$, \False)
            \State \RS.\Traverse(\&Refinement(), $\mathit{RSParam}$, \False)
            \State \RS.\Clean() \label{alg:reuseRS}
            % \ForAll{summarization buffers that have data}
            %     \State ReuseBuffers($\mathit{summBuffer}, \mathit{currentBatch}$)
            % \EndFor
            
            \If{$\mathit{model} == \mathit{Linearizable}$}
                \State $\mathit{localSeq} = \mathit{localSeq} + 1$            
            \EndIf
        \EndWhile
        \State \Return $\mathit{BSF}$
    \EndProcedure
    
    % \vspace*{1mm}
    % \Procedure{ReuseSumBuffers(SumBuff * $\mathit{currBuff}$, int $\mathit{currentBatch}$)}{} \label{alg:BufferReuse}

    %     \State $\mathit{currBuff->size[workerID]} \gets 0$
    %     \If{$\mathit{currBuff->seq[workerID]} < \mathit{currentBatch}$}
    %         \State $\mathit{currBuff->seq[workerID]} \gets \mathit{currentBatch}$
    %     \EndIf
    %     \State $\mathit{currBuff->seq[workerID]++}$
    % \EndProcedure
    
    \end{algorithmic}
    
    \caption{Implementation of a dynamic iSAX-based index using the traverse objects \BC, \TP.}
    \label{alg:DynamiciTraverseObject}
    \end{algorithm}

%%%%%%%%%%%%%%%%%%%%%%% Locality Awareness %%%%%%%%%%%%%%%%%%%%%%%%%%%%%%%%%%%%%%%%%%%%%%%%%%%%%
\chapter{Locality-Awareness}
\label{chapter:Locality-aware}

\begin{figure*}[tb]
    \centering
    \includegraphics[width=\textheight,angle=90]{figures/locality-aware/Refresh_ekosmas_2-Themis.pdf}
    \caption{Refresh flowchart.}
	\label{fig:refresh:flowchart}
\end{figure*}

\section{Static iSAX-based Indexes}

{\em Locality-awareness}  aims at capturing several design principles (Definition~\ref{def:principles})
for data series indexes which are crucial for achieving good performance.
A locality aware implementation respects these principles. 
% 
\begin{definition}
\label{def:principles}
Principles for {\em locality-aware} processing:
\begin{enumerate}
\item \normalfont
{\bf Data Locality.} Separate the data into {\em disjoint sets} and have a distinct thread
processing the data of each set. This results in reduced communication 
cost (cache misses and branch misprediction) among the threads.
\item \normalfont
{\bf High Parallelism \& Low Synchronization Cost.} 
Threads should work in parallel and independently from each other as much as possible. 
Whenever synchronization cannot be avoided, design the appropriate 
mechanisms to minimize its cost.
\item \normalfont
{\bf Load Balancing.} Share the workload equally to the different threads, thus
avoiding load imbalances between threads and having all threads busy at each
point in time. 

\end{enumerate}
\end{definition}
% 
Enuring locality awareness results in good performance and is thus 
a desirable property for big data processing. In existing static  iSAX-based indexes, 
a thread operates on chunks of \textit{RawData} and processes disjoint sets of summarization
buffers and subtrees of the index tree. Also, an iSAX-based index employs 
several priority queues to store leaf nodes containing candidate series.
Thus, iSAX-based indexes are {\em locality-aware}.

To describe \textit{ReFreSh} in more detail, consider a {\em blocking} locality-aware
implementation $\mathcal{A}$, which splits its workload into disjoint parts and assigns them to 
threads for processing. 
% 
The main idea behind it is to require from each thread that completes processing
its own workload (instead of blocking, waiting for other threads to also finish) 
to scan for unprocessed workloads (e.g. in an iSAX-based index, unprocessed parts of $RawData$
or unprocessed summarization buffers), and help completing their processing. 

\textit{ReFreSh} (Algorithm~\ref{alg:refresh}) transforms $\mathcal{A}$ into a {\em lock-free} 
locality-aware implementation $\mathcal{B}$ that achieves high parallelism. 
% 
Let $W$ be the workload that $\mathcal{A}$ processes 
and let $w_1, \ldots, w_k$ be the parts it is separated to ensure locality awareness.
\textit{ReFreSh} applies for each data structure $D$ of $\mathcal{A}$, 
the following steps (depicted in Figure~\ref{fig:refresh:flowchart}):


%%%%%%%%%%%%%%%%%%%%%%%%%%%%%%% REFRESH ALGORITHM %%%%%%%%%%%%%%%%%%%%%%%%%%%%%%%%%%%%%% 

\begin{algorithm}[htbp]
    \footnotesize
    \vspace*{2mm}
    
    \begin{algorithmic}[1]
    
    \Procedure{InitializeSharedVariables}{}
        \State Workload part $\mathit{W} \gets [w_1, w_2, \ldots, w_k]$ \label{alg:refresh:w}
        \State Boolean array $\mathit{D} \gets [d_1, d_2, \ldots, d_k]$, initially $d_i = \False$ \label{alg:refresh:d}
        \State Boolean array $\mathit{H} \gets [h_1, h_2, \ldots, h_k]$, initially $h_i = \False$ \label{alg:refresh:h}
    \EndProcedure
    
    \vspace*{1mm}
    \State{Code for each thread:}    
    \vspace*{1mm}
    
    \Procedure{Refresh}{}
        \While{$\mathit{W}$ has available parts}  \label{alg:refresh:process:start}
            \State $\mathit{w_i} \gets$ Acquire an available part of $\mathit{W}$
            \State Mark $\mathit{w_i}$ as acquired
            \If{$h_i == \False$}  \label{alg:refresh:process:if}
                \State Process $\mathit{w_i}$ in expeditive mode, while checking that $h_i$ remains $\False$ \label{alg:refresh:process:expeditive}
                \State If $h_i == \True$, switch to standard mode
            \Else
                \State Process $\mathit{w_i}$ in standard mode \label{alg:refresh:process:standard}
            \EndIf
            \State $d_i \gets \True$ \label{alg:refresh:d:true}
        \EndWhile
        
        \vspace*{1mm}
        \ForAll{$d_i \in D$ where $d_i == \False$}  \label{alg:refresh:scan:ForAll} 
            \State Backoff()  \Comment{Avoid helping, if possible} \label{alg:refresh:help:backoff}
            \If{$d_i == \False$}  \label{alg:refresh:help:if}
                \State $h_i \gets \True$ \label{alg:refresh:h:true}
                \State Process $\mathit{w_i}$ in standard mode, checking periodically if $d_i$ remains $\False$ \label{alg:refresh:help:process}
                \State If $d_i == \True$, stop processing $\mathit{w_i}$
                \State $d_i \gets \True$ \label{alg:refresh:help:d:true}
            \EndIf
        \EndFor
    \EndProcedure
    
    \end{algorithmic}
    
    \caption{\textit{ReFreSh} - A general approach for transforming a blocking data structure $\mathit{D}$ into a lock-free one.}
    \label{alg:refresh}
\end{algorithm}

\newpage

\begin{enumerate}
    \item It attaches a {\em flag} $d_i$, $1 \leq i \leq k$, (initially \False)  
    with each $w_i$ to identify whether $w_i$'s processing is done.  
    As soon as a thread finishes processing $w_i$, it sets $d_i$ to \True\ (line~\ref{alg:refresh:d:true}).  

    \item Threads in $\mathcal{B}$ execute the same algorithm as in $\mathcal{A}$ to acquire parts of $W$ to process,  
    until all parts have been acquired (lines~\ref{alg:refresh:process:start}-\ref{alg:refresh:d:true}).  
    The thread that acquires a workload is its {\em owner}.  

    \item To achieve lock-freedom, every thread $t$, then, {\em scans} all the flags ($d_i$, $1 \leq i \leq k$)  
    to find those parts that are still unfinished (line~\ref{alg:refresh:scan:ForAll}).  

    \item Thread $t$ {\em helps} by processing, one after the other, each part found unfinished during scan.  
    For each part $w_i$, $1 \leq i \leq k$, that $t$ helps, it periodically checks $d_i$  
    to see whether other threads completed the processing of $w_i$. If this is so,  
    $t$ stops helping $w_i$ (line~\ref{alg:refresh:help:process}).  
    A thread that completes the processing of $w_i$ changes $d_i$ to \True\ (line~\ref{alg:refresh:help:d:true}).  

    \item Due to helping, every data structure $D$, employed in $\mathcal{A}$, may be  
    concurrently accessed by many threads. Thus, $\mathcal{B}$ should provide an efficient  
    lock-free implementation for all data structures of $\mathcal{A}$.  
\end{enumerate}


In locality-aware implementations threads are expected to work
on their own parts of the data most of the time ({\em contention-free phase}), and they
may help other threads only for a small period of time at the end of their execution
({\em concurrent phase}).
In the contention-free phase, \textit{ReFreSh} avoids synchronization overheads incurred to
ensure lock-freedom. 
Specifically, it employs two implementations for each data structure $D$ of  $\mathcal{A}$,
one with low synchronization cost that does not support helping ({\em expeditive mode}),
and another that supports helping and has higher synchronization overhead ({\em standard mode}).  
To enable threads operate on the appropriate mode,  a {\em helping-indicator flag} $h_i$ 
(initially \False) , $1 \leq i \leq k$, is attached with each $w_i$, which indicates whether 
$w_i$'s processing should be performed on expeditive or standard mode.
% 
A thread $t$ starts by processing its assigned workload 
on expeditive mode (lines~\ref{alg:refresh:h} and \ref{alg:refresh:process:if}-\ref{alg:refresh:process:expeditive}).
Before $t$ starts helping some part $w_i$, it sets $h_i$ to \True\ (line~\ref{alg:refresh:h:true}),
to alert $w_i$'s owner thread to start running on standard mode (line~\ref{alg:refresh:process:expeditive}).

To avoid helping whenever it is not absolutely necessary, i.e. when no thread has
failed (or is extremely slow), \textit{RefreSh} provides an optional {\em backoff scheme} that is used
by every thread $t$ (line~\ref{alg:refresh:help:backoff}) 
before it attempts to help other threads (line~\ref{alg:refresh:help:if}-\ref{alg:refresh:help:process}).
A small delay before switching to standard mode, often positively affects performance.
The delay is usually an estimate of the actual time a thread requires to finish its
current workload, calculated at run time.
% 
To minimize the work performed by a helper, \textit{ReFreSh} could
be applied  {\em recursively} by splitting each part $w_i$ to subparts. 
This way, a helper helps only the remaining unfinished subparts of $w_i$
and not the whole $w_i$. Thus, the redundant work is further decreased. 
To achieve this, the subparts of $w_i$ should have their own flag and helping-indicator variables.

Lock-freedom is ensured due to the {\em helping code} (lines~\ref{alg:refresh:scan:ForAll}-
\ref{alg:refresh:help:d:true}). In \textit{ReFreSh}, only after a thread
$t$ processes a workload $w_i$, it sets $h_i$ to \True\, and $t$ performs 
the helping code after finishing with its assigned workloads. Thus, when $t$ completes
its helping code the processing of all parts of the workload has been completed. 
This means that $t$ may continue directly to the execution of the next stage, without
waiting for the other threads to complete the execution of the current stage. 
Therefore, this scheme renders the use of barriers useless, as needed to achieve lock-freedom. 

Summarizing, \textit{ReFreSh} is a general scheme for processing a locality-aware
static workload in a lock-free way, without sacrificing locality-awareness. 


\subsection{\textbf{ReFreSh with Dynamic Batches}}  

In an iSAX-based index, \textit{ReFreSh} is applied to all stages to implement the traverse
method. When data arrives dynamically in batches, \textit{ReFreSh} must be applied to process
each batch. This brings some complications when processing data structures such as the summarization
buffers and the iSAX tree.
% 
The core issue lies in how \textit{ReFreSh} determines when to switch execution modes.
Its behavior relies on boolean flags that initially have the value \texttt{False} to 
indicate execution in expeditive mode and transition
to \texttt{True} when processing becomes standard. After creating the initial index, processing any
ongoing batch will encounter these flags already set to \texttt{True}, causing \textit{ReFreSh} to 
run in \textit{standard mode}. This is suboptimal because \textit{standard mode} enables helping,
which is unnecessary unless a helper has actually arrived. To avoid such uneccessary performance
penalties,\textit{ReFreSh} must be modified to support dynamic batch processing effectively.  

To continue ensuring locality-awareness each batch is processed in a locality-aware way. 
Specifically for each batch $b_j$, let $w_j$ be the workload for processing $b_j$ 
and let $w(j,i)$ $1 \leq i \leq k$, be the parts of this workload.
Thus, for each batch $b_j$ we define flags $d(j,i)$, $h(j,i)$ and we run an instance
of \Refresh. Moreover we apply the following modifications:

\begin{enumerate}  
    \item \textbf{Replacing Boolean Flags with Counters:}  
    Instead of using boolean flags, DFreSh replaces them with counters. Each part $w(j,i)$,
    is assigned a counter $d(j,i)$ (initially set to 0). This counter tracks whether $w(j,i)$ has
    completed processing part $w(j,i)$ for $b_j$. Once a thread finishes processing $w(j,i)$, it
    sets $d(j,i)$ to $\mathit{j} + 1$, which corresponds to the index of the next
    batch (see Algorithm~\ref{alg:DRefresh}).  

    \item \textbf{Batch-Aware Execution of \textit{ReFreSh}:}  
    Each time \textit{ReFreSh} is executed while processing batch $b_j$, the following holds:
    \begin{itemize}  
        \item A part $w(j,i)$ is considered processed for a batch with index $j$ when $d(j,i) = j + 1$.  
        \item A helper has arrived at part $w(j,i)$ for batch $b_j$ when $h(j,i) = j + 1$.  
    \end{itemize}  
\end{enumerate}  
% 
By implementing these modifications, we ensure that (a) execution mode transitions 
occur only when necessary and (b) \textit{ReFreSh} is correctly reused when moving 
from one batch to another.
These changes on Algorithm~\ref{alg:refresh} are illustrated with red in Algorithm~\ref{alg:DRefresh}.

%%%%%%%%%%%%%%%%%%%%%%%%%%%%%%% Dynamic Refresh %%%%%%%%%%%%%%%%%%%%%%%%%%%%%%%%%%%%%%%%

\begin{algorithm}[htbp]
    \footnotesize
    \vspace*{2mm}
    
    \begin{algorithmic}[1]
    
    \State \textbf{Shared variables:}
    \State Workload part $\mathit{W_j} := \mathit{[w(j,1), w(j,2), \ldots, w(j,k)]}$ \label{alg:DRefresh:w}
    \State \textbf{Int} array $\mathit{D_J} := \mathit{[d(j,1), d(j,2), \ldots, d(j,k)]}$, initially $\mathit{d(j,i)} = 0, \forall i \in {1,...,k}$ \label{alg:DRefresh:d}
    \State \textbf{Int} array $\mathit{H_J} := \mathit{[h(j,1), h(j,2), \ldots, h(j,k)]}$, initially $\mathit{h(j,i)} = 0, \forall i \in {1,...,k}$ \label{alg:DRefresh:h}
    
    \vspace*{1mm}
    \State \textbf{Code for each thread:}
    
    \Procedure{DRefresh}{int $\mathit{j}$}
        \While{$\mathit{W}$ has available parts} \label{alg:DRefresh:process:start}
            \State $\mathit{w(j,i)} \gets$ acquire an available part of $\mathit{W_j}$
            \State Mark $\mathit{w(j,i)}$ as acquired \label{alg:DFreSh:acquire-part}
            \If{\rb{$(\mathit{h(j,i)} == \mathit{j})$} \textbf{or} \rb{$(\mathit{d(j,i)} == \mathit{j}$ 
            \textbf{and} $\mathit{h(j,i)} < \mathit{j})$}} \label{alg:DRefresh:process:if}
                \State Process $\mathit{w(j,i)}$ in expeditive mode, while checking the value of $h(j,i) \leq j$ \label{alg:DRefresh:process:expeditive}
                \If{\rb{$\mathit{h(j,i)} == \mathit{j} + 1$}}\label{alg:DFreSh:switch-mode}
                    \State Switch to standard mode
                \EndIf
            \Else
                \State Process $\mathit{w(j,i)}$ in standard mode \label{alg:DRefresh:process:standard}
            \EndIf
            \State \rb{$\mathit{CAS(\&d(j,i), j, j+1)}$} \label{alg:DRefresh:d:increase}
        \EndWhile
        
        \ForAll{$\mathit{d(j,i)} \in D$ where $\mathit{d(j,i)} == \mathit{j}$} \label{alg:DRefresh:scan:ForAll} 
            \State Backoff() \label{alg:DRefresh:help:backoff}
            \rb{\If{$\mathit{d(j,i)} == \mathit{j}$} \label{alg:DRefresh:help:if}
                \If{$\mathit{h(j,i)} == \mathit{j}$} \label{alg:DRefresh:h:true}
                    \State $\mathit{CAS(\&h(j,i), j, j+1)}$ 
                \ElsIf{$\mathit{h(j,i)} < \mathit{j}$}      \label{alg:DRefresh:h:fallenBehind}
                    \State \textbf{int} oldVal $\gets \mathit{h(j,i)}$
                    \State $\mathit{CAS(\&h(j,i), oldVal, j+1)}$ \label{alg:DRefresh:h:CAS}
                \EndIf
            \EndIf
            }
            \State Process $\mathit{w(j,i)}$ in standard mode, checking $\mathit{d(j,i)} == j$
            \rb{\State $\mathit{CAS(\&d(j,i), j, j+1)}$ }\label{alg:DRefresh:help:d:true}
        \EndFor
    \EndProcedure
    
    \end{algorithmic}
    
    \caption{Dynamic version of ReFresh - Code for processing batch $B_j$.}
    \label{alg:DRefresh}
    \end{algorithm}

    We explain these changes in more detail below.

    \begin{enumerate}
        \item As in \textit{ReFreSh}, threads acquire parts (line~\ref{alg:DFreSh:acquire-part}) of the
        workload $W_j$ of batch $B_j$ for processing until all parts have been acquired. Each thread
        must determine the appropriate processing mode (lines~\ref{alg:DRefresh:process:if}-
        \ref{alg:DRefresh:process:standard}). A helper has arrived at part $w(j,i)$ of
        batch $B_j$ if the value of $h(j,i)$ is equal to $j + 1$ (line~\ref{alg:DFreSh:switch-mode}).
        Similarly a part $W(j,i)$ is considered processed if the value of $d(j,i)$ is equal to $j+1$.
        Once a thread finishes processing a part, it must try to atomically increment $d(j,i)$ using a
        compare-and-swap (CAS) operation to ensure correctness, as multiple threads may attempt
        to update it simultaneously (line~\ref{alg:DRefresh:d:increase}). 
        We apply an optimization to reduce the number of failed executed compare-and-swap (CAS)
        instructions by checking whether its value is already greater than $j$ which is the old value
        before performing the actual compare-and-swap
        (this optimization is not visible in Algorithm~\ref{alg:DRefresh} for simplicity)

    
        \item To maintain lock-freedom, after all parts become acquired every thread $t$ 
        has to check for unfinished parts in case some threads become slow
        (lines~\ref{alg:DRefresh:scan:ForAll}-\ref{alg:DRefresh:help:d:true}). A part 
        $w(j,i)$ is considered unprocessed if the value of $d(j,1)$ is equal to $j$.
          
        \item While processing unfinished parts of batch $b_j$ thread $t$ periodically checks $d(j,i)$  
        to determine whether other threads have already completed the processing of part $w(j,i)$.
        For each unfinished part $w(j,i)$ that thread $t$ encounters, it first attempts to inform other
        threads, which may be working on the same part, that multiple threads are now processing it
        concurrently. To better understand the pseudocode, let's focus on a specific $i$ and analyze
        what happens with helpers on the $i_{th}$ part of each batch. If during the processing of the
        previous batch a helping thread existed for part $w(j,i)$, then the if condition
        on line~\ref{alg:DRefresh:h:true} is executed. Otherwise, if there was no helper because that
        part was already processed during the scan (lines~\ref{alg:DRefresh:scan:ForAll}-\ref{alg:DRefresh:help:if})
        the else if condition on line (line~\ref{alg:DRefresh:h:fallenBehind}) is executed. 
        The compare-and-swap (CAS) has to figure out the correct old value to use in order to update
        the value of the flag (line~\ref{alg:DRefresh:h:CAS}).
    \end{enumerate}

The flowchart of how \Refresh\ works is shown in Figure~\ref{fig:refresh:flowchart} where 
all the described above mechanisms are visible for a batch $b_j$.
%%%%%%%%%%%%%%%%%%%%%%% FreSh %%%%%%%%%%%%%%%%%%%%%%%%%%%%%%%%%%%%%%%%%%%%%%%%%%%%%%%%%%%%%%%%%%
\chapter{FreSh}
\label{chapter:FreSh}

We follow the data processing flow, described in Chapter~\ref{chapter:traverse-object},
employ $\mathit{BC}$, $\mathit{TP}$, $\mathit{PS}$, and $\mathit{RS}$,
and repeatedly apply \textit{ReFreSh} (from Chapter~\ref{chapter:Locality-aware}) to come up with
\textit{FreSh}, the first lock-free locality-aware data series index.

\section{Counter Object}

Many of our implementations use a {\em counter object}, which supports the
operation \NextIndex\ (Algorithm~\ref{alg:counter}) which returns a positive integer. 
Let \NextIndex\ be invoked $K$ times.
If crashes do not occur, every index in the range $R = 
\{0, \ldots, K-1\}$ should be returned exactly once; 
(i.e. by a distinct of the $K$ invocations) each integer 
is returned exactly once otherwise, as many indexes from $R$,
as the number of crashed  threads, 
are not returned, but the rest are returned exactly once.

Algorithm~\ref{alg:counter} uses two counters. The first, $\mathit{cntEM}$,
is used in the expeditive mode and is updated by simple writes
(line~\ref{alg:co:em:inc-elements}),
whereas the second, $\mathit{cntSM}$, is used in the standard mode and
is updated using \FAI\ (line~\ref{alg:co:FAI-pos}).
As long as no helpers exist ($\mathit{h = \False}$), the owner thread $t$ executes 
a code similar to the sequential code implementing a counter (lines~\ref{alg:co:em:acquire-pos}-
\ref{alg:co:em:checkh1}). When a helper arrives, it sets $\mathit{cntSM}$ to the current
value of $cntEM$ (lines~\ref{alg:co:em:ownerFlag}, \ref{alg:co:em:cntSM-init}),
and uses \FAI\ on $\mathit{cntSM}$ (line~\ref{alg:co:FAI-pos}) to get assigned new indexes. 

Lines~\ref{alg:co:em:elements-bot}-\ref{alg:co:FAI-pos} provide a mechanism for appropriately
synchronizing $t$ with the helper threads when they arrive. Assume that $t$ has read the
value $x$ in $\mathit{cntEM}$, and it is ready to write $x+1$ in $\mathit{cntEM}$,
when two helper threads $t_h$ and $t'_h$ arrive.
Thread $t_h$ reads $x$ in $\mathit{cntEM}$, then $t$ writes $x+1$ 
in $\mathit{cntEM}$ and $t'_h$ reads  $x+1$ in $\mathit{cntEM}$.
Depending on which of the two values will be written in $\mathit{cntSM}$
and the order in which the \FAI\ instructions will be executed, 
one of $t_h$, $t'_h$ may read $x$ in $\mathit{cntSM}$ and return it.
% 
In this case, $t$ should avoid returning $x$ a second time (to ensure the semantics of
a counter object). To achieve this, helpers record in $\mathit{first}$ the value with
which they initialize $\mathit{cntSM}$. 
% 
If $t$ discovers that helpers exist (lines~\ref{alg:co:em:checkh1},~\ref{alg:co:em:ownerFlag}), 
%is the thread to successfully execute the \CAS\ of line~\ref{27}, it stores
%the value $x+1$ there, so helper threads will return this or higher values. 
%In this case, it is safe for $t$ to return $x$ (line~\ref{28}).
%Otherwise, $t$ 
it may have to read $\mathit{first}$ to figure out which is the first value recorded
in $\mathit{cntSM}$. If it is $x$, it  retries
(line~\ref{alg:co:em:continue2}). Note that helpers should first agree on the value 
to record in $\mathit{cntSM}$ with the \CAS\ of line~\ref{alg:co:em:casfirst} and then all 
use this value to initialize $\mathit{cntSM}$ (lines~\ref{alg:co:em:pos2}-~\ref{alg:co:em:cntSM-init}).


\begin{algorithm}[t]
    \footnotesize
    \caption{Pseudocode for \NextIndex.}
    \label{alg:counter}
    \begin{algorithmic}[1]

        \State \textbf{Shared variables:}
        \State int $\mathit{cntEM}$ (initially $0$), $\mathit{cntSM}$ (initially $\bot$)

        \vspace{2mm}

        \Function{NextIndex}{$\mathit{*h}$} \textbf{returns} $\langle \mathit{int, \Boolean} \rangle$
            \State int $\mathit{pos}$
            \State int $\mathit{ownerFlag} \gets \False$
            
            \While{\True}
                \If{$\mathit{*h = \False}$} \label{alg:co:em}
                    \State $\mathit{pos} \gets \mathit{cntEM}$ \label{alg:co:em:acquire-pos}
                    \If{$\mathit{*h == \True}$} 
                        \State \textbf{continue} \label{alg:co:em:continue1}
                    \EndIf
                    \State $\mathit{cntEM} \gets \mathit{cntEM}+1$ \label{alg:co:em:inc-elements}
                    \If{$\mathit{*h == \False}$} 
                        \State \Return $\langle pos, \True \rangle$ \label{alg:co:em:checkh1}
                    \EndIf
                    \State $\mathit{ownerFlag} \gets \True$ \label{alg:co:em:ownerFlag}
                \EndIf

                \If{$\mathit{cntSM = \bot}$} \label{alg:co:em:elements-bot}
                    \If{$\mathit{ownerFlag \neq \True}$} \label{alg:co:em:ifHelper}
                        \State $\mathit{pos} \gets \mathit{cntEM}$ \label{alg:co:em:ownerFlag1}
                        \State $\mathit{\CAS(first, \bot, pos)}$ \label{alg:co:em:casfirst}
                        \State $\mathit{pos} \gets first$ \label{alg:co:em:pos2}
                    \EndIf
                    \State $\mathit{v} \gets \CAS(cntSM, \bot, pos)$ \label{alg:co:em:cntSM-init}
                    \If{$\mathit{v == \bot}$ \textbf{and} $\mathit{ownerFlag = \True}$}
                        \State \Return $\langle pos, \False \rangle$ \label{alg:co:em:ifOwner}
                    \ElsIf{$\mathit{first < cntEM}$} 
                        \State \textbf{continue} \label{alg:co:em:continue2}
                    \EndIf
                \EndIf

                \State $\mathit{pos} \gets \FAI(\mathit{cntSM})$ \label{alg:co:FAI-pos}
                \State \Return $\langle pos, \False \rangle$ \label{alg:co:em:return}
            \EndWhile

        \EndFunction

    \end{algorithmic}
\end{algorithm}



\section{Buffers Creation and Tree Population} 
\label{sec:buffers_and_tree}

\BC\ is implemented using a single buffer, called \RawData.
In \BC, \Put\ is never used, as we assume that the data are initially in \RawData.
% 
To implement \\
\Traverse, we employ \Refresh.  
We split \RawData\ into $k$ equally-sized {\em chunks} of consecutive 
elements , $w_1, \ldots, w_k$. This way we have a number of 
$k$ workloads. 
Threads use a counter object to get assigned chunks to process.
To reduce the cost of helping, \Fresh\ calls \Refresh\ recursively.
Specifically, it splits each chunk into smaller parts,
called {\em groups}, and employ \Refresh\ a second time for processing the groups of a chunk.
% 
In more detail, \Fresh\ maintains an additional counter object 
for each chunk of $RawData$. 
Each thread $t$ that acquires or helps a chunk, uses the counter object of the chunk to acquire
{\em groups} in the chunk to process. \Fresh\ also applies a third level of \Refresh\ recursion,
where each workload is comprised of the processing of just a single element of a group. 

Pseudocode for \BC.\Put\ and \BC.\Traverse\ is provided 
in Algorithm~\ref{alg:bc}. \\ \RawData\ is comprised of $k$ chunks,
each containing $m$ groups. Moreover, each group contains $r$ elements (line~\ref{alg:bc:r}). 
\Fresh\ uses three sets of done flags, $\mathit{DChunks}$, $\mathit{DGroups}$,
and $\mathit{DElements}$ (line~\ref{alg:bc:c}), storing one done flag for each chunk,
for each group, and for each element, respectively.
% 
Similarly, \Fresh\ employs three sets of counter objects, $\mathit{Chunks}$, $\mathit{Groups}$,
and $\mathit{Elements}$ (line~\ref{alg:bc:e}), to count the chunks, groups and elements, assigned
to threads for processing. 
%
\Fresh\ also employs two sets of {\em helping} flags (line~\ref{alg:bc:h}), 
$\mathit{HChunks}$  (for helping chunks) and $\mathit{HGroups}$ (for helping groups). 
For each $1 \leq i \leq k$, $\mathit{HChunks[i]}$ identifies whether there are helpers for chunk $i$.
Similarly, for each $1 \leq j \leq m$, $\mathit{HGroups[i][j]}$ identifies whether there are helpers
for group $j$ of chunk $i$. 

% 
In an invocation of
\Traverse(\&\BufferCreation, \RawData, $\mathit{Dchunks}$, \\ $\mathit{DGroups}$, $\mathit{DElements}$,
$\mathit{HChunks}$, $\mathit{HGroups}$, \False, $\mathit{Chunks}$, $\mathit{Groups}$, $\mathit{Elements}$, $1$),
$h$ is equal to \False. By the way a counter object works, it follows that 
no expeditive mode is ever executed at the first level of the recursion.
Note that at this level, the roles of $D_1$ and $H_1$ are played by the one-dimensional arrays
$DChunks$ and $HChunks$, respectively. Moreover, $DGroups$ and $DElements$ play the role of $D_2$ and $D_3$,
respectively, and $HGroups$ plays the role of $H_2$. Each chunk is processed by recursively calling
\Traverse\ ({\em level-2 recursion}) on line~\ref{alg:bc:recur} (with $\mathit{rlevel}$ being equal to $2$). 
The goal of a level-2 invocation of \Traverse\
is to process an entire chunk by splitting it into groups and calling
\Traverse\ once more ({\em level-3 recursion}) to process the elements of each group 
(recursive call of line~\ref{alg:bc:recur} with $\mathit{rlevel}$ being equal to $3$). 
Note that in a level-2 invocation corresponding to some chunk $i$, 
\RawData\ is the two-dimensional array containing the elements of the groups of chunk $i$.
Moreover, the role of $D_1$ is now played by the one-dimensional array $DGroups[i]$, 
and the role of $D_2$ by the two-dimensional array $DElements[i]$,
whereas $D_3$ is no longer needed and is $\mathit{NULL}$.
The role of $H_1$ is now played by the one-dimensional array $HGroups[i]$. 
Helping (lines~\ref{alg:bc:scan:for}-\ref{alg:bc:help:c:true}) follows the general pattern
described in Algorithm~\ref{alg:refresh}. 

%%%%%%%%%%%%%%%%%%%%%%%%%%%%%%%%%%%%%% PSEYDOCODE%%%%%%%%%%%%%%%%%%%%%%%%%%%%%%%%%%%%%%%%%%%%%%%%%%%%%%%%%%%%%%%%%%%%%%


\begin{algorithm}[t]
    \footnotesize
    \vspace*{2mm}

    \begin{algorithmic}[1]
    
    \State \textbf{Shared variables:}
    \State Set $\mathit{RawData}[1..k][1..m][1..r]$, initially containing all data series \label{alg:bc:r}
    \State \textbf{Boolean} $\mathit{DChunks}[1..k]$, $\mathit{DGroups}[1..k][1..m]$, $\mathit{DElements}[1..k][1..m][1..r]$, initially all \textbf{False} \label{alg:bc:c}
    \State \textbf{Boolean} $\mathit{HChunks}[1..k]$, $\mathit{HGroups}[1..k][1..m]$, initially all \textbf{False} \label{alg:bc:h}
    \State CounterObject $\mathit{Chunks}$, $\mathit{Groups}[1..k]$, $\mathit{Elements}[1..k][1..m]$ \label{alg:bc:e}
    \State \textbf{int} $\mathit{Size}[1..3] = \{k,m,r\}$

    \vspace*{1mm}
    \State \textbf{Code for each thread:}

    \Procedure{Traverse}{Function *BufferCreation, DataSeries $\mathit{RawData}[]$, Boolean $\mathit{D_1}[]$, Boolean $\mathit{D_2}[]$, Boolean $\mathit{D_3}[]$, Boolean $\mathit{H_1}[]$, Boolean $\mathit{H_2}[]$, Boolean $h$, CounterObject $Cnt_1$, CounterObject $Cnt_2[]$, CounterObject $Cnt_3[]$, int $\mathit{rlevel}$}
        \State \textbf{int} $\mathit{i}$
        \While{\textbf{True}} \label{alg:bc:while:start}
            \State $\langle i, * \rangle \gets Cnt_1.\NextIndex(\&h)$
            \If{$i > \mathit{Size}[rlevel]$} \textbf{break} \EndIf
            \State Mark $\mathit{RawData[i]}$ as acquired
            \If{$rlevel < 3$}
                \State Traverse($\mathit{BufferCreation, RawData[i], D_2[i], D_3[i], D_3[i], H_2[i],}$
                \Statex \quad $\mathit{NULL, H_1[i], Cnt_2[i], Cnt_3[i], Cnt_3[i], rlevel+1}$) \label{alg:bc:recur}
            \Else
                \State $*$\Call{BufferCreation}{$RawData[i]$}
            \EndIf
            \State $\mathit{D_1[i]} \gets \textbf{True}$ \label{alg:bc:c:true}
        \EndWhile

        \ForAll{$j$ such that $\mathit{D_1[j]}$ is \textbf{False}} \label{alg:bc:scan:for}
            \State Backoff() \Comment{Avoid helping if possible} \label{alg:bc:help:backoff}
            \If{$\mathit{D_1[j]}$ is \textbf{False}} \label{alg:bc:help:if}
                \State $\mathit{H_1[j]} \gets \textbf{True}$ \label{alg:bc:h:true}
                \If{$rlevel < 3$}
                    \State Traverse($\mathit{BufferCreation, RawData[j], D_2[j], D_3[j], D_3[j], H_2[j],}$
                    \Statex \quad $\mathit{NULL, H_1[j], Cnt_2[j], Cnt_3[j], Cnt_3[j], rlevel+1}$) \label{alg:bc:help:process}
                \Else
                    \State $*$\Call{BufferCreation}{$RawData[j]$}
                \EndIf
                \State $\mathit{D_1[j]} \gets \textbf{True}$ \label{alg:bc:help:c:true}
            \EndIf
        \EndFor
    \EndProcedure

    \end{algorithmic}

    \caption{Pseudocode for \textsc{Traverse} in \textsc{FreSh}. Code for thread $t$.}
    \label{alg:bc}
\end{algorithm}

The backoff time in \Fresh\ depends on the average execution time required by each
thread to process a group. Each thread $\mathit{t}$ counts the average time 
$T_{avg}$ it has spent to process all the parts it acquired, and whenever
it encounters a group to help, it sets the backoff time to be proportional
to $T_{avg}$ and performs helping only after backoff, if it is still needed.
This allows an owner thread that has not crashed, to finish the processing of its last chunk,
while it still executes in expeditive mode, thus avoiding the cost of switching
to and executing a standard mode.
% 
\Fresh\ implements \TP\ using a set of $2^w$ summarization buffers
($w$ is the number of segments of an iSAX summary),one for each bit sequence of $w$ bits.
To decide to which summarization buffer to store a pair,
\Fresh\ (as other iSAX-based indexes) 
examines the bit sequence consisting of the first bit of each 
of the $w$ segments of the pair's iSAX summary, 
and places the pair into the corresponding summarization buffer. 
Each of the summarization buffers is split into $N$ parts, one for each of the $N$ threads in the system.
Each thread uses its own part in each buffer to store the elements it inserts.


\noindent
{\bf Tree Population Stage.}
Similarly to the buffers creation stage, in tree population, the worker threads
have to traverse and process the elements of \TP, i.e. all pairs added in the summarization buffers.
Processing is now achieved by calling the \TreePopulation\ function (Algorithm~\ref{alg:iSAXTraverse})
for each pair. \TreePopulation\ finds the right subtree of the index tree to place each pair, and then 
simply calls \PS.\Put\ to add the pair into \PS, the next traverse object in the dataflow pipeline. 
% Recall that \TP.\Put\ is implemented by calling \MBInsert\ (Algorithm~\ref{alg:mb}).
To implement \TP.\Traverse, we split the elements of \TP\ into $2^w$ workloads 
as the number of summarization buffers, and apply \Refresh.
%,
Each thread $\mathit{t}$ repeatedly acquires summarization buffers using \FAI,
and process them to produce the corresponding trees. 
% 
Each summarization buffer could be further split into chunks and groups, and \Refresh\ could be called recursively.
Pseudocode for \Traverse\ of \TP\ closely follows that for \BC.
\BC\ and \TP\ are lock-free implementations of a traverse object. 

\section{Prunning and Refinement}

In \Fresh, \PS\ is implemented as a forest of $2^w$ leaf-oriented trees,
one for each of the summarization buffers. 
The trees of the forest are the root subtrees of a standard iSAX-based tree.
\TreePopulation\ simply transfers the pairs from a summarization buffer
to the appropriate subtree of the index tree. 
To support the concurrent population of a subtree by multiple threads, 
\Fresh\ utilizes Algorithm~\ref{alg:tree} 
%presented in Section~\ref{sec:trees}. % (and discussed later). 
%Specifically, the \Put\ operation of \PS\ (see Algorithm~\ref{alg:ps}) is implemented
%by simply calling \TreeInsert\ (Algorithm~\ref{alg:tree}). Whenever this routine
%is called by a thread that process an originally assigned summarization buffer, 
%it should have its $\mathit{isHelper}$ argument equal to $\False$,
%whereas when it is called by a thread that helps, it should have this argument equal to $\True$.
%
To implement \PS.\Traverse, \Fresh\ (Algorithm ~\ref{alg:ps}) uses \Refresh\ to process 
the different subtrees of the index tree. Specifically, 
each thread $t$ access a counter to get assigned a subtree $T$ to process.
To process the nodes of $T$, \Refresh\ is applied recursively. % (line~\ref{alg:ps:help:c:rec}). 
%Specifically, the threads working on $T$ %(owner and helpers) 
%use a counter to get assigned nodes in $T$.
A thread $t$ that is assigned node $i$ of $T$,
first searches for the $i$-th node, according to inorder,
and then processes it by invoking the \Prunning\ function %(line~\ref{alg:ps:prunning}) 
of Algorithm~\ref{alg:iSAXTraverse}. 
%As soon as $t$ figures out that all nodes of $T$ have been assigned for processing, 
%it re-traverses the nodes of $T$,
%checking their done flags to figure out whether there are nodes 
%whose processing has not been completed yet, and helps
%whenever it is needed. Similarly, when $t$ figures out that all subtrees have been
%assigned for processing, it examines the done flags of the subtrees to figure out
%whether there are subtrees whose processing has not yet been completed, and it helps if needed. 
%
%Pseudocode for implementing \Traverse\ of \PS\ is provided in .
To find the $i$-th node of $T$ in an efficient way, 
for each node $nd$ of $T$, \Fresh\ maintains a counter
$\mathit{cnt_{nd}}$ that counts the number of nodes in the left subtree
of $\mathit{nd}$. \Fresh\ %(lines~\ref{alg:ps:help:c:treenode-start}-\ref{alg:ps:help:c:treenode-end})
uses these counters to find the $i$-th node of $T$ by simply traversing a path in $T$.
The total number of nodes in T is calcualted by simply traversing the righmost path of
$T$ and summing up the counters stored in the traversed nodes.
%Pseudocode for \FindNode\ and \TotalNodes\ is provided in Algorithm~\ref{alg:ps}.
%\here{It is unclear to which algorithm the line numbers above refer.}

%%%%%%%%%%%%%%%%%%%%%%%%%%%%%%%%%%%% Pseudocode for PRUNNING %%%%%%%%%%%%%%%%%%%%%%%%%%%%%%%%%%%%%%%

\begin{algorithm}[htbp]
    \footnotesize
    \vspace*{2mm}
    
    \begin{algorithmic}[1]
    
    \State \textbf{Shared variables:}
    \State TreeNode *$\mathit{IndexTree}[1..2^w]$ \label{alg:ps:r}
    \State \textbf{bool} $\mathit{DTree[1..2^w]}$, $\mathit{HTree[1..2^w]}$, initially all \textbf{false} \label{alg:ps:c}
    \State CounterObject $\mathit{TreeCnt[1..2^w]}$
    
    \vspace*{1mm}
    \Procedure{Traverse}{Function *\Prunning, TreeNode *$T$, CounterObject *$\mathit{Cnt}$, 
        int $x$, bool $h$, int $\mathit{rlevel}$}
        \State int $\mathit{i}$
    
        \While{\textbf{true}}
            \State $\mathit{\langle i, * \rangle = Cnt.\NextIndex(\&h)}$ \label{alg:ps:help:c:cnt}
            \If{$\mathit{i > x}$}
                \State \textbf{break}
            \EndIf
            \If{$\mathit{rlevel < 2}$}
                \State \textbf{mark} $\mathit{IndexTree[i]}$ as acquired
                \State $\mathit{totalNds} \gets$ \Call{TotalNodes}{$\mathit{IndexTree[i]}$}
                \State Traverse($\mathit{Prunning, IndexTree[i], TreeCnt[i], totalNds,}$
                \Statex \quad $\mathit{\textbf{false}, rlevel+1}$) \label{alg:ps:help:c:rec}                
                \State $\mathit{DTree[i] := \textbf{true}}$ \label{alg:ps:help:c:true}
            \Else
                \State $\mathit{nd} \gets$ \Call{FindNode}{$\mathit{T,i}$} \label{alg:ps:findnode}
                \State \textbf{mark} $\mathit{nd}$ as acquired
                \State \Call{\Prunning}{$\mathit{nd}$} \label{alg:ps:prunning}
                \State \textbf{mark} $\mathit{nd}$ as done
            \EndIf
        \EndWhile
    
        \ForAll{$j$ such that $\mathit{DTree[j]}$ is \textbf{false}} \label{alg:ps:scan:for}
            \State \Call{Backoff}{} \Comment{Avoid helping, if possible} \label{alg:ps:help:backoff}
            \If{$\mathit{DTree[j]} == \textbf{false}$} \label{alg:ps:help:if}
                \State $\mathit{HTree[j] := \textbf{true}}$ \label{alg:ps:h:true}
                \State \Call{HelpTree}{\Prunning, $\mathit{IndexTree[j]}$}
            \EndIf
            \State $\mathit{DTree[j] := \textbf{true}}$ \label{alg:ps:help:c:true:helping}
        \EndFor
    \EndProcedure
    
    \vspace*{1mm}
    \Procedure{FindNode}{TreeNode *$T$, int $i$} \Comment{Returns TreeNode*} \label{alg:ps:help:c:treenode-start}
        \State TreeNode *$\mathit{p} \gets T$
        \State int $\mathit{nds} \gets 0$
    
        \While{$\mathit{p \neq NULL\ \And\ nds \neq i}$}
            \If{$\mathit{nds + p \rightarrow cnt + 1 < i}$}
                \State $\mathit{nds} \gets \mathit{nds + p \rightarrow cnt + 1}$
                \State $\mathit{p} \gets p \rightarrow rc$
            \Else
                \State $\mathit{p} \gets p \rightarrow lc$
            \EndIf
        \EndWhile
        \State \Return $\mathit{p}$ \label{alg:ps:help:c:treenode-end}
    \EndProcedure
    
    \vspace*{1mm}
    \Procedure{HelpTree}{Function *$f$, TreeNode *$T$}
        \If{$T == NULL$}
            \State \Return
        \EndIf
        \State \Call{HelpTree}{$f$, $T \rightarrow lc$}
        \If{$*T$ is unprocessed}
            \State $*f(*T)$
        \EndIf
        \State \Call{HelpTree}{$f$, $T \rightarrow rc$}
    \EndProcedure
    
    \end{algorithmic}
    
    \caption{Pseudocode for \Put\ and \Traverse\ of \PS\ in \Fresh. Code for thread $t \in \{ 1, \ldots, N-1\}$.}
    \label{alg:ps}
    \end{algorithm}


    \subsection{Insert in Leaf-Oriented Tree}
    \label{sec:leaf-oriented}

    Each node of the tree stores a key and the pointers
    to its left and right children.  (refer to Algorithm~\ref{alg:tree}).
    A leaf node stores additionally an array $D$, where the leaf's data are stored. 
    We assume that each data item is a pair containing a key and the associated information.
    A node may have its own key. For instance, in iSAX-based indexes, this key is the node's
    iSAX summary which summarizes all data series stored in it.
    The proposed implementation allows multiple insert operations to concurrently update array
    $D$ of a leaf. This results in enhanced parallelism and performance. 
    To achieve this, each leaf $\ell$ contains a counter, called $\mathit{Elements}$.
    % 
    Each thread $\mathit{t}$ that tries to insert data in $\ell$, uses
    $\mathit{Elements}$ to acquire a position $\mathit{pos}$ in the array $D$ of $\ell$.
    If $D$ is not full, (line~\ref{alg:tree:pos-in-D}) (i.e. $\mathit{pos} < M$), 
    $\mathit{t}$ stores the new element in $D[\mathit{pos}]$ (line~\ref{alg:tree:store-in-D}).
    Otherwise, in case the array is full, i.e. $\mathit{pos \geq M}$, then 
    $\mathit{t}$ attempts to split the leaf (line~\ref{alg:tree:pos-not-in-D}). 
    % 
    During spliting, $D$ may contain empty positions, since some
    threads may have acquired positions in $D$ but have not yet stored their
    elements there (line~\ref{alg:tree:store-in-D}).
    To avoid situations of missing elements, each leaf contains an $\mathit{Announce}$ array 
    with one position for each thread. A thread announces its operation
    in $\mathit{Announce}$ before it attempts to acquire a position in $D$.
    During spliting, a thread distributes to the new leaves it creates not only the
    elements found in $D$ but also those in $\mathit{Announce}$.

    %%%%%%%%%%%%%%%%%%%%%%%%%%%%%%Tree Insert Code%%%%%%%%%%%%%%%%%%%%%%%%%%%%%%%
    \begin{algorithm}[t]
        \footnotesize
        \caption{Type Definitions for the Lock-Free Tree}
        \label{alg:tree-types}
        \begin{algorithmic}[1]
            \Statex \textbf{Type Definitions:}
            \State \textbf{Type} Node:
            \State \quad int $\mathit{key}$
            \State \quad $\{Node,Leaf\}$ *$\mathit{left}$
            \State \quad $\{Node,Leaf\}$ *$\mathit{right}$
            \State \quad InsertRec $\mathit{Announce}[0..n-1]$
            \State \quad \textbf{Boolean} $\mathit{helpersExist}$
    
            \Statex
            \State \textbf{Type} InsertRec:
            \State \quad Data $\mathit{data}$
            \State \quad int $\mathit{position}$
    
            \Statex
            \State \textbf{Type} Leaf \textbf{extends} Node:
            \State \quad Data $D[0..m-1]$
            \State \quad CounterObject Elements
        \end{algorithmic}
    \end{algorithm}
    


    \begin{algorithm}[t]
        \footnotesize
        \caption{TraverseTree: a lock-free leaf-oriented tree with fat leaves, implementing a traverse object.
         Code for thread $t \in \{ 1, \ldots, N-1\}$.}
        \label{alg:tree}
        \begin{algorithmic}[1]
    
            \Statex \textbf{Shared Variables:}
            \State $\mathit{Tree} \gets \NULL$ \Comment{Initially points to a Leaf with initialized values}
    
            \Procedure{TreeInsert}{$\mathit{data}, \mathit{isHelper}$}
                \State $\mathit{leaf} \gets$ \NULL
                \State $\mathit{parent} \gets \mathit{Tree}$
                \State $\mathit{ptr} \gets$ \NULL
                \State $\mathit{pos}, \mathit{val} \gets 0$
                \State $\mathit{expeditive} \gets$ \False
    
                \While{\True} \label{alg:tree:while}
                    \State $\langle \mathit{leaf, parent} \rangle \gets$ \Call{Search}{$\mathit{data, parent}$} \label{alg:tree:search}
                    \If{$\mathit{parent} = \NULL$} \label{alg:tree:ptr-parent:null}
                        \State $\mathit{ptr} \gets \&\mathit{Tree}$
                    \ElsIf{$\mathit{parent} \rightarrow \mathit{left} = \mathit{leaf}$}
                        \State $\mathit{ptr} \gets \&\mathit{parent} \rightarrow \mathit{left}$ \label{alg:tree:ptr-left}
                    \Else
                        \State $\mathit{ptr} \gets \&\mathit{parent} \rightarrow \mathit{right}$ \label{alg:tree:ptr-right}
                    \EndIf
    
                    \If{$\mathit{isHelper} = \True$ \textbf{and} $\mathit{leaf} \rightarrow \mathit{helpersExist} = \False$}
                        \State $\mathit{leaf} \rightarrow \mathit{helpersExist} \gets \True$ \label{alg:tree:switch-mode:}
                    \EndIf
    
                    \State $\langle \mathit{pos, expeditive} \rangle \gets$ \Call{NextIndex}{$\& \mathit{leaf} \rightarrow \mathit{helpersExist}$} \label{alg:tree:get-pos}
    
                    \If{$\mathit{expeditive} = \False$}
                        \State $\mathit{leaf} \rightarrow \mathit{Announce}[t] \gets \langle \mathit{data, \bot} \rangle$ \label{alg:tree:announce:op}
                    \EndIf
    
                    \If{$\mathit{pos} < M$} \label{alg:tree:pos-in-D}
                        \If{$\mathit{expeditive} = \False$}
                            \State $\mathit{leaf} \rightarrow \mathit{Announce}[t].\mathit{position} \gets \mathit{pos}$ \label{alg:tree:announce:pos-in-D}
                        \EndIf
                        \State $\mathit{leaf} \rightarrow \mathit{D}[\mathit{pos}] \gets \mathit{data}$ \label{alg:tree:store-in-D}
                    \Else \label{alg:tree:pos-not-in-D}
                        \State \Call{SplitLeaf}{$\mathit{leaf}, \mathit{ptr}, \mathit{expeditive}$}
                    \EndIf
    
                    \If{$(*\mathit{ptr}) \rightarrow \mathit{helpersExist} = \True$} \label{alg:tree:st:finish}
                        \If{$\mathit{expeditive} = \False$ \textbf{and} $(*\mathit{ptr}) \rightarrow \mathit{Announce}[t].\mathit{position} \neq \bot$} \label{alg:tree:op-is-applied}
                            \State $(*\mathit{ptr}) \rightarrow \mathit{Announce}[t] \gets \langle \bot, \bot \rangle$ \label{alg:tree:clean}
                        \Else
                            \State \textbf{continue} \label{alg:tree:re-attempt}
                        \EndIf
                    \EndIf
                    \State \Return
                \EndWhile
            \EndProcedure
    
            \Procedure{SplitLeaf}{$\mathit{leaf}, \mathit{prt}, \mathit{expeditive}$}
                \State $\mathit{newNode} \gets$ \Call{NewNode}{} \label{alg:tree:split:new-internal-node}
                \State $\mathit{newNode} \rightarrow \mathit{left} \gets$ \Call{NewLeaf}{}
                \State $\mathit{newNode} \rightarrow \mathit{right} \gets$ \Call{NewLeaf}{}
                \State $\mathit{splitBuffer} \gets \emptyset$
    
                \If{$\mathit{expeditive} = \False$} \label{alg:tree:em:Announce-scan}
                    \For{$i \in \{0, \dots, n-1\}$ \textbf{where} $\mathit{leaf} \rightarrow \mathit{Announce}[i].\mathit{data} \neq \bot$} \label{alg:tree:Announce-scan}
                        \State $\mathit{ldata} \gets \mathit{leaf} \rightarrow \mathit{Announce}[i].\mathit{data}$
                        \If{$\mathit{leaf} \rightarrow \mathit{Announce}[i].\mathit{position} \neq \bot$} \label{alg:tree:announce-scan:pos-not-bot}
                            \State $\mathit{leaf} \rightarrow \mathit{D}[\mathit{leaf} \rightarrow \mathit{Announce}[i].\mathit{position}] \gets \mathit{ldata}$ \label{alg:tree:announce-scan:store-in-D}
                        \Else
                            \State \Call{AddToBuffer}{$\mathit{ldata}, \mathit{splitBuffer}$} \label{alg:tree:announce-scan:data-copy}
                        \EndIf
                        \State $\mathit{newNode} \rightarrow \mathit{Announce}[i] \gets \langle \mathit{ldata}, -1 \rangle$  \label{alg:tree:announce-scan:mark-op-applied}
                    \EndFor
                \EndIf
    
                \State \Call{Distribute}{$\mathit{leaf} \rightarrow \mathit{D}, \mathit{splitBuffer}, \mathit{newNode} \rightarrow \mathit{left}, \mathit{newNode} \rightarrow \mathit{right}$} \label{alg:tree:split}
                \State \Call{CAS}{$*\mathit{prt}, \mathit{leaf}, \mathit{newNode}$} \label{alg:tree:CAS}
            \EndProcedure
    
        \end{algorithmic}
    \end{algorithm}
    
    More specifically, a thread $t$ executing \TreeInsert\ repeatedly executes the following actions. 
    It first calls a standard \textit{Search} routine to traverse a path of the tree and find an appropriate
    $\mathit{leaf}$ and its $\mathit{parent}$ (line~\ref{alg:tree:search}).
    Pointer $\mathit{ptr}$ is a reference to the appropriate child field of $\mathit{parent}$,
    which needs to be changed to perform \TreeInsert.  (lines~\ref{alg:tree:ptr-parent:null}-\ref{alg:tree:ptr-right}).
    Next, $t$ accesses the counter object to acquire a position in $D$ (line~\ref{alg:tree:get-pos})
    and proceeds to announce the data that it wants to insert in the tree (line~\ref{alg:tree:announce:op}).
    Afterwards, it announces this position in $\mathit{Announce}$ and stores the data in $D[\mathit{pos}]$
    (line~\ref{alg:tree:store-in-D}), if $D$ is not full (line~\ref{alg:tree:pos-in-D}). 
    If $D$ is full, it calls \SplitLeaf\ to split $\mathit{leaf}$.
    If this \CAS\ succeeds, then the data have been added and \TreeInsert\ completes. 
    Otherwise, some other thread has successfully split the node.
    
    We finally discuss the following subtle scenario. Assume that the owner thread $t$ calls \TreeInsert,
    reaches a leaf $\mathit{l}$, and acquires the last valid position in array $D$ of $\mathit{l}$. 
    Thread $t$ executes in expeditive mode (so it does not announce its data),
    and before it records its data in $D$, it becomes slow. Next, a helper thread $t'$ reaches $\mathit{l}$, switches 
    $\mathit{l}$'s execution mode to standard, and splits $\ell$ (executing on standard mode). 
    Unfortunately, during this split, $t'$ will not take into 
    consideration the data of $t$, since $t$ neither has announced its operation
    (since $t$ was executing in expeditive mode), nor has yet written its data into $D$. 
    To disallow thread $t$ from finishing its operation without inserting its data, \TreeInsert\ provides the 
    following mechanism (lines~\ref{alg:tree:st:finish}-\ref{alg:tree:re-attempt}). Before it terminates, 
    thread $t$ re-reads the appropriate child field of the parent of $\ell$ (through $\mathit{ptr}$) 
    and checks the $\mathit{helpersExist}$ flag of the node $nd$ that $\mathit{ptr}$ points to,
    to figure out whether it can still operate on expeditive mode. In the scenario above, 
    $nd$ will be the node that $t'$ has allocated to replace $\ell$, and thus it has its 
    $\mathit{helpersExist}$ flag equal to \True\ (line~\ref{alg:tree:split:new-internal-node}). 
    This way, $t$ discovers that the execution mode for $\mathit{l}$ has changed
    (line~\ref{alg:tree:st:finish} and first condition of line~\ref{alg:tree:op-is-applied}),
    and re-attempts its \Insert\ (line~\ref{alg:tree:re-attempt}).
        
    \begin{lemma}
    \label{lem:tree}
    Algor.~\ref{alg:tree} is a {\em linearizable, lock-free} implementation of a leaf-oriented
    tree with fat leaves, supporting only insert operations.
    \end{lemma}

    %%%%%%%%%%%%%%%%%%%%%%%%%%%%% Refinement %%%%%%%%%%%%%%%%%%%%%%%%%%%%%%%%%%%
    \subsection{Refinement}

    To implement \RS, \Fresh\ uses a set of priorities queues
    each implemented using an array ( Algorithm~\ref{alg:pq}). 
    A thread inserts elements in all arrays in a round-robin fashion. 
    This technique results in almost equally-sized arrays, which is crucial
    for achieving load-balancing. 

    %%%%%%%%%%%%%%%%%%%%%%%%%%% Sorted Arrays Code %%%%%%%%%%%%%%%%%%%%%%%%%%%%%%%%%%%%%%%%%%%%%%%%%%%%%
    
    \begin{algorithm}[t]
        \footnotesize
        \caption{Priority Queue of \Fresh. Code for thread $t$.}
        \label{alg:pq}
        \begin{algorithmic}[1]
    
            \State \textbf{Shared variables:}
            \State $\mathit{\langle int, Data \rangle}$ $A[0..k-1]$, initially all $\langle \bot, \bot \rangle$
            \State CounterObject $\mathit{Cnt}$, initially $0$
            \State Boolean $\mathit{helpersExist}$, initially $\False$
            \State int $\mathit{insPos}$, initially $0$
            \State $\mathit{\langle int, Data \rangle}$ *$\mathit{SA}$, initially $\NULL$
    
            \vspace{2mm}
            
            \Procedure{Insert}{int $\mathit{priority},Data \mathit{values}$}
                \State int $\mathit{pos} \gets \FAI(\mathit{insPos})$ \label{alg:pq:ins:FAI}
                \State $\mathit{A[pos]} \gets \langle \mathit{priority, value} \rangle$ \label{alg:pq:ins:store}
            \EndProcedure
    
            \vspace{2mm}
    
            \Procedure{InitDeletePhase}{}
                \State $\mathit{\langle int, Data \rangle}$ *$\mathit{sa}$
                \State $\mathit{sa} \gets$ allocate local array of $insPos$ elements
                \State Copy non-$\bot$ elements of $A$ into $\mathit{sa}$ and sort them \label{alg:sa:local-copy}
                \State $\mathit{\CAS(\&SA, \NULL, sa)}$ \label{alg:sa:CAS}
            \EndProcedure
    
            \vspace{2mm}
    
            \Function{DeleteMin}{boolean $ \mathit{isHelper}$} \textbf{returns} $\mathit{Data}$
                \If{$\mathit{isHelper = \True}$ \textbf{and} $\mathit{helpersExist = \False}$}
                    \State $\mathit{helpersExist} \gets \True$
                \EndIf
                \State int $\mathit{pos} \gets \mathit{Cnt.\NextIndex(\&helpersExist})$ \label{alg:pq:deleteMin:next}
                \If{$\mathit{pos} \geq insPos$}
                    \State \Return $\bot$
                \EndIf
                \State \Return $\mathit{SA[pos].data}$
            \EndFunction
    
        \end{algorithmic}
    \end{algorithm}
    
    
    
    During query answering, an application may use the same index tree
    to answer more than one query. In that case, the done flags of the nodes and other
    variables need to be reset each time a new query starts. This may require syncrhonization. 
    To avoid this case, \Fresh\ implements the {\em done} flag as a counter (rather than as a boolean). 
    This counter describes the number of queries for which the node has been processed.
    
    To implement \RS.\Traverse, \Fresh\ first comes up with sorted versions of the arrays,
    shared to all threads. Then, it uses \Refresh\ to assign sorted arrays to threads 
    for processing. To process the elements of a sorted array $SA$, \Refresh\ is
    applied recursively. 
    %Specifically, the threads working on $SA$ 
    %use a counter object to get assigned elements of $SA$.
    Processing of an array element is performed by invoking the \Refinement\ function
    (Algorithm~\ref{alg:iSAXTraverse}). Helping is done at the level of 
    1. each individual priority queue and 
    2. the set of priority queues, in a way similar to that in \PS.
    \RS\ is a linearizable lock-free implementation of a traverse object. 
    %As soon as $t$ discovers that all elements of $PQ$ have been assigned for processing, 
    %it re-traverses the nodes of $PQ$,
    %checking their done flags to figure out whether there are elements
    %whose processing has not been completed yet, and helps
    %whenever it is needed. Similarly, when $t$ figures out that all subtrees have been
    %assigned for processing, it examines the done flags of the subtrees to figure out
    %whether there are subtrees whose processing has not yet been completed, and it helps if needed. 
    %Pseudocode for implementing \Put\ and \Traverse\ of \RS\ resemles that of \PS\ and 
    %is omitted.
    
    To update BSF, \Fresh\ repeatedly reads the current value $y$ of BSF, and attempts to
    atomically change it from $v$ to the new value $v'$, using \CAS, until it either succeeds or some value 
    smaller than or equal to $v'$ is written in BSF.
    
    %When an invocation of \Traverse(\&$\mathit{\Prunning, *, \True}$)\ by a thread $t$ on \RS\ completes, 
    %for every leaf $\ell$ in $\RS$, $\ell$ either has been processed or it has been pruned.
    %By all the claims we present above:
    
    \begin{lemma}
    \label{lem:rs}
    {\bf (1)} \RS\ is a linearizable lock-free implementation of a traverse object that supports \Put\ and
    \Traverse($\mathit{*, *, 0}$). 
    {\bf (2)} For every thread $t$, when an invocation of \Traverse(\&$\mathit{\Prunning, *, \True}$)\ by $t$ on \RS\ completes, 
    for every leaf $\ell$ in $\RS$, $\ell$ either has been processed or it has been pruned.
    \end{lemma}
    
    \begin{theorem}
    \label{thm:qa}
    \Fresh\ solves the 1-NN problem and provides a lock-free implementation of \QueryAnswering\
    (Alg.~\ref{alg:iSAXTraverse}). 
    \end{theorem}
    
\subsection{FreSh with Dynamic Batches}


\begin{algorithm}[t]
    \footnotesize
    \caption{Type Definitions for TreeInsertion with Dynamic Batches}
    \label{alg:tree-types-dynamic}
    \begin{algorithmic}[1]
        \Statex \textbf{Type Definitions:}
        \State \textbf{Type} Node:
        \State \quad int $\mathit{key}$
        \State \quad $\{Node,Leaf\}$ *$\mathit{left}$
        \State \quad $\{Node,Leaf\}$ *$\mathit{right}$
        \State \quad InsertRec $\mathit{Announce}[0..n-1]$
        \State \quad \textbf{Int} $\mathit{helpersExist}$

        \Statex
        \State \textbf{Type} InsertRec:
        \State \quad Data $\mathit{data}$
        \State \quad int $\mathit{position}$

        \Statex
        \State \textbf{Type} Leaf \textbf{extends} Node:
        \State \quad DynamicData $D[0..m-1]$
        \State \quad CounterObject Elements

        \Statex
        \State \textbf{Type} DynamicData \textbf{extends} Data:
        \State \quad int $\mathit{seq}$

    \end{algorithmic}
\end{algorithm}


\begin{algorithm}[t]
    \footnotesize
    \caption{DTraverseTree: a lock-free leaf-oriented tree with fat leaves, implementing a traverse object
    for dynamic batches. Code for thread $t \in \{0, \ldots, n-1\}$.}
    \label{alg:tree:dynamic}
    \begin{algorithmic}[1]

        \State \textbf{Shared variables:}
        \State \{Node,Leaf\} *$Tree$ :=  $\NULL$, initially pointing to a Leaf:
        \Statex \quad $\langle \mathit{key}, \NULL, \NULL, \langle \langle \bot, \bot \rangle, \ldots, \langle \bot, \bot \rangle \rangle,\False, \langle \bot, \ldots, \bot \rangle, \bot, 0 \rangle$

        \Function{TreeInsert}{$\mathit{data}, \mathit{isHelper}, \mathit{mode}, \mathit{batchID}$}
            \State Leaf *$\mathit{leaf}$
            \State \{Node,Leaf\} *$\mathit{parent}$ := $Tree$, **$ptr$
            \State int $\mathit{pos}, \mathit{val}$
            \State Boolean $\mathit{expeditive}$ \label{alg:tree:isHelper_dynamic}

            \While{\True} \label{alg:tree:while_dynamic}
                \State $\langle \mathit{leaf, parent} \rangle $ := $ \Search(\mathit{data, parent})$ \label{alg:tree:search:dynamic}
                \If{$\mathit{parent} = \NULL$} 
                    \State $ptr $ := $ \&Tree$ \label{alg:tree:ptr-null}
                \ElsIf{$\mathit{parent \rightarrow left = leaf}$} 
                    \State $ptr $ := $ \&\mathit{parent \rightarrow left}$
                \Else 
                    \State $ptr $ := $ \&\mathit{parent \rightarrow right}$ \label{alg:tree:ptr-right_dynamic}
                \EndIf

                \bl{
                \If{$\mathit{isHelper} = \True$ \textbf{and} $\mathit{leaf \rightarrow helpersExist \leq currentUpdate}$} \label{alg:tree:require-switch-mode_dynamic}
                    \State \Int\ tmp := $ \mathit{leaf \rightarrow helpersExist}$
                    \If{$\mathit{tmp == leaf \rightarrow helpersExist}$ \textbf{and} $\mathit{tmp \leq currentUpdate}$}
                        \State $\mathit{CAS(\&leaf \rightarrow helpersExist, currUpdate, currUpdate+1)}$
                    \EndIf
                \EndIf
                }

                \State $\mathit{\langle pos, expeditive \rangle} $ := $ \mathit{Elements.NextIndex(\&leaf \rightarrow helpersExist)}$ \label{alg:tree:object-pos_dynamic}
                
                \If{$\mathit{expeditive} = \False$} 
                    \State $\mathit{leaf \rightarrow Announce[t]} $ := $ \langle \mathit{data, \bot} \rangle$ \label{alg:tree:announce-op_dynamic}
                \EndIf

                \If{$\mathit{pos < M}$} \label{alg:tree:pos-in-D_dynamic}
                    
                    \If{$\mathit{expeditive = \False}$}
                        \State $\mathit{leaf \rightarrow Announce[t].position} $ := $ \mathit{pos}$ \label{alg:tree:pos-in-D:update_dynamic}
                    \EndIf
                    \If{\br{ $\mathit{Elements.GetIndex} \leq M$ \textbf{and} $\mathit{mode} = \mathit{LOGICAL}$}}
                        \State \br{$\mathit{CAS(\&leaf \rightarrow D[pos].seq, \bot, MARKED)}$} \label{alg:tree:store-marked}
                    \EndIf
                    \State $\mathit{leaf \rightarrow D[pos]} $ := $ \mathit{data}$ \label{alg:tree:store-in-D_dynamic}

                    
                    \If{\br{$\mathit{mode == LOGICAL}$}} \label{alg:tree:store-timestamp-start}
                        \If{\br{$\mathit{leaf \rightarrow D[pos].seq == \bot }$}} \label{alg:tree:store-timestamp:opt}
                            \State \br{\Int\ currSeq := GlobalSeq}
                            \State \br{$\mathit{CAS(\&leaf \rightarrow D[pos].seq, \bot, currSeq)}$} \label{alg:tree:store-timestamp-CAS}
                        \EndIf
                    
                    \ElsIf{\tl{$\mathit{mode == SYSTEM}$}}
                        \State \tl{$\mathit{leaf \rightarrow D[pos].seq} $ := $ \mathit{timeStamp[batchID]}$}
                    \EndIf \label{alg:tree:store-timestamp-end}
                \Else
                    \State \SplitLeaf($\mathit{leaf}, \mathit{ptr}, \mathit{expeditive}$) \label{alg:tree:pos-not-in-D_dynamic}
                \EndIf

                
                \If{\bl{$($*$ptr$) $\rightarrow helpersExist = currentUpdate+1$}} \label{alg:tree:st:finish_dynamic}
                    \If{$expeditive = \False$ \textbf{and} $($*$ptr$) $\rightarrow Announce[t].position \neq \bot$} \label{alg:tree:op-is-applied_dynamic}
                        \State $($*$ptr$) $\rightarrow Announce[t] $ := $ \langle \bot, \bot \rangle$ \label{alg:tree:clean_dynamic}
                    \Else
                        \State \textbf{continue} \label{alg:tree:re-attempt_dynamic}
                    \EndIf
                \EndIf

                \Return
            \EndWhile
        \EndFunction

    \end{algorithmic}
\end{algorithm}

%%%%%%%%%%%%%%%%%%%%%%%%%%%%%%%%%%%%%%%%%%%%%%%%%%%%%%%%%%%%%%%%%%%%%%%%%%%%%%%%%%

\begin{algorithm}[t]
    \footnotesize
    \begin{algorithmic}[1]
        \Procedure{SplitLeaf}{$\mathit{leaf}$, $\mathit{prt}$, $\mathit{expeditive}$, $\mathit{mode}$}
            \State $\mathit{newNode} $ :=  \textbf{new} Node initialized with $\langle \bot, \NULL,
             \NULL, \langle \langle \bot, \bot \rangle, \ldots, \langle \bot, \bot \rangle \rangle, \textbf{not} \mathit{expeditive} \rangle$ \label{alg:tree:split:new-internal-node-dynamic}
            \State $\mathit{newNode \rightarrow left}$ := \textbf{new} Leaf initialized with 
            \Statex $\langle \bot, \NULL, \NULL, \langle \langle \bot, \bot \rangle, \ldots, \langle \bot, \bot \rangle \rangle,\False, \langle \bot, \ldots, \bot \rangle, \bot, 0 \rangle$ \label{alg:tree:split:new-lc-dynamic}
            \State $\mathit{newNode \rightarrow right}$ := \textbf{new} Leaf initialized similarly \label{alg:tree:split:new-rc-dynamic}
            \State $\mathit{splitBuffer} $ := $\emptyset$
            
            \If{$\mathit{expeditive = \False}$} \label{alg:tree:em:Announce-scan-dynamic}
                \For{$i \in \{0, \ldots, n-1\}$ \textbf{with} $\mathit{leaf \rightarrow Announce[i].data} \neq \bot$} \label{alg:tree:Announce-scan-dynamic}
                    \State $\mathit{ldata} $ := $ \mathit{leaf \rightarrow Announce[i].data}$
                    \If{$\mathit{leaf \rightarrow Announce[i].position} \neq \bot$} \label{alg:tree:announce-scan:pos-not-bot-dynamic}
                        \State $\mathit{pos} $ := $ \mathit{leaf \rightarrow Announce[i].position}$
                        \State $\mathit{leaf \rightarrow D[\mathit{pos}].data} $ := $ \mathit{ldata}$ \label{alg:tree:announce-scan:store-in-D-dynamic}
                        
                        \If{\br{$\mathit{mode == LOGICAL}$}}
                            \If{\br{$\mathit{leaf \rightarrow D[\mathit{pos}].seq == MARKED}$}}
                                \State \br{$\mathit{CAS(\&leaf \rightarrow D[\mathit{pos}].seq, MARKED, \bot)}$}
                            \Else
                                \State \br{$\mathit{tmstamp} $ := $ \mathit{TIMESTAMP}$}
                                \If{\br{$\mathit{leaf \rightarrow D[\mathit{pos}].seq == \bot}$}}
                                    \State \br{$\mathit{CAS(\&leaf \rightarrow D[\mathit{pos}].seq, \bot,tmstamp )}$}
                                \EndIf
                            \EndIf
                        \ElsIf{\tl{$\mathit{mode == SYSTEM}$}}
                            \State \tl{$\mathit{tmstamp} $ := $ \mathit{timeStamp[batchID]}$}
                            \State \tl{$\mathit{leaf \rightarrow D[\mathit{pos}].seq} $ := $ \mathit{tmstamp}$}
                        \EndIf
                    \Else
                        \If{\br{$\mathit{mode == LOGICAL}$}}
                            \State \br{\textbf{Add} $(\mathit{ldata,\bot})$ \textbf{to} $\mathit{splitBuffer}$}
                        \Else
                            \State \tl{\textbf{Add} $(\mathit{ldata,timeStamp[batchID]})$ \textbf{to} $\mathit{splitBuffer}$}
                        \EndIf
                    \EndIf \label{alg:tree:announce-scan:data-copy-dynamic}
                    \State $\mathit{newNode \rightarrow Announce[i]} $ := $ \langle \mathit{ldata}, -1 \rangle$ \label{alg:tree:announce-scan:mark-op-applied-dynamic}
                \EndFor
            \EndIf \label{alg:tree-split-set-timestamp-start}
            
            \For{\textbf{each} element in $\mathit{leaf \rightarrow D[i]} \cup \mathit{splitBuffer}$}
                \If{\bl{$\mathit{element.seq == \bot\ \And\ element \in D}$}}
                    \If{\br{$\mathit{mode == LOGICAL}$}}
                        \If{\br{$\mathit{leaf \rightarrow D[\mathit{pos}].seq == MARKED}$}}
                            \State \br{$\mathit{CAS(\&leaf \rightarrow D[\mathit{pos}].seq, MARKED, \bot)}$}
                        \Else
                            \State \br{$\mathit{tmstamp} $ := $ \mathit{TIMESTAMP}$}
                            \If{\br{$\mathit{leaf \rightarrow D[\mathit{pos}].seq == \bot}$}}
                                \State \br{$\mathit{CAS(\&leaf \rightarrow D[\mathit{pos}].seq, \bot,tmstamp )}$}
                            \EndIf
                        \EndIf
                    \ElsIf{\tl{$\mathit{mode == SYSTEM}$}}
                        \State \tl{$\mathit{tmstamp} $ := $ \mathit{timeStamp[batchID]}$}
                        \State \tl{$\mathit{leaf \rightarrow D[\mathit{pos}].seq} $ := $ \mathit{tmstamp}$}
                    \EndIf
                \EndIf
                \State \textbf{Distribute} the element into $\mathit{newNode \rightarrow left}$ \textbf{or} $\mathit{newNode \rightarrow right}$
                \Statex   \textbf{based on its key}
            \EndFor \label{alg:tree-split-set-timestamp-end}
            
            \State \textbf{Fix the key of} $\mathit{newNode}$ \Comment{may result in further leaf splits} \label{alg:tree:split-dynamic}
            \State $\CAS(*\mathit{ptr}, \mathit{leaf}, \mathit{newNode})$ \label{alg:tree:CAS-dynamic}
            
            \If{\br{$\mathit{mode == LOGICAL}$}}
                \State \br{\textbf{Traverse} $\mathit{ptr \rightarrow left}$ \textbf{and} $\mathit{ptr \rightarrow right}$ 
                \Statex \br{\textbf{and for each element with} $\mathit{element.seq == \bot}$}}
                \State \br{\hspace{1em} $\mathit{tmstamp} $ := $ \mathit{TIMESTAMP}$}
                \State \br{\hspace{1em} $\mathit{CAS(\&element.pos.seq, \bot,tmstamp )}$}
            \EndIf
        \EndProcedure
    \end{algorithmic}
    \caption{SplitLeaf with batches: a lock-free split operation for a leaf-oriented tree with fat leaves.
        Code for thread $t \in \{0, \ldots, n-1\}$.}
    \label{alg:tree-split-dynamic}
\end{algorithm}

%%%%%%%%%%%%%%%%%%%%% Priority Queues %%%%%%%%%%%%%%%%%%%%%%%%%%%%%%%%%%%%%%%%%%%%%%%%%%%%%%%%%%%%%

\begin{algorithm}[t]
    \footnotesize
    \begin{algorithmic}[1]
        \State{Code for thread $p \in \{0, \ldots, n-1\}$}
        \State \textbf{Shared variables:}
        \State $\mathit{\langle int, Data \rangle}$ $A[0..k-1][0...n-1]$, initially all $\langle \bot, \bot \rangle$
        \State CounterObject Cnt, initially $0$
        \State Boolean $\mathit{helpersExist}$, initially \False
        \State int $\mathit{insPos}$, initially $0$
        \State $\mathit{\langle int, Data \rangle}$ *$\mathit{SA}$, initially \NULL
        \State int $\mathit{arrayID}$, initially $p$

        \Procedure{Insert}{$\mathit{priority}$, $\mathit{values}$}
            \State Append $\mathit{values}$ at $\mathit{A[arrayID][p]}$ \label{alg:pq:ins:store:dynamic}
            \State $\mathit{arrayID := (arrayID+1) \mod k}$
        \EndProcedure

        \Procedure{InitDeletePhase}{}
            \State $\mathit{\langle int, Data \rangle}$ *$\mathit{sa}$
            \For{\textbf{each} $\mathit{array}$ in $\mathit{A[k]}$}
                \If{$\mathit{array}$ has valid data}
                    \State $\mathit{insPos}$ := $\mathit{insPos}$ + array.size
                \EndIf
            \EndFor
            \State $\mathit{sa}$ := allocate local array of $insPos$ elements
            \State copy into $\mathit{sa}$ the non-$\bot$ elements of $A$ and sort them \label{alg:sa:local-copy-dynamic}
            \State $\mathit{CAS(\&SA, \NULL, sa)}$ \label{alg:sa:CAS-dynamic}
        \EndProcedure

        \Procedure{DeleteMin}{Boolean $\mathit{isHelper}$, int $\mathit{CurrentUpdate}$} \textbf{returns} $\mathit{Data}$
            \If{$\mathit{isHelper = \True}$ \textbf{and} $\mathit{helpersExist} \leq \mathit{CurrentUpdate}$}
                \State int tmp := $\mathit{helpersExist}$
                \If{$\mathit{tmp == helpersExist}$ \textbf{and} $\mathit{tmp} \leq \mathit{CurrentUpdate}$}
                    \State $\mathit{CAS(\& helpersExist, CurrentUpdate, CurrentUpdate+1)}$
                \EndIf
            \EndIf
            \State int $\mathit{pos} := \mathit{Cnt.\NextIndex(\&helpersExist})$ \label{alg:pq:ins:FAI-dynamic}
            \If{$\mathit{pos} \geq \mathit{insPos}$}
                \State \Return $\bot$
            \EndIf
            \If{$\mathit{mode == LOGICAL}$}
                \If{$\mathit{SA[pos].data.seq == MARKED}$}
                    \State \Return $\bot$
                \EndIf            
                \If{$\mathit{SA[pos].data.seq == \bot}$}
                    \State int $\mathit{currSeq := GlobalSeq}$
                    \State $\mathit{CAS(\&SA[pos].data.seq, \bot, currSeq)}$ \label{alg:pq:CAS:Dynamic}
                \EndIf
            \EndIf
            \State \Return $\mathit{SA[pos].data}$
        \EndProcedure
    \end{algorithmic}
    \caption{Priority queue of \Fresh with batches.}
    \label{alg:pq-dynamic}
\end{algorithm}

As discussed in previous chapters, we designed and implemented an extension of FreSh,
called \textit{DFreSh}, that supports insertions of dynamic batches while ensuring that
queries always return the best possible answer and respect linearizability.
In the pseudocode provided, we refer to this version as \textit{LOGICAL}.
This approach is completely lock-free, meaning that neither index workers nor query workers
need to wait for each other as long as they have available work. For example, index workers
can continue inserting batches while query workers process queries concurrently.  
% 
To ensure correctness, we need a mechanism that guarantees a query with a
specific timestamp always returns the same result. To achieve this, we introduce 
a sequence number mechanism. Specifically, we maintain a global shared counter,
\textit{GlobalSeq}, which is accessible to both index and query workers.
Query workers read the current value of \textit{GlobalSeq} and attempt to increment it
by one before answering a query. A query must be able to see all data series inserted
into the index with a timestamp smaller than or equal to its own before it is answered.  
%
The insertion process for index workers follows the same general approach as in FreSh,
using the \texttt{TreeInsert} method (see Algorithm~\ref{alg:tree:dynamic}). However,
the iSAX tree structure in this dynamic setting differs from that of FreSh.
Each leaf node now contains an array \( D \), where each element consists of
a sequence number, an iSAX summary, and a pointer to the original data series stored in
\texttt{RawData}. Despite these differences, the core insertion procedure remains
similar to what was described in Chapter~\ref{chapter:FreSh}.  
An index worker first locates the appropriate leaf to insert a new element
(line~\ref{alg:tree:search:dynamic}). It then attempts to acquire a position in the leaf
node. If space is available (line~\ref{alg:tree:pos-in-D_dynamic}), the data series is
stored as before. However, for an element to be considered valid, it must first be assigned
a sequence number, which is done using a Compare-And-Swap (CAS) operation
(line~\ref{alg:tree:store-timestamp-CAS}).  
Since queries are processed concurrently with insertions, a query worker may
encounter elements in a candidate leaf during the \textit{Refinement} stage. For each
element in the leaf, the query worker first checks its sequence number. If the sequence
number is smaller than or equal to the query's timestamp, the element must be considered
in the query result. If it has a larger value, the query worker skips it. However, if the
sequence number is unassigned (denoted by \( \bot \)), the query worker cannot ignore
the element and must help assign a sequence number using a CAS operation
(line~\ref{alg:pq:CAS:Dynamic}). Since both index and query workers can assign sequence
numbers, the CAS operation ensures correctness while the preceding conditional
check optimizes performance by avoiding unnecessary failed CAS attempts.
%
On the other hand, if the leaf is full, the index worker attempts to perform a split.  
During the split operation, the index worker first examines the announce array
for any operations that have been announced but not yet placed into the leaf.
If a thread has announced both the data and its position inside the leaf,
the thread performing the split rewrites the data into that position,
as was done in FreSh. Additionally, if the sequence number is unassigned,
the worker helps assign one. However, if the position has not yet been announced,
the data is temporarily stored in a buffer called \textit{splitBuffer}.  
%
Next, the index worker distributes the elements stored in \textit{splitBuffer},
along with those already in the leaf, to the new left and right child nodes
based on their iSAX summaries. During this process, for elements originally
from the leaf, the worker checks their sequence numbers and assigns one
if necessary. This step is crucial because the leaf remains connected to the
tree, making it visible to query workers, which may have included it in their
query results.  
%
Once all elements are distributed, the worker attempts to finalize the split
using a \CAS operation, following the same approach as in FreSh. However,
to fully complete the split, a traversal of the newly created child nodes
is required. This is because elements moved from the \textit{splitBuffer}
may still have unassigned sequence numbers. Importantly, this traversal must
not be performed before the split is established, as announce arrays are not
visible to query workers. If these elements receive a sequence number smaller
than or equal to that of an ongoing query before the split is finalized,
it could lead to correctness violations.
% 

There is one final scenario that can occur during splitting, which we address
using the keyword \textit{MARKED}. Consider the following case: an index worker
announces its data and acquires a position in the leaf but becomes slow before
announcing the position. Meanwhile, another index worker starts a split operation.
Since the first worker has not yet announced its position, its operation is treated
as incomplete and stored in the \textit{splitBuffer}.  
%
While the split is ongoing, the first worker resumes execution and inserts its element
into the leaf. At this point, the element becomes visible to queries and may receive
a sequence number-either from the inserting thread or from a query worker. This can
lead to a correctness violation because the same element might be assigned different
sequence numbers before and after the delay. If the thread performing the split has
already processed the position where the first worker is placing the element,
it will not include it, meaning the element will only receive a sequence number
when the new child nodes are traversed after the split. If this element happens
to be the best answer for an ongoing query, the query may or may not include it in
its results, violating correctness.  
%
To prevent this issue, we introduce a marking technique. Before storing data into 
the leaf, a thread first checks whether a split is in progress by reading the value
of \textit{NextIndex}. If a split is occurring, the thread marks its position
by setting the sequence number to \textit{MARKED} using a CAS operation before
inserting the data. A query encountering a \textit{MARKED} position treats it as 
invalid. During the split operation, all marked positions are reset to \( \bot \), 
ensuring that these elements receive a sequence number either from a query worker
or when processing the new child nodes after the split.


\chapter{Evaluation}
\label{chapter:Evaluation}

\section{Evaluation of static FreSh}

\noindent{\bf Setup.}
% 
We used a machine equipped with 2 Intel  Xeon E5-2650 v4 2.2GHz
CPUs with 12 cores each, and 30MB L3 cache. The machine runs
Ubuntu Linux 16.04.7. LTS and has 256GB of RAM. Code is written in C and
compiled using gcc v11.2.1 with O2 optimizations.

\noindent{\bf Datasets.}
We evaluated \Fresh\ and the competing algorithms (all algorithms are in-memory) using both real and synthetic datasets.
The synthetic data series, \emph{Random}, are generated as random-walks (i.e., cumulative sums) of 
steps that follow a Gaussian distribution (0,1).
This type of data has been extensively used
~\cite{conf/sigmod/Faloutsos1994,isax2plus,conf/kdd/Zoumpatianos2015,DBLP:journals/vldb/ZoumpatianosLIP18,DBLP:journals/pvldb/EchihabiZPB18,DBLP:journals/pvldb/EchihabiZPB19}, 
and models the distribution of stock market prices~\cite{conf/sigmod/Faloutsos1994}.
Our real datasets come from the domains of seismology and astronomy.
The seismic dataset, \emph{Seismic}, was obtained from the IRIS Seismic Data Access
archive~\cite{url/data/seismic}. It contains seismic instrument recordings from
thousands of stations worldwide and consists of 100 million data series of size 256,
i.e. its size is 100GB. 
The astronomy dataset, \emph{Astro}, represents celestial objects and was obtained 
from~\cite{journal/aa/soldi2014}. The dataset consists of 270 million data series of
size 256, i.e. its size is 265GB.
% 
Since the main memory of our machine is limited to 256GB, we only use the first 200GB
of the Astro dataset in our experiments.

\begin{table}[htbp]
    \centering
    \renewcommand{\arraystretch}{1.2} % Adjust row height for better readability
    \begin{tabular}{||c | c | c | c | c||} 
        \hline
        \textbf{Dataset} & \textbf{Data} & \textbf{Length} & \textbf{Size} & \textbf{Description}  \\ 
        & \textbf{Series} & \textbf{(floats)} & \textbf{(GB)} & \\ [0.5ex]
        \hline\hline
        Seismic & 100M & 256 & 100 & seismic records  \\ 
        \hline
        Astro & 270M & 256 & 265 & astronomical data \\ 
        \hline
        Random & 100M & 256 & 100 & random walks  \\ 
        \hline
    \end{tabular}
    \caption{Details of datasets used in experiments.}
    \label{table:datasets}
\end{table}
\clearpage
\noindent{\bf Evaluation Measures.}
% 
We measure (i) the {\em summarization time} required to calculate
the iSAX summaries and fill-in the summarization buffers, 
(ii) the {\em tree time} required to insert the items of the receive buffers in the
tree-index and (iii) the {\em query answering time} required to answer 100 queries
that are not part of the dataset. 
The  sum of the above times constitute the {\em total time}.
Experiments are repeated $5$ times and averages are reported.
All algorithms return exact results.

\subsection{Results}

\noindent{\bf \Fresh\ vs \MESSI.}
We compare \Fresh\ against \MESSI, which is the state-of-the-art blocking
in-memory data series index.To enable a fair comparison, we use an 
optimized version of the original \MESSI\ implementation,
where we have applied all the code enhancements incorporated by \Fresh.
% 
We also compare \Fresh\ against an extended version of MESSI, called \MESSIenh
, that allows several threads to concurrently populate the same sub-tree, during tree creation
(instead of using a single thread for each subtree).
This is implemented using fine-grained locks that are attached on each leaf node of a subtree.
\MESSIenh\ allows to compare the lock-free index creation phase of \Fresh\ against a more
efficient blocking one than that of original \MESSI.

Figure~\ref{fig:eval:fresh-messi-threads:random} shows that all algorithms
(\Fresh, \MESSI, and \MESSIenh) continue scaling as the number of threads is increasing,
for Seismic 100GB. This is true for all three phases. 
Moreover, the total execution time of \Fresh\ 
(Figure~\ref{fig:eval:fresh-messi-threads:random:total-from-4})
is almost the same as the total execution time of all its competitors, although it is the
only lock-free approach. 
As expected, the tree index creation time of \Fresh\ is smaller than \MESSI's 
(Figure~\ref{fig:eval:fresh-messi-threads:random:tree-from-4}), since \Fresh\ allows
subtrees to be populated concurrently by multiple threads, allowing parallelism during
this phase, in contrast to \MESSI. 
Interestingly, \Fresh\ achieves better performance than \MESSIenh, in most cases.
The results for Seismic are similar and are omitted for brevity.
Considering scalability as the size of the dataset increases, 
Figure~\ref{fig:eval:scale-dataset:random} demonstrates that \Fresh\ scales well
on all three datasets. In most cases, \Fresh\ is faster than \MESSI.
Following previous works~\cite{DBLP:journals/vldb/ZoumpatianosLIP18,PFP21-I},
we also conducted experiments with query workloads of increasing difficulty.
For these workloads, we select series at random from the collection, add to each point
Gaussian noise ($\mu = 0$, $sigma = 0.01-0.1$), and use these as our queries. 
Figure~\ref{fig:eval:scale-query-difficulty:seismic:total} presents the results for
the Seismic dataset, where \Fresh\ performs better than \MESSI\ in most cases.


%%%%%%%%%%%%%%%%%%%%%%%%% Threds Scalability FreSh-MESSI-MESSIenh RANDOM%%%%%%%%%%%%%%%%%%%%%%%%%%
\begin{figure}[htbp]
    \centering
    \begin{subfigure}{0.45\textwidth}
        \includegraphics[width=\textwidth]{figures/Experiments/fresh-messi-threads-random-total}
        \caption{Total}
        \label{fig:eval:fresh-messi-threads:random:total-from-4}
    \end{subfigure}    
    \begin{subfigure}{0.45\textwidth}
        \includegraphics[width=\textwidth]{figures/Experiments/fresh-messi-threads-random-summarization}
        \caption{Summarization}
        \label{fig:eval:fresh-messi-threads:random:recbuf-from-4}
    \end{subfigure}    

    \vspace{0.2cm} % Space between rows

    \begin{subfigure}{0.45\textwidth}
        \includegraphics[width=\textwidth]{figures/Experiments/fresh-messi-threads-random-tree}
        \caption{Tree index}
        \label{fig:eval:fresh-messi-threads:random:tree-from-4}
    \end{subfigure}    
    \begin{subfigure}{0.45\textwidth}
        \includegraphics[width=\textwidth]{figures/Experiments/fresh-messi-threads-random-query}
        \caption{Query answering}
        \label{fig:eval:fresh-messi-threads:random:queries-from-4}
    \end{subfigure}                

    \caption{Comparison of \Fresh\ against \MESSI\ and \MESSIenh\ on 100GB Random.}
    \label{fig:eval:fresh-messi-threads:random}
\end{figure}

%%%%%%%%%%%%%%%%%%%%%%%% DATASET SIZE: FRESH - MESSI RANDOM  %%%%%%%%%%%%%%%%%%%%%%%%%%%%

\begin{figure}[htbp]
    \centering
    \begin{subfigure}{0.45\textwidth} 
        \includegraphics[width=\textwidth]{figures/Experiments/scale-dataset-random-total.png}
        \caption{Total}
        \label{fig:eval:scale-dataset:random:total}
    \end{subfigure}    
    \begin{subfigure}{0.45\textwidth}
        \includegraphics[width=\textwidth]{figures/Experiments/scale-dataset-random-summarization.png}
        \caption{Summarization}
        \label{fig:eval:scale-dataset:random:summarization}
    \end{subfigure}    

    \vspace{0.2cm} % Space between rows

    \begin{subfigure}{0.45\textwidth}
        \includegraphics[width=\textwidth]{figures/Experiments/scale-dataset-random-tree.png}
        \caption{Tree index}
        \label{fig:eval:scale-dataset:random:tree-index}
    \end{subfigure}    
    \begin{subfigure}{0.45\textwidth}
        \includegraphics[width=\textwidth]{figures/Experiments/scale-dataset-random-query.png}
        \caption{Query answering}
        \label{fig:eval:scale-dataset:random:query-answering}
    \end{subfigure}                

    \caption{Comparison of \Fresh\ against \MESSI on the Random dataset for 24 threads.}
    \label{fig:eval:scale-dataset:random}
\end{figure}

%%%%%%%%%%%%%%%%%%%%%%%%%%%% REAL DATASETS: DATASET SIZE %%%%%%%%%%%%%%%%%%%%%%%%%%%%%%%%%%%%%%%%%%%%%%%%%%%%%%s

\begin{figure}[htbp]
    \centering
    \begin{subfigure}{0.45\textwidth}
        \includegraphics[width=\textwidth]{figures/Experiments/scale-dataset-seismic-total.png}
        \caption{Seismic - Total}
        \label{fig:eval:scale-dataset:seismic:total}
    \end{subfigure}
    \begin{subfigure}{0.45\textwidth}
        \includegraphics[width=\textwidth]{figures/Experiments/scale-dataset-seismic-query.png}
        \caption{Seismic - Query answering}
        \label{fig:eval:scale-dataset:seismic:QueryAnswering}
    \end{subfigure}        
    \begin{subfigure}{0.45\textwidth}
        \includegraphics[width=\textwidth]{figures/Experiments/scale-dataset-astro-total.png}
        \caption{Astro - Total}
        \label{fig:eval:scale-dataset:astro:total}
    \end{subfigure}
    \begin{subfigure}{0.45\textwidth}
        \includegraphics[width=\textwidth]{figures/Experiments/scale-dataset-astro-query.png}
        \caption{Astro - Query answering}
        \label{fig:eval:scale-dataset:astro:QueryAnswering}
    \end{subfigure}    

    \caption{Comparison of \Fresh\ against \MESSI\ on (a) Seismic and (b) Astro datasets for 24 threads.}
    \label{fig:eval:scale-dataset:real}
\end{figure}

%%%%%%%%%%%%%%%%%%%%%%%%%%%%%%%%%%%%%%%%%%%%%%%%%%%%%%%%%%%%%%%%%%%%%%%%%%%%%%%

\begin{figure*}[htbp]
    \centering
    \begin{subfigure}{0.45\textwidth}
        \includegraphics[width=\textwidth]{figures/Experiments/scale-query-difficulty-seismic-total.png}
        \caption{Seismic - Total (Query Difficulty)}
        \label{fig:eval:scale-query-difficulty:seismic:total}
    \end{subfigure}    
    \begin{subfigure}{0.45\textwidth}
        \includegraphics[width=\textwidth]{figures/Experiments/scale-dataset-tree-index-random.png}
        \caption{Tree Index - Random}
        \label{fig:eval:scale-dataset:tree-index:random}
    \end{subfigure}    

    \begin{subfigure}{0.45\textwidth}
        \includegraphics[width=\textwidth]{figures/Experiments/scale-dataset-tree-index-seismic.png}
        \caption{Tree Index - Seismic}
        \label{fig:eval:scale-dataset:tree-index:seismic}
    \end{subfigure}    
    % \begin{subfigure}{0.45\textwidth}
    %     \includegraphics[width=\textwidth]{figures/Experiments/baselines-random-total.png}
    %     \caption{Baseline Comparison - Random}
    %     \label{fig:eval:baselines:random:100GB:total}
    % \end{subfigure}    

    \caption{(a) Comparison of \Fresh\ against \MESSI\ on Seismic 100GB with variable query difficulty, where an increasing percentage of noise is added to the original queries.
    (b)-(c) Comparison of \Fresh\ index creation to other tree implementations.}
    \label{fig:eval:scale-dataset:tree-index}
\end{figure*}


\subsection{ FreSh vs Baselines.}
We compare \Fresh\ against several baseline {\em lock-free} implementations of the 
different stages of an iSAX-based index. 
Our results (Figure~\ref{fig:eval:baselines:random:100GB:total})
shows that \Fresh\ performs better than all these implementations.

\noindent
{\emph{\underline{Summarization Baseline:}}}
For buffer creation, we have experimented with three implementations: \DoAllSplit,
\FI, and \CASBased. All use a single summarization buffer with as many
elements as \RawData.
% 
\DoAllSplit\ splits \RawData\ into as many equaly-sized chunks as the number of threads.  
It stores a {\em done} flag with each data series, which is set after the data series is processed.
Each thread traverses \RawData\ (circularly), starting from the first element of its assigned chunk.
The thread first checks whether the done flag of a data series is set, and processes it only if not. 
%
In \FI, threads use \FAI\ to get assigned data series from \RawData\ to process.  
Each thread $t$ repeatedly performs the following:
It executes a \FAI\ on $O$ to get a position $v$ of RawData,
process the data series stored in this position, and afterwards sets its done flag to \True.
To achieve lock-freedom, \FI\ stores in RawData, a boolean flag (initally false) with each data series. 
This flag, which we call {\em done}, is set to true after the data series has been processed and its iSAX summary 
has been stored in the summarization buffer. 
When a thread figures out that all \RawData\ elements have been assigned,
(i.e., \FAI\ returns a value higher than the number of elements of RawData), 
it re-traverses \RawData\ to identify data series whose done flag is still \False,
and processes them. 
%
\CASBased\ works similarly to \FI, while it uses \CAS\ instructions, instead of \FAI.
%
\Fresh\ performs significantly better than all these implementations
(Fig.~\ref{fig:eval:baselines:random:100GB:total}).


\noindent
{\emph{\underline{Tree Population Baseline:}}}
Each thread is assigned elements of the summarization buffer using \FAI\
and inserts them in the index tree. 
To achieve lock-freedom in traversing the summarization buffer, 
we apply the \DoAllSplit, \FI, and \CASBased\ techniques we describe above.
%
To achieve lock-freedom in accessing the tree, we utilize a flagging technique~\cite{EFR+10},
in addition to our new tree implementation. 
A thread calls a search routine to traverse a path of the tree and reach an appropriate leaf
node $l$. Then, it flags the parent of $l$ using \CAS. If the flagging is successful, the insert
is performed. Then, $l$'s parent is unflagged (using \CAS).
Flagging stores a pointer to an info record in the flagged node.
Other threads may use this info record to help the insert complete. 

We have also experimented with \FINoSum, a lock-free implementation that 
avoids using the summarization buffers and inserts directly iSAX summaries in the index tree, 
by applying the \FI\ technique on \RawData.
%
\Fresh\ performs significantly better than all these implementations
(Figure~\ref{fig:eval:baselines:random:100GB:tree-index}).


\noindent
{\emph{\underline{Pruning Baseline:}}}
All baselines use a single instance of an existing skip-based lock-free priority queue~\cite{LJ13}
to store the candidate data series for refinement.
Threads uses \FAI\ to find the next node to examine in the index tree.
When a thread $t$ discovers that all nodes 
of the tree have been assigned for processing, it re-traverses 
the tree to find nodes that may still  be
unprocessed, and processes them. 
A flag is maintained for each tree node to indicate whether its
processing has been completed. This flag is set when the node is marked as processed. 
During re-traversal, $t$ examines
the flag of each element it visits, and does not process it if 
its flag is set (i.e., if the node is marked as processed).

\noindent
{\emph{\underline{Refinement Baseline:}}}
All threads, repeatedly call DeleteMin
to remove elements from the priority queue, and calculate their real distance computation. 
This simple algorithm is not lock-free as a thread may crash after it has
deleted an element from the queue. In this case, the deleted leaf will not be 
processed. This problem can be easily fixed but that would increase the cost 
of DeleteMin. Experiments show that even the non lock-free simple technique
above is quite costly in comparison to our approach.

Our results show that \Fresh\
performs significantly better than all these implementations,
for query answering time (that includes pruning and refinement, 
Figure~\ref{fig:eval:baselines:random:100GB:query-answering}),.

\begin{figure*}[htbp]
    \centering
    \begin{subfigure}{0.45\textwidth}  
        \includegraphics[width=\textwidth]{figures/Experiments/baselines-random-total.png}
        \caption{Total}
        \label{fig:eval:baselines:random:100GB:total}
    \end{subfigure}    
    \begin{subfigure}{0.45\textwidth}  
        \includegraphics[width=\textwidth]{figures/Experiments/baselines-random-summarization.png}
        \caption{Summarization}
        \label{fig:eval:baselines:random:100GB:summarization}
    \end{subfigure}    

    \begin{subfigure}{0.45\textwidth}  
        \includegraphics[width=\textwidth]{figures/Experiments/baselines-random-tree.png}
        \caption{Tree index}
        \label{fig:eval:baselines:random:100GB:tree-index}
    \end{subfigure}    
    \begin{subfigure}{0.45\textwidth}  
        \includegraphics[width=\textwidth]{figures/Experiments/baselines-random-query.png}
        \caption{Query answering}
        \label{fig:eval:baselines:random:100GB:query-answering}
    \end{subfigure}    

    \caption{Comparison of \Fresh\ against baseline implementations on 100GB Random.}
    \label{fig:eval:baselines:random:100GB}
\end{figure*}


%\subsection{Performance breakdown for index creation phase.}

We evaluate the techniques incorporated by \Fresh\ to create its
tree index by comparing it against three modified versions of it. 
Recall that in \Fresh\ each thread populates each of the subtrees it
acquires in expeditive mode, as long as no helper reaches the same leaf of the tree;
when this happens it changes its execution mode to standard. 
So, \Fresh\ allows leaves of the same subtree to be processed in different
modes of execution.

In the first modified version, called \FreshSub, threads start again by populating 
a subtree in expeditive mode, while they change to standard mode as long as a helper 
reaches this subtree (and not when it reaches one of its leaves, as \Fresh\ does); 
so, in \FreshSub\ all the leaves of a subtree are executed in a single mode at each 
point in time. 
In the second modified version, called \FreshSTD, threads populate subtrees using only 
the standard execution mode; i.e., there is no expeditive mode.
In the third modified implementation, called \FreshTreeCopy, a thread $t$ first populates a private
copy of the subtree (i.e. one that is accessible only to $t$) and only after its creation finishes,
$t$ tries to make it the (single) shared version of this subtree (by atomically changing a pointer 
using a \CAS\ instruction); threads help each other by following the same procedure.

Figures~\ref{fig:eval:scale-dataset:tree-index:random}-\ref{fig:eval:scale-dataset:tree-index:seismic}
compare \Fresh\ against the modified versions 
on Random and Seismic with variable dataset sizes and shows that it performs better than them,
in all cases. Interestingly, for Seismic 50GB \Fresh\ performs similarly to \FreshTreeCopy.
Recall that each thread works on its own private copy and, on each subtree, they contend at most once 
on the corresponding \CAS\ object. So, \FreshTreeCopy\ both restricts parallelism and minimizes the 
synchronization cost, which are properties that provide an advantage on Seismic. 


\noindent {\bf Thread Delays.} 
In order to study systems where processes may experience delays (e.g., due to page faults,
time sharing, or long phases of updates),
we came up with a simplistic benchmark, where we simulate delays at random points of a thread's execution. 
We do so by forcing threads to sleep for a specific amount of time, called {\em delay}.
Since \MESSI\ (and \MESSIenh) are lock-based, if a thread crashes, 
the algorithm will never terminate. Thus, we do not perform experiments for this case. 
This is due to the barriers that are utilized in their implementation.
These barriers are necessary for producing correct outcomes and their ignorance even by one thread
could compromise correctness. 
%
Figure~\ref{fig:eval:failure-query:random:failure1} illustrates 
that the delay even of a single thread causes a linear overhead on the performance of \MESSI, 
whereas it hardly has any impact in the performance of \Fresh.
This is so because in \Fresh, helper threads undertake the work to be done by the failed thread,
efficiently compensating the overhead from the failed thread delay.  
%S
Moreover, Figure~\ref{fig:eval:failure-query:random:failure2}
shows that \MESSI\ takes (almost) the full performance hit of delayed threads right from the beginning:
even a single delayed thread blocks 
the execution of all other threads, and hence of the entire algorithm.
\Fresh\ gracefully adapts to the situation of increasing number of delayed threads, 
achieving a speedup. % for Random and Seismic. 
When all-but-one threads fail, \Fresh\ is still faster than \MESSI\ because 
the single non-failing thread helps the others.
%
Note that these synthetic benchmarks are designed to simply illustrate the impact of lock-freedom 
on performance when threads may experience delays (or crash),
and not to capture some realistic setting. % for our studied problem.
Note that \MESSI\ will not terminate execution even if a single thread fails. 
In the case of failures (see Figure~\ref{fig:eval:variable-num-failures}),
\Fresh\ always terminates execution, performing almost identical to \MESSI\ 
with the same number of non-failing threads. This demonstrates that \Fresh adapts
to dynamic thread environments, maintaining high performance levels.


\begin{figure*}[htbp]
    \centering
    \begin{subfigure}{0.45\textwidth}  
        \includegraphics[width=\textwidth]{figures/Experiments/failure-delay-random-query25ms.png}
        \caption{A single thread is delayed.}
        \label{fig:eval:failure-query:random:failure1}
    \end{subfigure}    
    \begin{subfigure}{0.45\textwidth}  
        \includegraphics[width=\textwidth]{figures/Experiments/fail-threads-random-query25ms.png}
        \caption{Multiple threads are delayed.}
        \label{fig:eval:failure-query:random:failure2}
    \end{subfigure}    

    \caption{Comparison of \Fresh\ against \MESSI\ when varying delay and number of delayed threads.}
    \label{fig:eval:failure-query:random}
\end{figure*}

\begin{figure*}[htbp]
    \centering
    \begin{subfigure}{0.45\textwidth}  
        \includegraphics[width=\textwidth]{figures/Experiments/variable-num-failures-total}
        \caption{Total}
        \label{fig:eval:variable-num-failures:total}
    \end{subfigure}    
    \hfill
    \begin{subfigure}{0.45\textwidth}  
        \includegraphics[width=\textwidth]{figures/Experiments/variable-num-failures-query}
        \caption{Query answering}
        \label{fig:eval:variable-num-failures:queries}
    \end{subfigure}    
    \caption{Execution time on Random 100GB, when varying the number of threads that permanently fail 
    (\Fresh with permanent failures in red circles, \Fresh without failures in purple triangles, \MESSI\ without failures in blue crosses).}
    \label{fig:eval:variable-num-failures}
\end{figure*}

\clearpage

\section{Evaluation of Dynamic FreSh}

\noindent{\bf Setup.}
To evaluate DFreSh we used a different machine equipped with 
2 Intel(R) Xeon(R) Gold 5318Y CPU @ 2.10GHz
CPUs with 24 cores each, and 36MB L3 cache. The machine runs
Ubuntu Linux Ubuntu 24.04.1 LTS and has 256GB of RAM. Code is written in 
C and compiled using gcc (Ubuntu 13.2.0-23ubuntu4) 13.2.0 with O2 optimizations.
\noindent{\bf Datasets.} The datasets remain the same.
\noindent{\bf Evaluation Measures.}
We measure (i) the {\em index construction time} , 
(ii) the {\em query answering time} required to answer 50 queries per batch that are not
part of the dataset, 
(iii) we measure the latency of query answering and 
(iv) we provide a comprehend analysis that shows the performace of DFreSh with 
different settings, eg different delays between update batches, (different size of batches). 
Experiments are repeated $5$ times and averages are reported. All the algorithms
return exact results.

\noindent{\bf Dynamic FreSh vs FreSh.}
We compare the dynamic version of \Fresh, referred to as DFreSh, with \Fresh, 
the current state-of-the-art lock-free in-memory data series index. Our implementation 
includes two versions of DFreSh, each employing a different timestamping algorithm: 
Logical TS and System TS. Additionally, we implemented variations of both that directly 
insert data into the iSAX tree, bypassing the summarization buffers used in 
\Fresh\ during its first phase.
Unlike DFreSh, which supports insertions of batches dynamically, \Fresh\ is a static
system designed to process only a single preloaded dataset. To simulate incoming updates
in \Fresh, the entire dataset must be reloaded with each new update batch appended, requiring 
multiple runs. To ensure a fair comparison, we measure only the time required to 
incorporate the new update batch, isolating the update performance of both systems.

%%%%%%%%%%%%%%%%%%%%%%%%%%%%%%%%%%%%%%%%%%%%%%
\subsection{Evaluation using Random Datatasets}
All experiments performed begin by preloading the index with 10GB of data, called
\textit{initial data} ,followed by processing consecutive batches of $XGB$ until 
whole dataset is processed. For this experimental analysis we have also decided
that each batch is accompanied by 50 new queries. Preloading ensures that even the first 
queries can find answers. We have also desided to add a delay between consecutive batches.
This delay start when a batch arrives and indicates when the next batch is available for
processing.

Figure ~\ref{fig:dfresh-fresh-random} illustrates the overall performance of all 
algorithms when building a 100GB index using random data. Specifically,
figure ~\ref{fig:actual-index-Construction-time} compares the actual index construction times 
of all algorithms, including \Fresh. As expected, the No Buffers version 
perform worse due to reduced cache locality caused by bypassing summarization buffers.
Despite the overhead of managing dynamic updates (e.g such as handling timestamps), 
the buffered versions (Logical and System TS) achieve performance comparable to \Fresh. 
Figure ~\ref{fig:actual-query-answering-time} shows the 
actual query answering time for all algorithms. The buffered versions ouperform No Buffers versions 
because the same index worker operates on a subtree, meaning the cache lines of that worker 
contatin data drom the same subtree leading to improved cache locality. In addition to that
\Fresh\ avoids concurrent reads and writes, reducing cache misses.



The minimum delay, set at 2 seconds (Figure ~\ref{fig:min-delay}), corresponds to the 
time required by the fastest algorithm to answer 50 queries on the smallest index size 
(during the insertion of the first batch). The maximum delay, set at 7 seconds
(Figure ~\ref{fig:max-delay}), allows the slowest algorithm to complete 50 queries on the
largest index size (last batch). In figure~\ref{query-answering-breakdown-random} the bars represent query answering times,
divided into two segments: the light segment indicates concurrent query answering
during index construction, while the dark segment represents solo query answering.
A dark segment indicates insufficient delay, causing some queries to run during
the insertion of other batches.
%
%Once again, the No Buffers versions perform the worst, as cache misses are more costly when not 
%all the data in  a cache line is useful. In contrast, the buffered versions perform better because 
%the same index worker operates on a subtree, meaning the cache lines of that worker 
%contain data from the same subtree, leading to improved cache locality.
Query answering performance is heavily influenced by the delay. Shorter delays result
in poorer performance, as query workers have less time to process the same number
of queries while index construction is ongoing. This results in more pending queries,
increasing computational effort and response times.



%%%%%%%%%%%%%%%%%%%%%%%%% OVERALL PERFORMANCE %%%%%%%%%%%%%%%%%%%%%%%%%%%%%%%%%%%%%%%%%%%%%%%%%%

\begin{figure*}
	\centering
	\begin{subfigure}[c]{0.45\textwidth}
		\includegraphics[width=1\textwidth]   {figures/Experiments/Dynamic/Delays/index_construction_all.png}
		\caption{Index Construction Time}
		\label{fig:actual-index-Construction-time}
	\end{subfigure}
	\begin{subfigure}[c]{0.45\textwidth}
		\includegraphics[width=1\textwidth]   {figures/Experiments/Dynamic/Delays/qa_delay_x_axis.png}
		\caption{Query Answering Time}
		\label{fig:actual-query-answering-time}
	\end{subfigure}
	\caption{Dynamic FreSh Benchmarks - Random 100GB/450Q}
	\label{fig:dfresh-fresh-random}
\end{figure*}

%%%%%%%%%%%%%%%%%%%%%%%%%%% Smallest - Largest Delay %%%%%%%%%%%%%%%%%%%%%%%%%%%%%%%%%%%%%%%%%%%%%%%%%%%

\begin{figure*}
	\centering
	\begin{subfigure}[c]{0.4\textwidth}
		\includegraphics[width=1\textwidth]   {figures/Experiments/Dynamic/Delays/shortest_delay.png}
		\caption{Minimum Delay}
		\label{fig:min-delay}
	\end{subfigure}
	\begin{subfigure}[c]{0.4\textwidth}
		\includegraphics[width=1\textwidth]   {figures/Experiments/Dynamic/Delays/longest_delay.png}
		\caption{Maximum Delay}
		\label{fig:max-delay}
	\end{subfigure}
	\caption{Minimum - Longest Delay}
	\label{fig:min-max-delay}
\end{figure*}

%%%%%%%%%%%%%%%%%%%%%%%%%%%%%%% X AXIS Size %%%%%%%%%%%%%%%%%%%%%%%%%%%%%%%%%%%%%%%%%%%%%%%%%%%%%%%%%%%%%%%
\begin{figure*}
	\centering
	\begin{subfigure}[c]{0.45\textwidth}
		\includegraphics[width=1\textwidth]   {figures/Experiments/Dynamic/xAxis/x_axis_delay[2].png}
		\caption{Query Answering Time Delay 2 sec}
		\label{fig:QA-xAxis-delay-2}
	\end{subfigure}
	\begin{subfigure}[c]{0.45\textwidth}
		\includegraphics[width=1\textwidth]   {figures/Experiments/Dynamic/xAxis/x_axis_delay[3].png}
		\caption{Query Answering Time Delay 3 sec}
		\label{fig:QA-xAxis-delay-3}
	\end{subfigure}
	\begin{subfigure}[c]{0.45\textwidth}
		\includegraphics[width=1\textwidth]   {figures/Experiments/Dynamic/xAxis/x_axis_delay[5].png}
		\caption{Query Answering Time Delay 5 sec}
		\label{fig:QA-xAxis-delay-5}
	\end{subfigure}
	\begin{subfigure}[c]{0.45\textwidth}
		\includegraphics[width=1\textwidth]   {figures/Experiments/Dynamic/xAxis/x_axis_delay[7].png}
		\caption{Query Answering Time Delay 7 sec}
		\label{fig:QA-xAxis-delay-7}
	\end{subfigure}
	\caption{Query Answering Time vs. Index Size}
\end{figure*}


%%%%%%%%%%%%%%%%%%%%%%%%%%%%%% Query Breakdown %%%%%%%%%%%%%%%%%%%%%%%%%%%%%%%%%%%%%%%%%%%%%%%%%%%%%%%%%%%

\begin{figure*}
	\centering
	\begin{subfigure}[c]{0.45\textwidth}
		\includegraphics[width=1\textwidth]   {figures/Experiments/Dynamic/Breakdown/dataset_104857600_lockfree_Messi_Results_query_answering_breakdown_10485760_2.png}
		\caption{Delay 2 sec}
		\label{fig:query-answering-breakdown-2}
	\end{subfigure}
	\begin{subfigure}[c]{0.45\textwidth}
		\includegraphics[width=1\textwidth]   {figures/Experiments/Dynamic/Breakdown/dataset_104857600_lockfree_Messi_Results_query_answering_breakdown_10485760_3.png}
		\caption{Delay 3 sec}
		\label{fig:query-answering-breakdown-3}
	\end{subfigure}
	\begin{subfigure}[c]{0.45\textwidth}
		\includegraphics[width=1\textwidth]   {figures/Experiments/Dynamic/Breakdown/dataset_104857600_lockfree_Messi_Results_query_answering_breakdown_10485760_5.png}
		\caption{Delay 5 sec}
		\label{fig:query-answering-breakdown-5}
	\end{subfigure}
	\begin{subfigure}[c]{0.45\textwidth}
		\includegraphics[width=1\textwidth]   {figures/Experiments/Dynamic/Breakdown/dataset_104857600_lockfree_Messi_Results_query_answering_breakdown_10485760_7.png}
		\caption{Delay 7 sec}
		\label{fig:query-answering-breakdown-7}
	\end{subfigure}
	\caption{Query Answering Breakdown - Random 100GB/450Q}
	\label{query-answering-breakdown-random}
\end{figure*}

\clearpage
\subsubsection{Query Progress}
Figures~\ref{fig:query-progress-delay-2}-\ref{fig:query-progress-delay-7} illustrate
the progress of DFreSh while batches are being inserted. These graphs provide an overview
of the number of queries answered as the index grows, with the x-axis representing the
index size.
%
For example, in Figure~\ref{fig:progress-queries-2-logical}, the first bar shows
that 50 queries were answered between the initial index state and the insertion of
the first batch. This indicates that the implementation of DFreSh using logical timestamps
processes all queries associated with that batch before the next insertion. In contrast,
Figure~\ref{fig:progress-queries-2-system-no-buffers} shows that the implementation
without summarization buffers and using system timestamps answers only 31 out of 50
queries in the same timeframe.
%
These graphs also provide insights into the overall query answering performance, as
summarized in Figure~\ref{fig:actual-query-answering-time}. Notably, as the delay
increases, even the slowest implementations,those without summarization buffers, reduce
the number of queries carried over to subsequent batches. This suggests that with
sufficient delay, even these slower implementations can complete all queries within
the available time.
%%%%% SMALLEST DELAY
\begin{figure*}
	\centering
	\begin{subfigure}[c]{0.45\textwidth}
		\includegraphics[width=1\textwidth]   {figures/Experiments/Dynamic/Progress/2/average_query_time_per_batch_version_999777015_10485760_10_delay[2].png}
		\caption{Logical TS buffers}
		\label{fig:progress-queries-2-logical}
	\end{subfigure}
	\begin{subfigure}[c]{0.45\textwidth}
		\includegraphics[width=1\textwidth]   {figures/Experiments/Dynamic/Progress/2/average_query_time_per_batch_version_999777018_10485760_10_delay[2].png}
		\caption{System TS buffers}
		\label{fig:progress-queries-2-system}
	\end{subfigure}
	\begin{subfigure}[c]{0.45\textwidth}
		\includegraphics[width=1\textwidth]   {figures/Experiments/Dynamic/Progress/2/average_query_time_per_batch_version_999777016_10485760_10_delay[2].png}
		\caption{Logical TS No buffers}
		\label{fig:progress-queries-2-logical-no-buffers}
	\end{subfigure}
	\begin{subfigure}[c]{0.45\textwidth}
		\includegraphics[width=1\textwidth]   {figures/Experiments/Dynamic/Progress/2/average_query_time_per_batch_version_999777017_10485760_10_delay[2].png}
		\caption{System TS No buffers}
		\label{fig:progress-queries-2-system-no-buffers}
	\end{subfigure}
	\caption{Query progress with minimum delay 2 sec.}
	\label{fig:query-progress-delay-2}
\end{figure*}

%%%%%%%%%Intermediate delays 3sec
\begin{figure*}
	\centering
	\begin{subfigure}[c]{0.4\textwidth}
		\includegraphics[width=1\textwidth]   {figures/Experiments/Dynamic/Progress/3/average_query_time_per_batch_version_999777015_10485760_10_delay[3].png}
		\caption{Logical TS buffers}
		\label{fig:progress-queries-3-logical}
	\end{subfigure}
	\begin{subfigure}[c]{0.4\textwidth}
		\includegraphics[width=1\textwidth]   {figures/Experiments/Dynamic/Progress/3/average_query_time_per_batch_version_999777018_10485760_10_delay[3].png}
		\caption{System TS buffers}
		\label{fig:progress-queries-3-system}
	\end{subfigure}
	\begin{subfigure}[c]{0.4\textwidth}
		\includegraphics[width=1\textwidth]   {figures/Experiments/Dynamic/Progress/3/average_query_time_per_batch_version_999777016_10485760_10_delay[3].png}
		\caption{Logical TS No buffers}
		\label{fig:progress-queries-3-logical-no-buffers}
	\end{subfigure}
	\begin{subfigure}[c]{0.4\textwidth}
		\includegraphics[width=1\textwidth]   {figures/Experiments/Dynamic/Progress/3/average_query_time_per_batch_version_999777017_10485760_10_delay[3].png}
		\caption{System TS No buffers}
		\label{fig:progress-queries-3-system-no-buffers}
	\end{subfigure}
	\caption{Query progress with Intermediate delay 3 sec.}
	\label{fig:query-progress-delay-3}
\end{figure*}


%%%%%%%%%Intermediate delays 5sec
\begin{figure*}
	\centering
	\begin{subfigure}[c]{0.4\textwidth}
		\includegraphics[width=1\textwidth]   {figures/Experiments/Dynamic/Progress/5/average_query_time_per_batch_version_999777015_10485760_10_delay[5].png}
		\caption{Progress of Queries Logical TS buffers}
		\label{fig:progress-queries-5-logical}
	\end{subfigure}
	\begin{subfigure}[c]{0.4\textwidth}
		\includegraphics[width=1\textwidth]   {figures/Experiments/Dynamic/Progress/5/average_query_time_per_batch_version_999777018_10485760_10_delay[5].png}
		\caption{Progress of Queries System TS buffers}
		\label{fig:progress-queries-5-system}
	\end{subfigure}
	\begin{subfigure}[c]{0.4\textwidth}
		\includegraphics[width=1\textwidth]   {figures/Experiments/Dynamic/Progress/5/average_query_time_per_batch_version_999777016_10485760_10_delay[5].png}
		\caption{Progress of Queries Logical TS No buffers}
		\label{fig:progress-queries-5-logical-no-buffers}
	\end{subfigure}
	\begin{subfigure}[c]{0.4\textwidth}
		\includegraphics[width=1\textwidth]   {figures/Experiments/Dynamic/Progress/5/average_query_time_per_batch_version_999777017_10485760_10_delay[5].png}
		\caption{Progress of Queries System TS No buffers}
		\label{fig:progress-queries-5-system-no-buffers}
	\end{subfigure}
	\caption{Query progress with Intermediate delay 5 sec.}
	\label{fig:query-progress-delay-5}

\end{figure*}

%%%%% LARGEST DELAY
\begin{figure*}
	\centering
	\begin{subfigure}[c]{0.4\textwidth}
		\includegraphics[width=1\textwidth]   {figures/Experiments/Dynamic/Progress/7/average_query_time_per_batch_version_999777015_10485760_10_delay[7].png}
		\caption{Progress of Queries Logical TS buffers}
		\label{fig:progress-queries-7-logical}
	\end{subfigure}
	\begin{subfigure}[c]{0.4\textwidth}
		\includegraphics[width=1\textwidth]   {figures/Experiments/Dynamic/Progress/7/average_query_time_per_batch_version_999777018_10485760_10_delay[7].png}
		\caption{Progress of Queries System TS buffers}
		\label{fig:progress-queries-7-system}
	\end{subfigure}
	\begin{subfigure}[c]{0.4\textwidth}
		\includegraphics[width=1\textwidth]   {figures/Experiments/Dynamic/Progress/7/average_query_time_per_batch_version_999777016_10485760_10_delay[7].png}
		\caption{Progress of Queries Logical TS No buffers}
		\label{fig:progress-queries-7-logical-no-buffers}
	\end{subfigure}
	\begin{subfigure}[c]{0.4\textwidth}
		\includegraphics[width=1\textwidth]   {figures/Experiments/Dynamic/Progress/7/average_query_time_per_batch_version_999777017_10485760_10_delay[7].png}
		\caption{Progress of Queries System TS No buffers}
		\label{fig:progress-queries-7-system-no-buffers}
	\end{subfigure}
	\caption{Query progress with largest delay, 7 sec.}
	\label{fig:query-progress-delay-7}
\end{figure*}



\subsubsection{Latency}

Figure \ref{fig:latency-random} presents the average latency for each query. 

\begin{definition}[Query Latency]
Query latency is defined as the time between a query's arrival and the start of its processing.
\end{definition}

As mentioned earlier, queries arrive in batches along with the update batch, meaning 
that all queries in a batch share the same arrival time. The latency of a query is 
defined as the time required to process all pending queries that arrived before it.
%
When the delay is long enough to process all queries in a batch, the latency of the 
$i$-th query can be calculated as the sum of the times required to answer all previous 
queries in the same batch:

\[
\text{Latency}(q[i]) = \sum_{j=0}^{i-1} \text{qa}[j],
\]

where $\text{qa}[j]$ represents the time needed to answer the $j$-th query.

However, if the delay is not sufficient to process all queries within a batch, 
the latency of the $i$-th query is calculated differently. It includes the time 
required to answer the pending queries from previous updates (queries from previous batches) 
and the time for all the queries that arrived before the $i$-th query in the current batch:

\[
\text{Latency}(q[i]) = \sum_{j=k}^{n-1} \text{qa}[j] + \sum_{j=0}^{i-1} \text{qa}[j],
\]

where the first sum accounts for the pending queries from previous updates
and the second sum accounts for the queries from the current batch up to the $i$-th query.
%
For \Fresh\, calculating latency is more complicated because it is a static index, meaning 
query answering begins only after the index construction is complete. The time available 
for answering queries in a given update batch can be expressed as:
\[
\text{AvailableQA}(qa[i]) = \text{DELAY} - \text{IC}[i],
\]
where $\text{AvailableQA}(qa[i])$ represents the available time for answering queries 
in the $i$-th update batch, $\text{DELAY}$ is the delay interval, and $\text{IC}[i]$ 
is the time required to insert the $i$-th batch into the index.

If the delay is sufficient to answer all the queries in time, the latency can be calculated as:

\[
\text{Latency}(q[i]) = \text{IC}[i] + \sum_{j=0}^{i-1} \text{qa}[j],
\]

where $\text{IC}[i]$ represents the time to append the $i$-th batch to the index, and 
$\text{qa}[j]$ is the time required to answer the $j$-th query. If the delay is insufficient, 
unanswered queries from the current batch remain pending and carry over to the next query 
answering period. These pending queries will be processed during the 
$\text{AvailableQA}(qa[i+1])$ interval of the subsequent update batch. 
Thus, the latency can be calculated as:

\[
\text{Latency}(q[i]) = \sum_{k=n}^{m} \text{IC}[k] + \sum_{j=0}^{i-1} \text{qa}[j],
\]

where the $n$-th batch contains the queries in question, and the $m$-th batch represents 
when query $i$ is being answered. If the current batch also includes pending queries from 
previous batches, the latency must also account for the time to answer those queries. 
Thus, the general formula for calculating latency in \Fresh\ is:

\[
\text{Latency}(q[i]) = \sum_{k=n}^{m} \text{IC}[k] + \sum_{j=p}^{pq} \text{qa}[j] + \sum_{j=0}^{i-1} \text{qa}[j],
\]
where $n$ represents the batch that contains the query $i$,-th 
$m$ represents the batch the the $i$-th query is being answered, 
and queries $p$ to $pq$ represents the pending queries from previous baches.

In Figure \ref{fig:latency-random}, System TS Buffers and FreSh exhibit similar latency. 
The key difference between these two implementations, aside from the fact that System TS Buffers 
support dynamic updates, is that System TS Buffers allow concurrent query execution with timestamps 
earlier than the timestamp of the inserting batch. In our implementation, each query receives 
an incrementing timestamp, meaning that the timestamp of query $i+1$ will be larger than that 
of query $i$. As a result, only a small portion of the queries will be fast enough to be answered
before the update batch acquires a timestamp. After this point, queries must wait for the index to 
be completed before they can proceed. In the worst-case scenario for System TS Buffers, queries 
are never fast enough to be answered before the new batch acquires a timestamp, causing it to 
behave similarly to FreSh.

\begin{figure*}
	\centering
	\begin{subfigure}[c]{0.6\textwidth}
		\includegraphics[width=1\textwidth]   {figures/Experiments/Dynamic/Latency/average_latency.png}
		\label{fig:average-latency}
	\end{subfigure}
	\caption{Query Latency}
	\label{fig:latency-random}
\end{figure*}

\subsubsection{DFreSh with different batch sizes}

We have also conducted experiments using different batch sizes. In our analysis,
we focus on batch sizes of $5$GB, $10$GB, and $20$GB. Each batch corresponds to
answering $50$ queries. Consequently, for $5$GB batches, we process a total of
$900$ queries over a $100$GB dataset, whereas for $20$GB batches, we process
$250$ queries over a $110$GB dataset, considering that the initial index size
remains $10$GB throughout our experiments.

Figure~\ref{fig:dfresh-fresh-random-different-batches} presents the overall
performance of DFreSh across different batch sizes. The results indicate that the
behavior remains consistent, as batch size does not significantly impact either the
index construction or query answering performance. In particular, the average query
answering time shows only minimal variation across different batch sizes.

\begin{figure*}
	\centering
	\begin{subfigure}[c]{0.45\textwidth}
		\includegraphics[width=1\textwidth]   {figures/Experiments/Dynamic/5GB/7/indexConstruction_7_5GB.png}
		\caption{Real Index Construction Time 5GB Batches}
		\label{fig:actual-index-Construction-time-5GB}
	\end{subfigure}
	\begin{subfigure}[c]{0.45\textwidth}
		\includegraphics[width=1\textwidth]	 {figures/Experiments/Dynamic/5GB/delays_xaxis_5GB.png}
		\caption{Real Query Answering Time 5GB Batches}
		\label{fig:actual-query-answering-time-5GB}
	\end{subfigure}
	\begin{subfigure}[c]{0.45\textwidth}
		\includegraphics[width=1\textwidth]   {figures/Experiments/Dynamic/20GB/7/dataset_115343360_lockfree_Messi_ResultsindexConstruction_7_20GB.png}
		\caption{Real Index Construction Time 20GB Batches}
		\label{fig:actual-index-Construction-time-20GB}
	\end{subfigure}
	\begin{subfigure}[c]{0.45\textwidth}
		\includegraphics[width=1\textwidth]	 {figures/Experiments/Dynamic/20GB/dataset_115343360_lockfree_Messi_Results_query_answering_initial[10485760]_delays_20GB.png}
		\caption{Real Query Answering Time 20GB Batches}
		\label{fig:actual-query-answering-time-20GB}
	\end{subfigure}
	\caption{Dynamic FreSh Benchmarks with different batches}
	\label{fig:dfresh-fresh-random-different-batches}
\end{figure*}


\clearpage
\subsection{Evaluation using Real Datatasets}

This section focuses on experiments conducted using real datasets. Our
experimental analysis of DFreSh requires executing 50 queries per inserted
batch. However, the original dataset contained only 100 queries, which were
insufficient for our experiments. To generate additional queries, we randomly
selected series from the dataset, added Gaussian noise 
($\mu = 0$, $sigma = 0.1$) to each point, and used these modified series 
as queries. Answering queries on real datasets typically requires more time,
as observed in the evaluation of FreSh and previous works, since real queries
tend to be more challenging. The added noise further increases query difficulty,
making the evaluation even more demanding.As a result, the delays used for 
random datasets cannot be applied to real datasets. 
Given that query answering is more computationally intensive than index
construction in this setting, we adjusted the distribution of worker threads
between these tasks. Instead of assigning $36$ workers to index construction
and $12$ to query answering, as in previous experiments, we reversed the
allocation, dedicating $36$ workers to query answering and $12$ to index
construction. This ensures that more computational resources are available
for the more demanding phase of the process.

\noindent{\textbf{Astro Dataset}}  
The initial index size is set to $10$GB, and we use only the first $100$GB
of the dataset. This decision is based on two factors: (1) previous experiments
have shown that dataset size does not affect the average performance of DFreSh,
and (2) the experiment is computationally expensive and time-consuming.


\begin{figure*}
	\centering
	\begin{subfigure}[c]{0.45\textwidth}
		\includegraphics[width=1\textwidth]   {figures/Experiments/Dynamic/ASTRO/index_construction_astro.png}
		\caption{Astro: Real Index Construction Time }
		\label{fig:actual-index-Construction-time-astro}
	\end{subfigure}
	\begin{subfigure}[c]{0.45\textwidth}
		\includegraphics[width=1\textwidth]	 {figures/Experiments/Dynamic/ASTRO/astro_query_answering_xaxis.png}
		\caption{Astro: Real Query Answering Time}
		\label{fig:actual-query-answering-time-astro}
	\end{subfigure}
	\caption{Astro: DFreSh Overall Performance}
	\label{fig:dfresh-performance-astro}
\end{figure*}

Figure~\ref{fig:dfresh-performance-astro} presents the overall performance
of DFreSh. Specifically, Figure~\ref{fig:actual-index-Construction-time-astro}
shows the index construction time, where, as expected, versions utilizing
summarization buffers achieve better performance. We observe that
\textit{System Timestamps} outperform \textit{Logical Timestamps}. 
This is expected, as system timestamps eliminate the need for CAS instructions
during timestamp assignment and reduce concurrency in the iSAX tree between
workers handling different phases due to the waiting protocol we described,
which in turn leads to fewer cache misses. 
Figure~\ref{fig:actual-query-answering-time-astro} illustrates the total query
answering time for processing 450 queries. As expected, summarization buffers
improve performance. Additionally, as the delay increases, different versions
tend to perform better because more queries are processed within each batch.
Consequently, fewer queries are carried over to subsequent batches, reducing
the number of queries that must be answered on an increasingly larger index.

%%%%%%%%%%%%%%%%%%%%%%%%%%%%%%% Query Breakdown %%%%%%%%%%%%%%%%%%%%%%%%%%%%%%
\noindent{\textbf{Query Breakdown}}
%
Figure~\ref{fig:dfresh-query-breakdown-astro} illustrates the distribution of query
answering time in DFreSh. It shows how much of the query processing occurs during the
concurrent execution phase, including the delay times, and how much takes place after
index construction is complete. This distinction is represented by the light and dark
segments of the bars, respectively. Additionally, this figure provides insight into
the difficulty of query answering. The required delay is $10$ times larger than that 
in the Random dataset.

\begin{figure*}
	\centering
	\begin{subfigure}[c]{0.45\textwidth}
		\includegraphics[width=1\textwidth]   {figures/Experiments/Dynamic/ASTRO/breakdown_astro_50.png}
		\caption{Astro: Query Breakdown 50 Delay}
		\label{fig:actual-query-breakdown-50-astro}
	\end{subfigure}
	\begin{subfigure}[c]{0.45\textwidth}
		\includegraphics[width=1\textwidth]	 {figures/Experiments/Dynamic/ASTRO/breakdown_astro_60.png}
		\caption{Astro: Astro: Query Breakdown 60 Delay}
		\label{fig:actual-query-breakdown-60-astro}
	\end{subfigure}
	\begin{subfigure}[c]{0.45\textwidth}
		\includegraphics[width=1\textwidth]	 {figures/Experiments/Dynamic/ASTRO/breakdown_astro_70.png}
		\caption{Astro: Astro: Query Breakdown 70 Delay}
		\label{fig:actual-query-breakdown-70-astro}
	\end{subfigure}
	\caption{Astro: Query Performance Breakdown}
	\label{fig:dfresh-query-breakdown-astro}
\end{figure*}

%%%%%%%%%%%%%%%%%%%%%%%%%%%%%%%% Query Progress %%%%%%%%%%%%%%%%%%%%%%%%%%%%%%%%%%%%%%
\noindent{\textbf{Query Progress}}
%
Figures~\ref{fig:query-progress-50-astro} to~\ref{fig:query-progress-70-astro} show 
the average query answering time as the index size increases. The details of what each
bar represents are explained in the same experiment conducted on the random dataset.
A few observations can be made from these graphs. As expected, as the delay increases,
all versions of DFreSh continue to answer queries in a timely manner. However, in
Figures~\ref{fig:logical-ts-50-astro} to~\ref{fig:system-ts-50-astro}, a different
pattern emerges between the \textit{Logical} and \textit{System} timestamps. Although
\textit{System} timestamps appear fast enough to answer queries promptly in the early
stages, they end up perform worse than \textit{Logical}.
%
From these graphs, we observe that when the index is around half its full capacity,
the average query answering time reaches its peak. This is likely because the queries
being answered during this period are the most difficult or the queries cause contention
and thus cache misses. Since \textit{Logical} timestamps support more concurrency
between workers, they face greater contention, resulting in fewer queries being
answered in that phase. For instance, between the $40$-$60$GB range, the
\textit{System} version answers $89$ queries in total, while the \textit{Logical}
version answers only $80$. However, in the next range ($70$-$80$GB), \textit{Logical}
timestamps manage to close the $9$-query gap, answering queries in a smaller
average time.


\begin{figure*}
	\centering
	\begin{subfigure}[c]{0.45\textwidth}
		\includegraphics[width=1\textwidth]   {figures/Experiments/Dynamic/ASTRO/Batch_processing/50/average_query_time_per_batch_version_999777015_10485760_10_delay[50].png}
		\caption{Astro: Logical TS Buffers}
		\label{fig:logical-ts-50-astro}
	\end{subfigure}
	\begin{subfigure}[c]{0.45\textwidth}
		\includegraphics[width=1\textwidth]	 {figures/Experiments/Dynamic/ASTRO/Batch_processing/50/average_query_time_per_batch_version_999777018_10485760_10_delay[50].png}
		\caption{Astro: System TS Buffers}
		\label{fig:system-ts-50-astro}
	\end{subfigure}
	\begin{subfigure}[c]{0.45\textwidth}
		\includegraphics[width=1\textwidth]	 {figures/Experiments/Dynamic/ASTRO/Batch_processing/50/average_query_time_per_batch_version_999777016_10485760_10_delay[50].png}
		\caption{Astro: Logical TS No Buffers}
		\label{fig:logical-ts-no-50-astro}
	\end{subfigure}
	\begin{subfigure}[c]{0.45\textwidth}
		\includegraphics[width=1\textwidth]	 {figures/Experiments/Dynamic/ASTRO/Batch_processing/50/average_query_time_per_batch_version_999777017_10485760_10_delay[50].png}
		\caption{Astro: System TS No Buffers}
		\label{fig:system-ts-no-50-astro}
	\end{subfigure}
	\caption{Astro: Query Progress 50 delay}
	\label{fig:query-progress-50-astro}
\end{figure*}
%%%%%%%%%%%%%%%%%%%%%%%%%%%%%% Mesaio Delay %%%%%%%%%%%%%%%%%%%%%%%%%%%%%%%%%%%%%%%%%%%%%%%
\begin{figure*}
	\centering
	\begin{subfigure}[c]{0.45\textwidth}
		\includegraphics[width=1\textwidth]   {figures/Experiments/Dynamic/ASTRO/Batch_processing/60/average_query_time_per_batch_version_999777015_10485760_10_delay[60].png}
		\caption{Astro: Logical TS Buffers}
		\label{fig:logical-ts-60-astro}
	\end{subfigure}
	\begin{subfigure}[c]{0.45\textwidth}
		\includegraphics[width=1\textwidth]	 {figures/Experiments/Dynamic/ASTRO/Batch_processing/60/average_query_time_per_batch_version_999777018_10485760_10_delay[60].png}
		\caption{Astro: System TS Buffers}
		\label{fig:system-ts-60-astro}
	\end{subfigure}
	\begin{subfigure}[c]{0.45\textwidth}
		\includegraphics[width=1\textwidth]	 {figures/Experiments/Dynamic/ASTRO/Batch_processing/60/average_query_time_per_batch_version_999777016_10485760_10_delay[60].png}
		\caption{Astro: Logical TS No Buffers}
		\label{fig:logical-ts-no-60-astro}
	\end{subfigure}
	\begin{subfigure}[c]{0.45\textwidth}
		\includegraphics[width=1\textwidth]	 {figures/Experiments/Dynamic/ASTRO/Batch_processing/60/average_query_time_per_batch_version_999777017_10485760_10_delay[60].png}
		\caption{Astro: System TS No Buffers}
		\label{fig:system-ts-no-60-astro}
	\end{subfigure}
	\caption{Astro: Query Progress 60 delay}
	\label{fig:query-progress-60-astro}
\end{figure*}
%%%%%%%%%%%%%%%%%%%%%%%%%% Largest%%%%%%%%%%%%%%%%%%%%%%%%%%%%%%%%%%%%%%%%%%%%%%%%%%%%%%%%
\begin{figure*}
	\centering
	\begin{subfigure}[c]{0.45\textwidth}
		\includegraphics[width=1\textwidth]   {figures/Experiments/Dynamic/ASTRO/Batch_processing/70/average_query_time_per_batch_version_999777015_10485760_10_delay[70].png}
		\caption{Astro: Logical TS Buffers}
		\label{fig:logical-ts-70-astro}
	\end{subfigure}
	\begin{subfigure}[c]{0.45\textwidth}
		\includegraphics[width=1\textwidth]	 {figures/Experiments/Dynamic/ASTRO/Batch_processing/70/average_query_time_per_batch_version_999777018_10485760_10_delay[70].png}
		\caption{Astro: System TS Buffers}
		\label{fig:system-ts-70-astro}
	\end{subfigure}
	\begin{subfigure}[c]{0.45\textwidth}
		\includegraphics[width=1\textwidth]	 {figures/Experiments/Dynamic/ASTRO/Batch_processing/70/average_query_time_per_batch_version_999777016_10485760_10_delay[70].png}
		\caption{Astro: Logical TS No Buffers}
		\label{fig:logical-ts-no-70-astro}
	\end{subfigure}
	\begin{subfigure}[c]{0.45\textwidth}
		\includegraphics[width=1\textwidth]	 {figures/Experiments/Dynamic/ASTRO/Batch_processing/70/average_query_time_per_batch_version_999777017_10485760_10_delay[70].png}
		\caption{Astro: System TS No Buffers}
		\label{fig:system-ts-no-70-astro}
	\end{subfigure}
	\caption{Astro: Query Progress 70 delay}
	\label{fig:query-progress-70-astro}
\end{figure*}

%%%%%%%%%%%%%%%%%%%%%%%%%% Latency %%%%%%%%%%%%%%%%%%%%%%%%%%%%%%%%%%%%%%%%%%%%%%%
\noindent{\textbf{Latency}}
%
Figure~\ref{fig:query-latency} shows the average query latency for the
Astro dataset. As expected, the latency increases as query answering
becomes more resource-intensive. The latency is calculated as described
in the previous sections, and the observed pattern aligns with expectations.

\begin{figure*}
	\centering
	\begin{subfigure}[c]{0.45\textwidth}
		\includegraphics[width=1\textwidth]   {figures/Experiments/Dynamic/ASTRO/average_latency_ASTRO.png}
	\end{subfigure}
	\caption{Astro: Query Latency}
	\label{fig:query-latency}
\end{figure*}
%%%%%%%%%%%%%%%%%%%%%%%%%%%%% SEISMIC DATASET %%%%%%%%%%%%%%%%%%%%%%%%%%%%%%%%%%%%%%%%%%%%%%%%%%%%%%%%%%%
\noindent{\textbf{Seismic Dataset}} 

The initial index size is set to $10$GB, and the total size of the
dataset is limited to  $90$GB.
 
\begin{figure*}
	\centering
	\begin{subfigure}[c]{0.45\textwidth}
		\includegraphics[width=1\textwidth]   {figures/Experiments/Dynamic/SEISMIC/index_construction_seismic.png}
		\caption{Seismic: Real Index Construction Time }
		\label{fig:actual-index-Construction-time-seismic}
	\end{subfigure}
	\begin{subfigure}[c]{0.45\textwidth}
		\includegraphics[width=1\textwidth]	 {figures/Experiments/Dynamic/SEISMIC/query_answering_xaxis.png}
		\caption{Seismic: Real Query Answering Time}
		\label{fig:actual-query-answering-time-seismic}
	\end{subfigure}
	\caption{Seismic: DFreSh Overall Performance}
	\label{fig:dfresh-performance-seismic}
\end{figure*}

Figure~\ref{fig:dfresh-performance-seismic} presents the overall performance of DFreSh
on the Seismic dataset. As with the Astro dataset, 
Figure~\ref{fig:actual-index-Construction-time-seismic} shows that versions utilizing 
summarization buffers achieve better index construction performance. Similarly, 
Figure~\ref{fig:actual-query-answering-time-seismic} illustrates the total query
answering time for processing 400 queries, where summarization buffers again lead to
improved efficiency. Based on our analysis, the chosen delays for this dataset are set
to 25, 35, and 50 seconds. The observations made for the \textit{Astro} dataset also 
apply here, highlighting consistent trends across different real-world datasets.

%%%%%%%%%%%%%%%%%%%%%%%%%%%%%%% Query Breakdown %%%%%%%%%%%%%%%%%%%%%%%%%%%%%%
\noindent{\textbf{Query Breakdown}}
%
Figure~\ref{fig:dfresh-query-breakdown-seismic} illustrates the distribution of query 
answering time in DFreSh for the Seismic dataset. As in the Astro dataset, it
distinguishes between queries processed during the concurrent execution phase,
including delay times, and those answered after index construction is complete,
represented by the light and dark segments of the bars, respectively. Once again,
we observe the increased difficulty of query answering in real datasets compared to
synthetic ones, with the required delay being approximately seven times higher.

\begin{figure*}
	\centering
	\begin{subfigure}[c]{0.45\textwidth}
		\includegraphics[width=1\textwidth]   {figures/Experiments/Dynamic/SEISMIC/25/breakdown_seismic_25.png}
		\caption{Seismic: Query Breakdown 25 Delay}
		\label{fig:actual-query-breakdown-25-seismic}
	\end{subfigure}
	\begin{subfigure}[c]{0.45\textwidth}
		\includegraphics[width=1\textwidth]	 {figures/Experiments/Dynamic/SEISMIC/35/breakdown_seismic_35.png}
		\caption{Seismic: Query Breakdown 35 Delay}
		\label{fig:actual-query-breakdown-35-seismic}
	\end{subfigure}
	\begin{subfigure}[c]{0.45\textwidth}
		\includegraphics[width=1\textwidth]	 {figures/Experiments/Dynamic/SEISMIC/50/breakdown_seismic_50.png}
		\caption{Seismic: Query Breakdown 50 Delay}
		\label{fig:actual-query-breakdown-50-seismic}
	\end{subfigure}
	\caption{Seismic: Query Performance Breakdown}
	\label{fig:dfresh-query-breakdown-seismic}
\end{figure*}

% %%%%%%%%%%%%%%%%%%%%%%%%%%%%%%%% Query Progress %%%%%%%%%%%%%%%%%%%%%%%%%%%%%%%%%%%%%%
 \noindent{\textbf{Query Progress}}
%
Figures~\ref{fig:query-progress-25-seismic} to~\ref{fig:query-progress-50-seismic} 
illustrate the average query answering time as the index size grows. The meaning of
each bar remains the same as in the corresponding experiment on the random dataset.
As expected, increasing the delay allows all versions of DFreSh to maintain timely
query answering. Unlike previous cases, this dataset exhibits no unexpected behavior,
with patterns aligning well with our observations from earlier experiments.

\begin{figure*}
	\centering
	\begin{subfigure}[c]{0.45\textwidth}
		\includegraphics[width=1\textwidth]   {figures/Experiments/Dynamic/SEISMIC/batch_answering/25/average_query_time_per_batch_version_999777015_10485760_10_delay[25].png}
		\caption{Seismic: Logical TS Buffers}
		\label{fig:logical-ts-25-seismic}
	\end{subfigure}
	\begin{subfigure}[c]{0.45\textwidth}
		\includegraphics[width=1\textwidth]	 {figures/Experiments/Dynamic/SEISMIC/batch_answering/25/average_query_time_per_batch_version_999777018_10485760_10_delay[25].png}
		\caption{Seismic: System TS Buffers}
		\label{fig:system-ts-25-seismic}
	\end{subfigure}
	\begin{subfigure}[c]{0.45\textwidth}
		\includegraphics[width=1\textwidth]	 {figures/Experiments/Dynamic/SEISMIC/batch_answering/25/average_query_time_per_batch_version_999777016_10485760_10_delay[25].png}
		\caption{Seismic: Logical TS No Buffers}
		\label{fig:logical-ts-no-25-seismic}
	\end{subfigure}
	\begin{subfigure}[c]{0.45\textwidth}
		\includegraphics[width=1\textwidth]	 {figures/Experiments/Dynamic/SEISMIC/batch_answering/25/average_query_time_per_batch_version_999777017_10485760_10_delay[25].png}
		\caption{Seismic: System TS No Buffers}
		\label{fig:system-ts-no-25-seismic}
	\end{subfigure}
	\caption{Seismic: Query Progress 25 delay}
	\label{fig:query-progress-25-seismic}
\end{figure*}
% %%%%%%%%%%%%%%%%%%%%%%%%%%%%%% Mesaio Delay %%%%%%%%%%%%%%%%%%%%%%%%%%%%%%%%%%%%%%%%%%%%%%%
\begin{figure*}
	\centering
	\begin{subfigure}[c]{0.45\textwidth}
		\includegraphics[width=1\textwidth]   {figures/Experiments/Dynamic/SEISMIC/batch_answering/35/average_query_time_per_batch_version_999777015_10485760_10_delay[35].png}
		\caption{Seismic: Logical TS Buffers}
		\label{fig:logical-ts-35-seismic}
	\end{subfigure}
	\begin{subfigure}[c]{0.45\textwidth}
		\includegraphics[width=1\textwidth]	 {figures/Experiments/Dynamic/SEISMIC/batch_answering/35/average_query_time_per_batch_version_999777018_10485760_10_delay[35].png}
		\caption{Seismic: System TS Buffers}
		\label{fig:system-ts-35-seismic}
	\end{subfigure}
	\begin{subfigure}[c]{0.45\textwidth}
		\includegraphics[width=1\textwidth]	 {figures/Experiments/Dynamic/SEISMIC/batch_answering/35/average_query_time_per_batch_version_999777016_10485760_10_delay[35].png}
		\caption{Seismic: Logical TS No Buffers}
		\label{fig:logical-ts-no-35-seismic}
	\end{subfigure}
	\begin{subfigure}[c]{0.45\textwidth}
		\includegraphics[width=1\textwidth]	 {figures/Experiments/Dynamic/SEISMIC/batch_answering/35/average_query_time_per_batch_version_999777017_10485760_10_delay[35].png}
		\caption{Seismic: System TS No Buffers}
		\label{fig:system-ts-no-35-seismic}
	\end{subfigure}
	\caption{Seismic: Query Progress 35 delay}
	\label{fig:query-progress-35-seismic}
\end{figure*}
% %%%%%%%%%%%%%%%%%%%%%%%%%% Largest%%%%%%%%%%%%%%%%%%%%%%%%%%%%%%%%%%%%%%%%%%%%%%%%%%%%%%%%
\begin{figure*}
	\centering
	\begin{subfigure}[c]{0.45\textwidth}
		\includegraphics[width=1\textwidth]   {figures/Experiments/Dynamic/SEISMIC/batch_answering/50/average_query_time_per_batch_version_999777015_10485760_10_delay[50].png}
		\caption{Seismic: Logical TS Buffers}
		\label{fig:logical-ts-50-seismic}
	\end{subfigure}
	\begin{subfigure}[c]{0.45\textwidth}
		\includegraphics[width=1\textwidth]	 {figures/Experiments/Dynamic/SEISMIC/batch_answering/50/average_query_time_per_batch_version_999777018_10485760_10_delay[50].png}
		\caption{Seismic: System TS Buffers}
		\label{fig:system-ts-50-seismic}
	\end{subfigure}
	\begin{subfigure}[c]{0.45\textwidth}
		\includegraphics[width=1\textwidth]	 {figures/Experiments/Dynamic/SEISMIC/batch_answering/50/average_query_time_per_batch_version_999777016_10485760_10_delay[50].png}
		\caption{Seismic: Logical TS No Buffers}
		\label{fig:logical-ts-no-50-seismic}
	\end{subfigure}
	\begin{subfigure}[c]{0.45\textwidth}
		\includegraphics[width=1\textwidth]	 {figures/Experiments/Dynamic/SEISMIC/batch_answering/50/average_query_time_per_batch_version_999777017_10485760_10_delay[50].png}
		\caption{Seismic: System TS No Buffers}
		\label{fig:system-ts-no-50-seismic}
	\end{subfigure}
	\caption{Seismic: Query Progress 50 delay}
	\label{fig:query-progress-50-seismic}
\end{figure*}

%%%%%%%%%%%%%%%%%%%%%%%%%% Latency %%%%%%%%%%%%%%%%%%%%%%%%%%%%%%%%%%%%%%%%%%%%%%%
\noindent{\textbf{Latency}}
%
Figure~\ref{fig:query-latency-seismic} presents the average query latency for the Seismic
dataset. As expected, its latency follows a similar trend to that of the Astro dataset,
given that both are real datasets. The calculation method remains the same, as described in
previous sections, and the observed pattern is consistent with our expectations.

\begin{figure*}
	\centering
	\begin{subfigure}[c]{0.45\textwidth}
		\includegraphics[width=1\textwidth]   {figures/Experiments/Dynamic/SEISMIC/average_latency_SEISMIC.png}
	\end{subfigure}
	\caption{Seismic: Query Latency}
	\label{fig:query-latency-seismic}
\end{figure*}





% %BIB
\bibliographystyle{plain}
\cleardoublepage
\addcontentsline{toc}{chapter}{Bibliography}
\bibliography{bib/thesis}



\end{document}
